\section{Algorithms}
\label{sec:algorithms}

%-------------------------------------------------------------------------------
\subsection{Breadth-First Search (BFS)}
%-------------------------------------------------------------------------------
\label{sec:bfs}

The breadth-first search (BFS)
builds on the observation that a vector-matrix multiplication $\grbv{f}\grbm{A}$ expresses
the navigation from the nodes selected by vector $\grbv{f}$ in the graph represented
by $\grbm{A}$.

A direction-optimizing push/pull BFS \cite{DBLP:conf/sc/BeamerAP12} is simple
to express in GraphBLAS \cite{DBLP:conf/icpp/YangBO18}.  If $\grbm{A}$ is held by row,
then $\grbm{fA}$ is a push step, while $\grbm{B}\grbt\grbv{f}$ is a pull step, where
$\grbm{B}=\grbm{A}\grbt$ is the explicit transpose of $\grbm{A}$, also held by row.
Other \grb libraries, \eg GraphBLAST, store both directions and perform
direction-optimization automatically~\cite{DBLP:journals/corr/abs-1908-01407}.

The GraphBLAS BFS relies on the $\grbanysecondi$ %ANY-SECONDI
semiring to compute a single step,
$\grbv{q} \grbmask{\grbneg \grbstr{\grbv{p}}} = \grbv{q}\grbt\grbm{A}$, where $\grbv{q}$ is the current frontier
(using $\grbv{q}$ as short for queue),
$\grbv{p}$ is the parent vector, and $\grbm{A}$ is the adjacency matrix.

Consider a matrix multiply for conventional linear algebra, where the $\grbplus$ %PLUS
monoid sums a set of $t$ entries to obtain a single scalar for computing
$c_{ij} = \sum a_{ik} b_{kj}$ in the matrix multiply $\grbm{C} = \grbm{A}\grbm{B}$.  The $\grbany$ %ANY
monoid performs the reduction of $t$ entries to a single number by merely selecting
any one of the $t$ entries as the result $c_{ij}$.  The selection is done
non-deterministically, allowing for a benign race condition.  In the BFS, this
corresponds to selecting any valid parent of a newly discovered node.  Indeed,
the creation of the $\grbany$ %ANY
operator was inspired by Scott Beamer's \verb'bfs.cc'
method in the GAP benchmark, which has the same benign race condition.  The $\grbany$ %ANY
monoid translates the concept of this benign race condition to construct a
valid BFS tree into a linear algebraic operation, suitable for implementation
in GraphBLAS.

The $\grbsecondi$ %SECONDI
operator is the multiplicative operator in the $\grbanysecondi$ %ANY-SECONDI
semiring, where the result of $a_{ik} b_{kj}$ is simply the index $k$ in the
semiring for $\grbm{C} = \grbm{A}\grbm{B}$.  This gives the id of the parent node for a newly
discovered node in the next frontier.  The $\grbany$ %ANY
monoid then selects any valid
parent $k$.

\begin{algorithm}[htb]
    \caption{Parents BFS.}
    \label{alg:bfs-parents}
    \DontPrintSemicolon
    \KwIn{$\grbm{A}, \grbs{startVertex}$}
    \Fn{ParentsBFS}{
        $\grbv{p}(\grbs{startVertex}) = \grbs{startVertex}$ \;
        $\grbv{q}(\grbs{startVertex}) = \grbs{startVertex}$ \;
        \For{$\grbs{level} = 1$ \KwTo $\grbnrows{\grbm{A}}-1$}{
            $\grbv{q}\grbt \grbmaskreplace{\grbneg \grbstr{\grbv{p}\grbt}} = \grbv{q}\grbt \grbanysecondi \grbm{A}$ \;
            $\grbv{p} \grbmask{\grbstr{\grbv{q}}} = \grbv{q}$ \;
            \If{$\grbnvals{\grbv{q}} = 0$}{return}
        }
    }
\end{algorithm}

\begin{algorithm}[htb]
    \caption{Direction Optimizing Parent BFS.}
    \label{alg:bfs-parents-do}
    \DontPrintSemicolon
    \KwIn{$\grbm{A}, \grbm{A}\grbt, \grbs{startVertex}$}
    \Fn{DirectionOptimizingBFS}{
        $\grbv{q}(\grbs{startVertex}) = 0$ \;
        \For{$\grbs{level} = 1$ \KwTo $\grbnrows{\grbm{A}}-1$}{
            \If{$\mathit{Push}(\grbm{A}, \grbv{q})$}{ %\Comment{Decide to push/pull}
                $\grbv{q}\grbt \grbmaskreplace{\grbneg \grbstr{\grbv{p}\grbt}} = \grbv{q}\grbt \grbanysecondi \grbm{A}$
            }
            \Else{
                $\grbv{q} \grbmaskreplace{\grbneg \grbstr{\grbv{p}}} = \grbm{A}\grbt \grbanysecondi \grbv{q}$
            }
            $\grbv{p} \grbmask{\grbstr{\grbv{q}}} = \grbv{q}$ \;
            \If{$\grbnvals{\grbv{q}} = 0$}{return}
        }
    }
\end{algorithm}


The push-only BFS is shown in
\autoref{alg:bfs-parents}, while the push/pull BFS is \autoref{alg:bfs-parents-do}.

% $\grbanysecondi$

%-------------------------------------------------------------------------------
\subsection{Betweenness Centrality}
%-------------------------------------------------------------------------------
\label{sec:bc}

\begin{algorithm}[htb]
	\caption{Betweenness centrality.}
	\label{alg:bc}
	%\KwData{...}
	%\KwResult{...}
	\Fn{BrandesBC}{
		\Comment{$\grbm{P}(k,j)$ = \# paths from $k$th source to node $j$}
		$\grbm{P} \grbassign \{ \grba{s}, [1, 1, \ldots, 1] \}$ \;
		\Comment{$\grbm{F}$: \# paths in the current frontier}
		$\grbm{F} \grbassign \{ \grba{s}, [1, 1, \ldots, 1] \}$ \;
                $\grbm{F} \grbmaskreplace{\grbneg\grbstr{\grbm{P}}} =\grbm{F} \grbplusfirst \grbm{A}$  \;

		\Comment{BFS phase}
		\For{$\grbs{d} = 0$ \KwTo $\grbnrows{\grbm{A}}$}{
			$\grbnewmatrix{\grbm{S}[\grbs{d}]}{\grbbool}{}{\grbs{ns}}{\grbs{n}} $ \;
			$\grbm{S}[\grbs{d}]\grbmask{\grbstr{\grbm{F}}} = 1$ \Comment{$\grbm{S}[d]$ = pattern of $\grbm{F}$}
			$\grbm{P} \grbaccumeq{+} \grbm{F}$ \;
			$\grbm{F}\grbmaskreplace{\grbneg\grbstr{\grbm{P}}} = \grbm{F} \grbplusfirst \grbm{A}$ \;
                        \If{$\grbnvals{\grbm{F}} = 0$}{break}
		}

		\Comment{Backtrack phase}
		$\grbnewmatrix{\grbm{B}}{\grbfloat}{64}{\grbs{ns}}{\grbs{n}}$ \;
		$\grbm{B}(:) = 1.0$ \;
		$\grbnewmatrix{\grbm{W}}{\grbfloat}{64}{\grbs{ns}}{\grbs{n}}$ \;

		\For{$\grbs{i} = \grbs{d} - 1$ \KwDownto $0$}{
			$\grbm{W}\grbmaskreplace{\grbstr{\grbm{S}[\grbs{i}]}} = \grbm{B}  \grbdiv_\cap \grbm{P}$ \;
			$\grbm{W}\grbmaskreplace{\grbstr{\grbm{S}[\grbs{i} - 1]}} = \grbm{W} \grbplusfirst \grbm{A}\grbt$ \;
			$\grbm{B} \grbaccumeq{+} \grbm{W} \times_\cap \grbm{P}$ \;
		}

		\Comment{$\grbv{centrality}(j) = \sum_i (\grbm{B}(i,j) - 1)$}
		$\grbv{centrality}(:) = -\grbs{ns}$ \;
		$\grbv{centrality} \grbaccumeq{+} \grbreduce{+}{\grbs{i}}{\grbm{B}}{\grbs{i},:} $ \;
	}
\end{algorithm}


\autoref{alg:bc}

also does push/pull as discussed in \autoref{sec:extensions}

\subsection{PageRank}
\label{sec:pagerank}

% The PR in GAP does not work well if there are dangling vertices in the graph.
% The Graphalytics benchmark has a PageRank variant which avoids this problem~\cite{DBLP:journals/corr/abs-2011-15028}.

\begin{algorithm}[htb]
    \caption{PageRank (as specified in the GAPBS).}
    \label{alg:pagerank}
    \KwData{$\grbm{A} \in \grbbool^{n \times n}$ \Comment*{adjacency matrix}}
    \KwDataXX{$\grbs{damping}$ \Comment*{damping factor}}
    \KwDataXX{$\grbs{tol}$ \Comment*{stopping tolerance}}
    \KwDataXX{$\grbs{itermax}$\Comment*{maximum number of iterations}}
    \KwResult{$\grbv{r} \in \grbfloat^n$}
    \Fn{PageRank}{
        % $\grbv{pr}(:) = 1 / n$ \;
        % $\grbv{outdegrees} = [\grbplus_{\grbs{j}} \grbm{A}(:, \grbs{j})] $ \;
        % %$\grbv{nondangling} = [\grblor_{\grbs{j}} \grbm{A}(:, \grbs{j})] $ \;

        % \For{$\grbs{k} = 1$ \KwTo $\grbs{numIterations}$}{
        %     $\grbv{importance} = \grbv{pr} \grbdiv \grbv{outdegrees}$ \;
        %     $\grbv{importance} = \grbf{times}{\grbv{importance}, \grbs{\grbalpha}} $ \Comment*{apply the $\grbf{times}{x, s} = x \cdot s$ operator}
        %     $\grbv{importance} = \grbv{importance} \grbplustimes \grbm{A} $ \;

        %     $\grbv{danglingVertexRanks} \grbmask{\grbneg{\grbv{outdegrees}}} = \grbv{pr}(:) $ \;
        %     $\grbs{totalDanglingRank} = \grbfrac{\grbs{\grbalpha}}{\grbs{n}} \grbtimes \grbreduce{\grbplus}{\grbs{i}}{\grbv{danglingVertexRanks}}{\grbs{i}} $ \;

        %     $\grbv{pr} = \grbfrac{1-\grbs{\grbalpha}}{\grbs{n}} \grbplus \grbs{totalDanglingRank} $ \;
        %     $\grbv{pr} = \grbv{pr} \grbplus \grbv{importance} $ \;
        % }
        $\grbs{teleport} = \frac{1 - \alpha}{n}$ \;
        $\grbv{r}(0:n-1) = \frac{1}{n}$, $\grbv{t} = \grbfloat^n$ \;
        $\grbv{d_{out}} = \grbreduce{+}{j}{\grbm{A}}{:, j}$ \Comment{precomputed rowdegree}
        $\grbv{d} = \grbv{d_{out}} \grbewisemult{\grbdiv} \grbs{damping}  $ \Comment{prescale with damping}

        %$\grbv{pr}(:) = 1 / n$ \;
        
        \For{$\grbs{k} = 1$ \KwTo $\grbs{numIterations}$}{
            swap $\grbv{t}$ and $\grbv{r}$ \Comment{$\grbv{t}$ is now the prior rank}
            $\grbv{w} = \grbv{t} \grbewisemult{\grbdiv} \grbv{d}$ \;
            $\grbv{r}(0:n-1) = \grbs{teleport}$ \;
            $\grbv{r} += \grbm{A}\grbt \grbv{w}$ \;
            $\grbv{t} -= \grbv{r}$ \;
            $\grbv{t} = abs(\grbv{t})$ \;
            \If{$\grbreduce{+}{j}{\grbv{t}}{:} < \grbs{tol}$}{return}
        }
    }
\end{algorithm}


\autoref{alg:pagerank}

\subsection{SSSP}
\label{sec:sssp}

\begin{algorithm}[htb]
	\caption{SSSP (delta-stepping).}
	\label{alg:sssp-delta-stepping}
	\KwData{\;
		$\quad \grbm{A}, \grbm{A_H}, \grbm{A_L} \in \grbmatrixtype{\grbfloat}{}{\grbcnt{V}}{\grbcnt{V}} $ \;
		$\quad \grbs{s}, \grbs{i} \in \grbscalartype{\grbuint}{} $ \;
		$\quad \Delta \in \grbscalartype{\grbfloat}{} $ \;
		$\quad \grbv{t}, \grbv{t_{Req}} \in \grbvectortype{\grbfloat}{}{\grbcnt{V}} $ \;
		$\quad \grbv{t_{B_i}}, \grbv{e} \in \grbvectortype{\grbuint}{}{\grbcnt{V}} $ \; % e was S in the original paper, changed to lowercase for consistency
	}
	%\KwResult{...}
	\Fn{DeltaStepping}{
		$\grbm{A_L} = \grbm{A}\grbselect{0 < \grbm{A} \leq \Delta} $ \;
		$\grbm{A_H} = \grbm{A}\grbselect{\Delta < \grbm{A}} $ \;
		$\grbv{t}(:) = \infty $ \;
		$\grbv{t}(\grbs{s}) = 0 $ \;
		\While{$\grbnvals{ \grbv{t}\grbselect{\grbs{i} \Delta \leq \grbv{t}} } \neq 0$}{
			$\grbs{s} = 0 $ \;
			$\grbv{t_{B_i}} = \grbv{t} \grbselect{\grbs{i} \Delta \leq \grbv{t} < (\grbs{i} + 1) \Delta}$ \;
			\While{$\grbv{t_{B_i}} \neq 0$}{
				$\grbv{t_{Req}} = \grbm{A_L\grbt} \grbminplus (\grbv{t} \grbtimes \grbv{t_{B_i}})$ \;
				$\grbv{e} = \grbv{t} \grbselect{0 < \grbv{e} \grbplus \grbv{t_{B_i}} }$ \;
				$\grbv{t_{B_i}} = \grbv{t} \grbselect{\grbs{i} \Delta \leq \grbv{t_{Req}} < (\grbs{i} + 1) \Delta} \grbtimes (\grbv{t_{Req}} \grbewiseadd{\grbmin} \grbv{t} ) $\;
				$\grbv{t} = \grbv{t} \grbewiseadd{\grbmin} \grbv{t_{Req}}$ \;
			}
			$\grbv{t_{Req}} = \grbm{A_H\grbt} \grbminplus (\grbv{t} \grbtimes \grbv{e})$ \;
			$\grbv{t} = \grbv{t} \grbewiseadd{\grbmin} \grbv{t_{Req}}$ \;
			$\grbs{i} = \grbs{i} + 1 $ \;
	}
	}
\end{algorithm}


\autoref{alg:sssp-delta-stepping}

\subsection{Triangle Count}
\label{sec:triangle-count}

\begin{algorithm}[htb]
	\caption{Triangle count (``SandiaDot'' variant).}
	\label{alg:triangle-count-sandiadot}
	\KwData{$\grbm{A} \in \grbbool^{n \times n}$}
	\KwResult{$t \in \grbuint$}
	\Fn{TriangleCount}{
		$\grbm{L} = \grbtril{\grbm{A}}$ \;
		$\grbm{U} = \grbtriu{\grbm{A}}$ \;
		$\grbm{C}\grbmask{\grbstr{\grbm{L}}} = \grbm{L} \grbarithmeticplustimes \grbm{U}\grbt$ \;
		$\grbs{t} = \grbreduce{\grbarithmeticplus}{\grbs{i}\grbs{j}}{\grbm{C}}{\grbs{i}, \grbs{j}}$ \;
	}
\end{algorithm}


\autoref{alg:triangle-count-sandiadot}

\subsection{Connected Components}
\label{sec:connected-components}

\begin{algorithm}[htb]
	\caption{Connected components (FastSV).}
	\label{alg:fastsv}
	%\KwData{...}
	%KwResult{...}
	\Fn{FastSV}{
        $\grbs{n} = \grbnrows{\grbm{A}}$ \;
        $\grbv{gf} = \grbv{f}$ \;
        $\grbv{dup} = \grbv{gf}$ \;
        $\grbv{mngf} = \grbv{gf}$ \;
        $\{ \grba{i}, \grba{x} \} \grbassign \grbv{f}$ \;
        \Repeat{$\grbs{sum} == 0$}{
            \Comment{Step 1: Stochastic hooking}
            $\grbv{mngf} = \grbv{mngf} \grbmin \grbm{A} $ \;
            $\grbv{mngf} = \grbv{mngf} \grbsecondmin \grbv{gf}$ \;
            $\grbv{f}(\grba{x}) = \grbv{f} \grbmin \grbv{mngf} $ \;
            \Comment{Step 2: Aggressive hooking}
            $\grbv{f} = \grbv{f} \grbmin \grbv{mngf} $ \;
            \Comment{Step 3: Shortcutting}
            $\grbv{f} = \grbv{f} \grbmin \grbv{gf} $ \;
            \Comment{Step 4: Calculate grandparents}
            $\{ \grba{i},  \grba{x} \} \grbassign \grbv{f}$ \;
            $\grbv{gf} = \grbv{f}(\grba{x})$ \;
            \Comment{Step 5: Check termination}
            $\grbv{diff} = \grbv{dup} \neq \grbv{gf} $ \; % \neq or 'isne'?
            $\grbs{sum} = [\grbplus_{\grbs{i}} \grbv{diff}(\grbs{i}) ] $ \;
            $\grbv{dup} = \grbv{gf}$ \; %LAGRAPH_OK (GrB_assign (dup, 0, 0, gp, GrB_ALL, 0, 0));
        }
	}
\end{algorithm}


\autoref{alg:fastsv}

\hyphenation{Suite-Sparse}
\hyphenation{Graph-BLAS}
\hyphenation{Suite-Sparse-Graph-BLAS}

\newcommand{\suitesparse}{SuiteSparse\xspace}
\newcommand{\grb}{GraphBLAS\xspace}
\newcommand{\ssgrb}{SuiteSparse:GraphBLAS\xspace}
\newcommand{\gxb}{\ssgrb}
\newcommand{\lagraph}{LAGraph\xspace}
\newcommand{\pygrb}{pygraphblas\xspace}
\newcommand{\grblas}{grblas\xspace}

% Define new boolean flags using etoolbox ('\newbool' is similar to '\newtoggle').
% This workaround is needed as simply putting the newcommands inside 'IfFileExists' did not do the job
% as it broke with 'Illegal parameter number in definition of \reserved@a', a symptom probably caused
% by the lack of protection (\protect). Anyways, the workaround is actually cleaner.

%\newcommand{\grbreduce}[2]{\left[{#1}_j \, {#2}(:, j) \right]}
\ifbool{ascii}{ % ASCII mode ======================================================================
    \newcommand{\grbm}[1]{{\ifbool{colored}{\color{brown}}{}{\mathtt{#1}}}}% matrix
    \newcommand{\grbv}[1]{{\ifbool{colored}{\color{lilac}}{}{\mathtt{#1}}}}% vector
    \newcommand{\grba}[1]{{\ifbool{colored}{\color{gray}}{}{\mathtt{#1}}}}% array
    \newcommand{\grbs}[1]{{\ifbool{colored}{\color{blue}}{}{\mathtt{#1}}}}% scalar

    \newcommand{\grbstr}[1]{{\{#1\}}}
    \newcommand{\grbmask}[1]{<\! #1 \!>}
    \newcommand{\grbmaskreplace}[1]{<\!<\! #1 \!>\!>}
    \newcommand{\grbneg}{\texttt{!}}
    \newcommand{\grbassign}{\mathrel{\texttt{<-}}}
    \newcommand{\grbf}[2]{\grboperation{#1}(#2)}
    \newcommand{\grbreduce}[4]{[ {#1 #3} ]} % omit the indices
    \newcommand{\grbt}{\texttt{'}} % transpose
    \newcommand{\grbdiv}{\grbbinaryop{DIV}}
    \newcommand{\grbminus}{\grbbinaryop{MINUS}}
    \newcommand{\grbaccum}{\texttt{\ensuremath{+}}}
    \newcommand{\grbaccumeq}[1]{\mathbin{\texttt{\ensuremath{\ifstrempty{#1}{\grbaccum}{#1}=}}}}

    \newcommand{\grbplus}{\grbbinaryop{+}}
    \newcommand{\grbtimes}{\grbbinaryop{\times}}
    \newcommand{\grbapply}{\grbbinaryop{\odot}}

    \newcommand{\grbfrac}[2]{(#1)/(#2)}

    \newcommand{\grbbool}{\mathtt{bool}} % booleans
    \newcommand{\grbuint}{\mathtt{uint}} % unsigned integers
    \newcommand{\grbint}{\mathtt{int}}   % integers
    \newcommand{\grbfloat}{\mathtt{fp}}  % floats (?)

    \newcommand{\grbplaceholder}[1]{\mathsf{#1}}

    \newcommand{\grbscalartype}[2]{\mathtt{#1#2()}}
    \newcommand{\grbvectortype}[3]{\mathtt{#1#2(#3)}}
    \newcommand{\grbmatrixtype}[4]{\mathtt{#1#2(#3, #4)}}

    \newcommand{\grbnewscalar}[3]{\mathtt{#1 = \grbscalartype{#2}{#3}}}
    \newcommand{\grbnewvector}[4]{\mathtt{#1 = \grbvectortype{#2}{#3}{#4}}}
    \newcommand{\grbnewmatrix}[5]{\mathtt{#1 = \grbmatrixtype{#2}{#3}{#4}{#5}}}

    \newcommand{\grbalpha}{\mathtt{alpha}}
    \newcommand{\grboperator}[1]{\mathtt{#1}}

    \newcommand{\grbrange}[2]{#1:#2}
    \newcommand{\grbdontcare}{\_}

    \newcommand{\grboperationnoarg}[1]{\mathtt{#1}}

    \newcommand{\grbewiseadd}[1]{\grbbinaryop{#1[intersection]}}
    \newcommand{\grbewisemult}[1]{\grbbinaryop{#1[union]}}
}{ % LaTeX mode ===================================================================================
    \newcommand{\grbm}[1]{{\ifbool{colored}{\color{brown}}{}{\mathbf{#1}}}}% matrix
    \newcommand{\grbv}[1]{{\ifbool{colored}{\color{lilac}}{}{\mathbf{#1}}}}% vector
    \newcommand{\grba}[1]{{\ifbool{colored}{\color{gray}}{}{\textbf{\textit{#1}}}}}% array
    \newcommand{\grbs}[1]{{\ifbool{colored}{\color{blue}}{}{\mathit{#1}}}}% scalar

    \newcommand{\grbmask}[1]{\langle #1 \rangle}
    %\newcommand{\grbstr}[1]{{\{#1\}}}
    \newcommand{\grbstr}[1]{s(#1)}
    %\newcommand{\grbmaskreplace}[1]{\langle\!\langle #1 \rangle\!\rangle}
    \newcommand{\grbmaskreplace}[1]{\langle #1, \mathrm{r} \rangle}
    \newcommand{\grbneg}{\neg}

    % use the \mapsfrom symbol extracted from the stix package as suggested in https://tex.stackexchange.com/a/331899/71109
    \DeclareFontEncoding{LS1}{}{}
    \DeclareFontSubstitution{LS1}{stix}{m}{n}
    \DeclareSymbolFont{arrows1}{LS1}{stixsf}{m}{n}
    \global\let\mapsfrom\undefined % undefine \mapsfrom because some templates such as acmart already have it
    \DeclareMathSymbol{\mapsfrom}{\mathrel}{arrows1}{"AB}
    \newcommand{\grbassign}{\mapsfrom}

    \newcommand{\grbf}[2]{\grboperation{#1}{#2}}
    \newcommand{\grbreduce}[4]{[ {#1}_{#2}\, #3(#4) ]}
    \newcommand{\grbt}{^\mathsf{T}} %^{\top}} % transpose
    \newcommand{\grbdiv}{\grbbinaryop{\oslash}}
    \newcommand{\grbminus}{\grbbinaryop{\ominus}}
    \newcommand{\grbaccum}{\texttt{\ensuremath{\odot}}}
    \newcommand{\grbaccumeq}[1]{\mathbin{\ensuremath{\ifstrempty{#1}{\grbaccum}{#1}\!\!=}}}

    \newcommand{\grbplus}{\oplus}
    \newcommand{\grbtimes}{\otimes}
    \newcommand{\grbapply}{\odot}
    
    \newcommand{\grbfrac}[2]{\frac{#1}{#2}}

    \newcommand{\grbbool}{\mathbb{B}}  % booleans
    \newcommand{\grbuint}{\mathbb{N}}  % unsigned integers
    \newcommand{\grbint}{\mathbb{Z}}   % integers
    \newcommand{\grbfloat}{\mathbb{Q}} % floats (?)
    
    \newcommand{\grbplaceholder}[1]{\mathsf{#1}}

    \newcommand{\grbscalartype}[2]{#1_{#2}}
    \newcommand{\grbvectortype}[3]{#1_{#2}^{#3}}
    \newcommand{\grbmatrixtype}[4]{#1_{#2}^{#3 \times #4}}

    \newcommand{\grbnewscalar}[3]{\text{let: } #1 \in \grbscalartype{#2}{#3}}
    \newcommand{\grbnewvector}[4]{\text{let: } #1 \in \grbvectortype{#2}{#3}{#4}}
    \newcommand{\grbnewmatrix}[5]{\text{let: } #1 \in \grbmatrixtype{#2}{#3}{#4}{#5}}

    \newcommand{\grbalpha}{\alpha}
    \newcommand{\grboperator}[1]{\mathsf{#1}}

    \newcommand{\grbrange}[2]{#1 \! : \! #2}
    \newcommand{\grbdontcare}{\textvisiblespace}

    \newcommand{\grboperationnoarg}[1]{\mathrm{#1}}

    \newcommand{\grbewiseadd}[1]{\grbbinaryop{#1_\cup}}
    \newcommand{\grbewisemult}[1]{\grbbinaryop{#1_\cap}}
}

% do not lange/rangle for tuples as it is already used for masks
% do not use grbtuple for the time being
%\newcommand{\grbtuple}[1]{( #1 )}
\newcommand{\tuple}[1]{(#1)}

% trying to avoid too much syntax (e.g. using wedge/vee symbols for LAND/LOR)
%\newcommand{\grblorland}{\lor\!.\!\land}

\newcommand{\grbsemiringops}[2]{\mathbin{\grboperator{#1.#2}}}
\newcommand{\grbplustimes}{\grbsemiringops{\grbplus}{\grbtimes}}

\newcommand{\grbanypair}{\grbsemiringops{any}{pair}}
\newcommand{\grbanyfirst}{\grbsemiringops{any}{first}}
\newcommand{\grbanysecond}{\grbsemiringops{any}{second}}
\newcommand{\grbanyfirstj}{\grbsemiringops{any}{firstj}}
\newcommand{\grbanyfirstjone}{\grbsemiringops{any}{firstj1}}
\newcommand{\grbminfirstj}{\grbsemiringops{min}{firstj}}
\newcommand{\grbminfirstjone}{\grbsemiringops{min}{firstj1}}
\newcommand{\grbanysecondi}{\grbsemiringops{any}{secondi}}
\newcommand{\grbanysecondione}{\grbsemiringops{any}{secondi1}}
\newcommand{\grbminsecondi}{\grbsemiringops{min}{secondi}}
\newcommand{\grbminsecondione}{\grbsemiringops{min}{secondi1}}
\newcommand{\grblorland}{\grbsemiringops{lor}{land}}
\newcommand{\grbminplus}{\grbsemiringops{min}{plus}}
\newcommand{\grbmaxplus}{\grbsemiringops{max}{plus}}
\newcommand{\grbmaxfirst}{\grbsemiringops{max}{first}}
\newcommand{\grbminfirst}{\grbsemiringops{min}{first}}
\newcommand{\grbminsecond}{\grbsemiringops{min}{second}}
\newcommand{\grbmaxsecond}{\grbsemiringops{max}{second}}
\newcommand{\grbsecondmin}{\grbsemiringops{second}{min}}
\newcommand{\grbsecondmax}{\grbsemiringops{second}{max}}
\newcommand{\grbarithmeticplustimes}{\grbsemiringops{plus}{times}} % not necessary because this is the default

\newcommand{\grbbinaryop}[1]{\mathbin{\grboperator{#1}}}
\newcommand{\grbany}{\grbbinaryop{any}}
\newcommand{\grbpair}{\grbbinaryop{pair}}
\newcommand{\grbland}{\grbbinaryop{land}}
\newcommand{\grblor}{\grbbinaryop{lor}}
\newcommand{\grbband}{\grbbinaryop{band}}
\newcommand{\grbbor}{\grbbinaryop{bor}}
\newcommand{\grbmin}{\grbbinaryop{min}}
\newcommand{\grbmax}{\grbbinaryop{max}}
\newcommand{\grbfirst}{\grbbinaryop{first}}
\newcommand{\grbsecond}{\grbbinaryop{second}}
\newcommand{\grbfirsti}{\grbbinaryop{firsti}}
\newcommand{\grbsecondi}{\grbbinaryop{secondi}}
\newcommand{\grbfirstione}{\grbbinaryop{firsti1}}
\newcommand{\grbsecondione}{\grbbinaryop{secondi1}}
\newcommand{\grbfirstj}{\grbbinaryop{firstj}}
\newcommand{\grbsecondj}{\grbbinaryop{secondj}}
\newcommand{\grbfirstjone}{\grbbinaryop{firstj1}}
\newcommand{\grbsecondjone}{\grbbinaryop{secondj1}}
\newcommand{\grbarithmeticplus}{\grbbinaryop{plus}} % usually not necessary because this is the default addition
\newcommand{\grbarithmetictimes}{\grbbinaryop{times}} % usually not necessary because this is the default multiplication
\newcommand{\grbisne}{\grbbinaryop{isne}}
\newcommand{\grbplustext}{\grbbinaryop{plus}}
\newcommand{\grbtimestext}{\grbbinaryop{times}}

\newcommand{\grbgenericop}{\grboperator{op}}

% boolean values
\newcommand{\grbbooleanvalue}[1]{\mathtt{#1}}
\newcommand{\grbtrue}{\grbbooleanvalue{TRUE}}
\newcommand{\grbfalse}{\grbbooleanvalue{FALSE}}
\newcommand{\grbT}{\grbbooleanvalue{T}}
\newcommand{\grbF}{\grbbooleanvalue{F}}
\newcommand{\grbstring}{\textrm{String}}
\newcommand{\grbdate}{\textrm{Date}}

% cardinality / count
\newcommand{\grbcnt}[1]{| #1 |}

% operations
\newcommand{\grboperation}[2]{\grboperationnoarg{#1}(#2)}
\newcommand{\grbnrows}[1]{\grboperation{nrows}{#1}}
\newcommand{\grbncols}[1]{\grboperation{ncols}{#1}}
\newcommand{\grbnvals}[1]{\grboperation{nvals}{#1}}
\newcommand{\grbclear}[1]{\grboperation{clear}{#1}}
\newcommand{\grbdiag}[1]{\grboperation{diag}{#1}}
\newcommand{\grbselect}[1]{\grbmask{#1}}
\newcommand{\grbkron}[1]{\grboperation{kron}{#1}}
\newcommand{\grbtril}[1]{\grboperation{tril}{#1}}
\newcommand{\grbtriu}[1]{\grboperation{triu}{#1}}
\newcommand{\grbondiag}[1]{\grboperation{ondiag}{#1}}
\newcommand{\grboffdiag}[1]{\grboperation{offdiag}{#1}}

\newcommand{\grbind}[1]{\mathrm{ind}(#1)}

\section{GraphBLAS Theory and Notation}
\label{sec:notation}

%\setlength{\tabcolsep}{1.9pt}
%\renewcommand\arraystretch{0.85}

\begin{table*}[htbp]
    \centering
    \begin{tabular}{llr@{}ll}
        \toprule
        \multicolumn{1}{c}{\bf operation/method} & \multicolumn{1}{c}{\bf name}                        & \multicolumn{2}{c}{\bf notation}                                                                       & \multicolumn{1}{c}{\bf comment}                                                     \\
        % <operations>
        \midrule
        \tt mxm                                  & matrix-matrix multiplication                        & $\grbm{C} \grbmask{\grbm{M}}        $                                                                  & $\grbaccumeq{} \grbm{A} \grbplustimes \grbm{B}$                                     \\
        \tt vxm                                  & vector-matrix multiplication                        & $\grbv{\grbv{w}} \grbmask{\grbv{m}} $                                                                  & $\grbaccumeq{} \grbv{u} \grbplustimes \grbm{A}$                                     \\
        \tt mxv                                  & matrix-vector multiplication                        & $\grbv{w} \grbmask{\grbv{m}}        $                                                                  & $\grbaccumeq{} \grbm{A} \grbplustimes \grbv{u}$                                     \\
        \midrule
        \multirow{2}{*}{\tt eWiseAdd}            & element-wise addition                               & $\grbm{C} \grbmask{\grbm{M}} $                                                                         & $\grbaccumeq{} \grbm{A} \grbewiseadd{\grbgenericop} \grbm{B}$                       \\
                                                 & set union of patterns                               & $\grbv{w} \grbmask{\grbv{m}} $                                                                         & $\grbaccumeq{} \grbv{u} \grbewiseadd{\grbgenericop} \grbv{v}$                       \\
        \midrule
        \multirow{2}{*}{\tt eWiseMult}           & element-wise multiplication                         & $\grbm{C} \grbmask{\grbm{M}} $                                                                         & $\grbaccumeq{} \grbm{A} \grbewisemult{\grbgenericop} \grbm{B}$                      \\
                                                 & set intersection of patterns                        & $\grbv{w} \grbmask{\grbv{m}} $                                                                         & $\grbaccumeq{} \grbv{u} \grbewisemult{\grbgenericop} \grbv{v}$                      \\
        \midrule
        \multirow{4}{*}{\tt extract}             & extract submatrix                                   & $\grbm{C} \grbmask{\grbm{M}} $                                                                         & $\grbaccumeq{} \grbm{A}(\grba{i}, \grba{j})$                                        \\
                                                 & extract column vector                               & $\grbv{w} \grbmask{\grbv{m}} $                                                                         & $\grbaccumeq{} \grbv{A}(:, \grbs{j})$                                               \\
                                                 & extract row vector                                  & $\grbv{w} \grbmask{\grbv{m}} $                                                                         & $\grbaccumeq{} \grbv{A}(\grbs{i}, :)$                                               \\
                                                 & extract subvector                                   & $\grbv{w} \grbmask{\grbv{m}} $                                                                         & $\grbaccumeq{} \grbv{u}(\grba{i})$                                                  \\
        \midrule
        \multirow{4}{*}{\tt assign}              & assign matrix to submatrix with mask for $\grbm{C}$ & $\grbm{C} \grbmask{\grbm{M}} (\grba{i},\grba{j}) $                                                     & $\grbaccumeq{} \grbm{A}$                                                            \\
                                                 & assign scalar to submatrix with mask for $\grbm{C}$ & $\grbm{C} \grbmask{\grbm{M}} (\grba{i},\grba{j}) $                                                     & $\grbaccumeq{} \grbs{s}$                                                            \\
                                                 & assign vector to subvector with mask for $\grbv{w}$ & $\grbv{w} \grbmask{\grbv{m}} (\grba{i}) $                                                              & $\grbaccumeq{} \grbv{u}$                                                            \\
                                                 & assign scalar to subvector with mask for $\grbv{w}$ & $\grbv{w} \grbmask{\grbv{m}} (\grba{i}) $                                                              & $\grbaccumeq{} \grbs{s}$                                                            \\
        % \midrule
        % \multirow{4}{*}{\tt subassign (GxB)} & assign matrix to submatrix with submask for $\grbm{C}(\grba{i},\grba{j})$ & $\grbm{C}(\grba{i},\grba{j}) \grbmask{\grbm{M}} $                                                      & $\grbaccumeq{} \grbm{A}$                                                            \\
        %                                      & assign scalar to submatrix with submask for $\grbm{C}(\grba{i},\grba{j})$ & $\grbm{C}(\grba{i},\grba{j}) \grbmask{\grbm{M}} $                                                      & $\grbaccumeq{} \grbs{s}$                                                            \\
        %                                      & assign vector to subvector with submask for $\grbv{w}(\grba{i})$          & $\grbv{w}(\grba{i}) \grbmask{\grbv{m}} $                                                               & $\grbaccumeq{} \grbv{u}$                                                            \\
        %                                      & assign scalar to subvector with submask for $\grbv{w}(\grba{i})$          & $\grbv{w}(\grba{i}) \grbmask{\grbv{m}} $                                                               & $\grbaccumeq{} \grbs{s}$                                                            \\
        \midrule
        \multirow{2}{*}{\tt apply}               & \multirow{2}{*}{apply unary operator}               & $\grbm{C} \grbmask{\grbm{M}} $                                                                         & $\grbaccumeq{} \grbf{f}{\grbm{A}, \grbs{k}}$                                        & \multirow{2}{*}{$k$: thunk} \\
                                                 &                                                     & $\grbv{w} \grbmask{\grbv{m}} $                                                                         & $\grbaccumeq{} \grbf{f}{\grbv{u}, \grbs{k}}$                                                 \\
        \midrule
        \multirow{2}{*}{\tt select}              & \multirow{2}{*}{apply select operator}              & $\grbm{C} \grbmask{\grbm{M}} $                                                                         & $\grbaccumeq{} \grbm{A}\grbselect{\grbf{f}{\grbm{A}, \grbs{k}}}$                    & \multirow{2}{*}{$k$: thunk} \\
                                                 &                                                     & $\grbv{w} \grbmask{\grbv{m}} $                                                                         & $\grbaccumeq{} \grbv{u}\grbselect{\grbf{f}{\grbv{u}, \grbs{k}}}$                    \\
        \midrule
        \multirow{3}{*}{\tt reduce}              & reduce matrix to column vector                      & $\grbv{w} \grbmask{\grbv{m}} $                                                                         & $\grbaccumeq{} \grbreduce{\grbplus}{\grbs{j}}{\grbm{A}}{:,\grbs{j}}$                \\
                                                 & reduce matrix to scalar                             & $\grbs{s} $                                                                                            & $\grbaccumeq{} \grbreduce{\grbplus}{\grbs{i}, \grbs{j}}{\grbm{A}}{\grbs{i},\grbs{j}}$ \\
                                                 & reduce vector to scalar                             & $\grbs{s} $                                                                                            & $\grbaccumeq{} \grbreduce{\grbplus}{\grbs{i}}{\grbm{u}}{\grbs{i}}$                  \\
        \midrule
        \multirow{1}{*}{\tt transpose}           & transpose                                           & $\grbm{C} \grbmask{\grbm{M}} $                                                                         & $\grbaccumeq{} \grbm{A}\grbt$                                                       \\
        % \midrule
        % \tt kronecker                        & Kronecker multiplication                                                  & $\grbm{C} \grbmask{\grbm{M}}$                                                                          & $\grbaccumeq{} \grbkron{\grbm{A}, \grbm{B}}$                                        \\
        \midrule\midrule
        % </operations>
        % <methods>
        \multirow{2}{*}{\tt new}                 & new matrix                                          & \multicolumn{2}{l}{$\grbnewmatrix{\grbm{A}}{\grbplaceholder{TYPE}}{\grbplaceholder{PRECISION}}{n}{m}$}                                                                                       \\
                                                 & new vector                                          & \multicolumn{2}{l}{$\grbnewvector{\grbv{u}}{\grbplaceholder{TYPE}}{\grbplaceholder{PRECISION}}{n}$}                                                                                          \\
        \midrule
        \multirow{2}{*}{\tt build}               & matrix from tuples                                  & $\grbm{C}\ $                                                                                           & $\grbassign \left\{ \grba{i}, \grba{j}, \grba{x} \right\} $                         \\
                                                 & vector from tuples                                  & $\grbv{w}\ $                                                                                           & $\grbassign \left\{ \grba{i}, \grba{x} \right\} $                                   \\
        \midrule
        \multirow{2}{*}{\tt extractTuples}       & \multirow{2}{*}{extract index/value arrays}         & $ \left\{ \grba{i}, \grba{j}, \grba{x} \right\} $                                                      & $\grbassign \grbm{A} $                                                              \\
                                                 &                                                     & $ \left\{ \grba{i}, \grba{x} \right\} $                                                                & $\grbassign \grbv{u}   $                                                            \\
        \midrule
        \multirow{2}{*}{\tt dup}                 & duplicate matrix                                    & $\grbm{C} $                                                                                            & $\grbassign \grbm{A}$                                                               \\
                                                 & duplicate vector                                    & $\grbv{w} $                                                                                            & $\grbassign \grbv{u}$                                                               \\
        \midrule
        \multirow{2}{*}{\tt extractElement}      & \multirow{2}{*}{extract scalar element}             & $\grbs{s} $                                                                                            & $\ = \grbm{A}(\grbs{i}, \grbs{j})$                                                  \\
                                                 &                                                     & $\grbs{s} $                                                                                            & $\ = \grbv{u}(\grbs{i})$                                                            \\
        \midrule
        \multirow{2}{*}{\tt setElement}          & \multirow{2}{*}{set element}                        & $\grbm{C}(\grbs{i}, \grbs{j}) $                                                                        & $\ = \grbs{s}$                                                                      \\
                                                 &                                                     & $\grbv{w}(\grbs{i})$                                                                                   & $\ = \grbs{s}$                                                                      \\
        \bottomrule
        % </methods>
    \end{tabular}
    \caption{GraphBLAS operations and methods based on \cite{DBLP:journals/toms/Davis19}.
        \emph{Notation:}
        Matrices and vectors are typeset in bold, starting with uppercase ($\grbm{A}$) and lowercase ($\grbv{u}$) letters, respectively.
        Scalars including indices are lowercase italic ($\grbs{k}$, $\grbs{i}$, $\grbs{j}$) while arrays are lowercase bold italic ($\grba{x}$, $\grba{i}$, $\grba{j}$).
        $\grbplus$ and $\grbtimes$ are the addition and multiplication operators forming a semiring and default to conventional arithmetic $+$ and $\times$ operators.
        $\grbaccum$ is the accumulator operator.
        \todo{do we need everything, e.g. subassign, Kronecker?}
        \todo{matrices can be tranposed through descriptors, denoted with e.g. $\grbm{B}\grbt$}
        \todo{masks}
    }
    \label{tab:graphblas-notation}
\end{table*}


In this section, we summarize the key concepts in \grb.
Then, we present a concise notation for the operations and methods defined in the \grb standard.
Additionally, we demonstrate how the operations can be interpreted as graph processing primitives if vectors encode nodes in the and the matrices are adjacency matrices. \todo{adjacency matrices can be rectangular}
Note that the standard does not have such restrictions, hence, users of the API are free to define other kinds of vectors and matrices (such as incidence matrices) as well.

%``guiding principles'': understandable (biggest probability of first guess to be correct), similar to existing notations~\cite{GraphBLASv13}
%\todo{adjacency matrices should be square or generic rectangular matrices (representing bipartite graphs)...?}

\subsection{Overview}

Here, we give a brief overview of the theoretical aspects of the \grb. For more details, we refer the reader to tutorials~\cite{gabor_szarnyas_2020_4318870} and the specification documents~\cite{GraphBLASv13,GxBUserGuide}.

\paragraph{Data structures}
\grb builds on the duality between the graph and matrix data structures.
Namely, a directed graph $G = (V, E)$ can be represented with an adjacency matrix $\grbm{A} \in \grbbool^{|V| \times |V|}$ where $\grbm{A}_{i,j} = \grbtrue$ iff $(v_i, v_j) \in E$.
Induced subgraphs where source nodes are selected from $V_1$ and target nodes are selected from $V_2$ ($V_1, V_2 \subseteq V$) can be represented with $\grbm{A} \in \grbbool^{|V_1| \times |V_2|}$.
Vectors are used to encode data for nodes, \eg $\grbv{u} \in \grbbool^{|V|}$ can be used to select a subset of nodes, while $\grbv{u} \in \grbbool^{|V|}$.
Vectors and matrices can be defined over multiple types, \eg a nonnegative integer ($\grbuint$) matrix can encode the number of paths between two nodes, while a floating point ($\grbdouble$) matrix can encode edge weights.

\paragraph{Semirings}
\grb uses matrix operations %(\autoref{sec:operations})
to express graph processing primitives, \eg a matrix-vector multiplication $\grbm{A} \grbplustimes \grbm{u}$ finds incoming neighbors of a given set of nodes.
\grb allows users to use arbitrary semirings for multiplication operations.
In general, the multiplication operator $\grbtimes$ is used to define how to combine the values of matching input elements, while the addition operator $\grbplus$ defines how the results should be summarized.
For example, the $\grbminplus$ semiring uses $\grbplus$ as the multiplication operator to compute the path length and $\grbmin$ as the addition operator to determine the length of the shortest path.

\paragraph{Masks and accumulators}
All \grb operations whose output is a vector or a matrix, allow the use of masks to limit the scope of the computation. %? ~\cite{DBLP:conf/ipps/AzadBG15}
This implies that the computation should be performed on a given set of nodes (for vector masks) or a given set of edges (for matrix masks).
The accumulator operator $\grbaccum$ is a binary operator that determines how the result of an operation should be applied to its output.
The interplay of masks and the accumulators is discussed in the specifications~\cite{GraphBLASv13,GxBUserGuide}.

\paragraph{Notation}
To present our algorithms, we use the notation shown in \autoref{tab:graphblas-notation}.
Matrices and vectors are typeset in bold, starting with uppercase ($\grbm{A}$) and lowercase ($\grbv{u}$) letters, respectively.
Scalars including indices are lowercase italic ($\grbs{k}$, $\grbs{i}$, $\grbs{j}$) while arrays are lowercase bold italic ($\grba{x}$, $\grba{i}$, $\grba{j}$).

\subsection{Operations}
\label{sec:operations}

\paragraph{Matrix multiplication}
\label{sec:mxm}

The \emph{matrix-matrix multiplication} operation $\grbm{C} \grbmask{\grbm{M}} \grbaccumeq{} \grbm{A} \grbplustimes \grbm{B}$ expresses navigation step starting from edges of $\grbm{A}$ and uses the edges of $\grbm{B}$.
The result matrix $\grbm{C}$ contains paths with $\grbm{C}_{i,k}$ containing the summarized paths with start node $i$ in the graph of $\grbm{A}$ and end node $k$ in the graph of $\grbm{B}$.
The \emph{vector-matrix multiplication} operation $\grbv{w}\grbt \grbmask{\grbv{m}\grbt} \grbaccumeq{} \grbv{u}\grbt \grbplustimes \grbm{A}$ performs navigation starting from the nodes selected in vector $\grbv{u}$ among the edges encoded in matrix $\grbm{A}$,
while the \emph{matrix-vector multiplication} operation $\grbv{w} \grbmask{\grbv{m}} \grbaccumeq{} \grbm{A} \grbplustimes \grbv{u}$ performs navigation in the reverse direction of the edges encoded in $\grbm{A}$.
The result vector $\grbm{w}$ contains the set of reached nodes with the values computed on the semiring (combining the initial node values with the outgoing edge values using $\grbtimes$ and summarizing the results in the same end node using $\grbplus$).

\paragraph{Element-wise addition}

The \emph{element-wise addition} operation applies the operator $\grbgenericop$ on the elements selected by the \emph{union of the structures of its inputs},
\ie nodes/edges which are present in at least one of the inputs.
%For matrices, the operation is denoted with $\grbm{C} \grbmask{\grbm{M}} \grbaccumeq{} \grbm{A} \grbewiseadd{\grbgenericop} \grbm{B}$,
%for vectors, it is denoted with $\grbv{w} \grbmask{\grbv{m}} \grbaccumeq{} \grbv{u} \grbewiseadd{\grbgenericop} \grbv{v}$.

\paragraph{Element-wise multiplication}

The \emph{element-wise multiplication} operation applies the operator $\grbgenericop$ on the elements selected by the \emph{intersection of the structures of its inputs},
\ie nodes/edges which are present in both inputs.
The notation is the same as that of the element-wise addition operation except using $\grbewisemult{\grbgenericop}$.

\paragraph{Extract}
For adjacency matrix $\grbm{A}$,
the \emph{extract submatrix} operation $\grbm{C} \grbmask{\grbm{M}} \grbaccumeq{} \grbm{A}(\grba{i}, \grba{j})$ selects the induced subgraph where
the source nodes of the edges are in array $\grba{i}$ and
the target nodes of the edges are in array $\grba{j}$.
The \emph{extract vector} operation $\grbv{w} \grbmask{\grbv{m}} \grbaccumeq{} \grbv{A}(\grbs{i}, :)$ selects a column vector containing node $\grbs{i}$'s neighbors along incoming edges.
Finally, the \emph{extract subvector} operation $\grbv{w} \grbmask{\grbv{m}} \grbaccumeq{} \grbv{u}(\grba{i})$ selects the nodes with indices in array $\grba{i}$.

%with an optional mask $\grbmask{\grbm{M}}$ to limit the computation to certain edges.

\paragraph{Assign}

\todo{edge values, vertex values, constant value}

%\paragraph{Subassign} ?

\paragraph{Apply}

\todo{apply}


\paragraph{Select}

\todo{selectively keep edges/nodes based on condition, including the indices of ops}

\paragraph{Reduce}
For adjacency matrix $\grbm{A}$, 
the \emph{row-wise reduction} $\grbv{w} \grbmask{\grbv{m}} \grbaccumeq{} \grbreduce{\grbplus}{\grbs{j}}{\grbm{A}}{:,\grbs{j}}$ represents a summarization of the values on outgoing edges for each node (represented by row vector $\grbm{A}(:, \grbs{j})$) to vector $\grbv{w}$ with an optional mask $\grbmask{\grbv{m}}$ to limit the computation to certain nodes.
For matrix $\grbm{A}$, the \emph{reduction to scalar} $\grbs{s} \grbaccumeq{} \grbreduce{\grbplus}{\grbs{i}, \grbs{j}}{\grbm{A}}{\grbs{i},\grbs{j}}$ represents a summarization of all edge values.
For vector $\grbv{u}$, the \emph{reduction to scalar} $\grbs{s} \grbaccumeq{} \grbreduce{\grbplus}{\grbs{i}}{\grbm{u}}{\grbs{i}}$ represents a summarization of all node values.

\paragraph{Transposition}
The transposed variant of matrix $\grbm{A}$ is denoted with $\grbm{A}\grbt$ where $\grbm{A}\grbt(i, j) = \grbm{A}(j, i)$.
For adjacency matrix $\grbm{A}$, matrix $\grbm{A}\grbt$ contains the edges in the reverse direction.
Transposition can be applied as a standalone \grb operation $\grbm{C} \grbmask{\grbm{M}} \grbaccumeq{} \grbm{A}\grbt$ with an optional mask $\grbmask{\grbm{M}}$ and also to the input/output matrices of operations, for example:
$$\grbm{C}^{[\grbtransposesymbol]} \grbmask{\grbm{M}} \grbaccumeq{} \grbm{A}^{[\grbtransposesymbol]} \grbplustimes \grbm{B}^{[\grbtransposesymbol]}$$

% For symmetric matrices representing an undirected graph, $\grbm{A}\grbt = \grbm{A}$

%\paragraph{Kronecker}
%omitted to keep the paper concise

\subsection{Masks}

\todo{explain masks}

\todo{explain replace/merge as per Scott's email}
% REPLACE: C :=              M .* (C + AB)
% MERGE:   C := (!M .* C) U [M .* (C + AB)]

\todo{revise Scott, Tim D, and Tim M}

\todo{consider adding semiring table}

what do masks mean?

the mask restricts the scope of the computation to the elements selected by the mask ($\grbm{m}$) or the complement of these elements ($\grbneg\grbm{m}$)

there are variations based on
(1)~how the elements are selected (\emph{valued} masks take the element values into consideration, while \emph{structural} masks only use the structure (pattern) of the mask
(2)~how the elements outside the selected ones are treated (they are \emph{replaced} with implicit zeros or they are kept intact and \emph{merged} with the results of the computation)

by default, masks should use \emph{merge} semantics, i.e. the computation can only effect elements selected by the mask, elements outside the mask are unaffected
$\grbv{w}\grbmask{\grbv{m}}$ and $\grbv{w}\grbmask{\grbneg \grbv{m}}$

\emph{replace} semantics, i.e. they annihilate all elements outside the mask. this is denoted with
$\grbv{w}\grbmaskreplace{\grbv{m}}$ and $\grbv{w}\grbmaskreplace{\grbneg \grbv{m}}$

by default, element values in the mask are checked and elements with zero value are not used
to avoid this, we use \emph{structural masks}
$\grbv{w}\grbmask{\grbstr{\grbv{m}}}$ and $\grbv{w}\grbmask{\grbneg \grbstr{\grbv{m}}}$

Combining \emph{replace semantics} and \emph{structural masks} is possible:
$\grbv{w}\grbmaskreplace{\grbstr{\grbv{m}}}$
$\grbv{w}\grbmaskreplace{\grbneg \grbstr{\grbv{m}}}$


% \begin{tabular}{ll}
%     $\grbv{w}\grbmask{\grbv{m}}$                 & compute for nodes selected by the values of $\grbv{m}$ & keep       \\
%     $\grbv{w}\grbmaskreplace{\grbv{m}}$          & compute for nodes selected by the values in $\grbv{m}$ & discard  \\
%     $\grbv{w}\grbmask{\grbstr{\grbv{m}}}$        & compute for nodes selected by the pattern of $\grbv{m}$ & keep      \\
%     $\grbv{w}\grbmaskreplace{\grbstr{\grbv{m}}}$ & compute for nodes selected by the pattern of $\grbv{m}$ & discard \\
% \end{tabular}

Initializing scalars, vectors, and matrices (GraphBLAS methods):

\begin{itemize}
    \item $\grbnewscalar{\grbs{s}}{\grbfloat}{64}$
    \item $\grbnewvector{\grbv{w}}{\grbfloat}{32}{n}$
    \item $\grbnewmatrix{\grbm{A}}{\grbuint}{16}{m}{n}$
    \item $\grbnewmatrix{\grbm{A}}{\grbint}{64}{k}{m}$
\end{itemize}

separate operator + descriptor ($\grbt$)

vectors can be interpreted as row/column vectors, we do not transpose them manually

extensions



\subsection{Methods}

% methods don't use masks



\todo{conversion table?}
\section{Notation}
\label{sec:notation}

%\setlength{\tabcolsep}{1.9pt}
%\renewcommand\arraystretch{0.85}

\begin{table*}[htbp]
    \centering
    \begin{tabular}{llr@{}ll}
        \toprule
        \multicolumn{1}{c}{\bf operation/method} & \multicolumn{1}{c}{\bf name}                        & \multicolumn{2}{c}{\bf notation}                                                                       & \multicolumn{1}{c}{\bf comment}                                                     \\
        % <operations>
        \midrule
        \tt mxm                                  & matrix-matrix multiplication                        & $\grbm{C} \grbmask{\grbm{M}}        $                                                                  & $\grbaccumeq{} \grbm{A} \grbplustimes \grbm{B}$                                     \\
        \tt vxm                                  & vector-matrix multiplication                        & $\grbv{\grbv{w}} \grbmask{\grbv{m}} $                                                                  & $\grbaccumeq{} \grbv{u} \grbplustimes \grbm{A}$                                     \\
        \tt mxv                                  & matrix-vector multiplication                        & $\grbv{w} \grbmask{\grbv{m}}        $                                                                  & $\grbaccumeq{} \grbm{A} \grbplustimes \grbv{u}$                                     \\
        \midrule
        \multirow{2}{*}{\tt eWiseAdd}            & element-wise addition                               & $\grbm{C} \grbmask{\grbm{M}} $                                                                         & $\grbaccumeq{} \grbm{A} \grbewiseadd{\grbgenericop} \grbm{B}$                       \\
                                                 & set union of patterns                               & $\grbv{w} \grbmask{\grbv{m}} $                                                                         & $\grbaccumeq{} \grbv{u} \grbewiseadd{\grbgenericop} \grbv{v}$                       \\
        \midrule
        \multirow{2}{*}{\tt eWiseMult}           & element-wise multiplication                         & $\grbm{C} \grbmask{\grbm{M}} $                                                                         & $\grbaccumeq{} \grbm{A} \grbewisemult{\grbgenericop} \grbm{B}$                      \\
                                                 & set intersection of patterns                        & $\grbv{w} \grbmask{\grbv{m}} $                                                                         & $\grbaccumeq{} \grbv{u} \grbewisemult{\grbgenericop} \grbv{v}$                      \\
        \midrule
        \multirow{4}{*}{\tt extract}             & extract submatrix                                   & $\grbm{C} \grbmask{\grbm{M}} $                                                                         & $\grbaccumeq{} \grbm{A}(\grba{i}, \grba{j})$                                        \\
                                                 & extract column vector                               & $\grbv{w} \grbmask{\grbv{m}} $                                                                         & $\grbaccumeq{} \grbv{A}(:, \grbs{j})$                                               \\
                                                 & extract row vector                                  & $\grbv{w} \grbmask{\grbv{m}} $                                                                         & $\grbaccumeq{} \grbv{A}(\grbs{i}, :)$                                               \\
                                                 & extract subvector                                   & $\grbv{w} \grbmask{\grbv{m}} $                                                                         & $\grbaccumeq{} \grbv{u}(\grba{i})$                                                  \\
        \midrule
        \multirow{4}{*}{\tt assign}              & assign matrix to submatrix with mask for $\grbm{C}$ & $\grbm{C} \grbmask{\grbm{M}} (\grba{i},\grba{j}) $                                                     & $\grbaccumeq{} \grbm{A}$                                                            \\
                                                 & assign scalar to submatrix with mask for $\grbm{C}$ & $\grbm{C} \grbmask{\grbm{M}} (\grba{i},\grba{j}) $                                                     & $\grbaccumeq{} \grbs{s}$                                                            \\
                                                 & assign vector to subvector with mask for $\grbv{w}$ & $\grbv{w} \grbmask{\grbv{m}} (\grba{i}) $                                                              & $\grbaccumeq{} \grbv{u}$                                                            \\
                                                 & assign scalar to subvector with mask for $\grbv{w}$ & $\grbv{w} \grbmask{\grbv{m}} (\grba{i}) $                                                              & $\grbaccumeq{} \grbs{s}$                                                            \\
        % \midrule
        % \multirow{4}{*}{\tt subassign (GxB)} & assign matrix to submatrix with submask for $\grbm{C}(\grba{i},\grba{j})$ & $\grbm{C}(\grba{i},\grba{j}) \grbmask{\grbm{M}} $                                                      & $\grbaccumeq{} \grbm{A}$                                                            \\
        %                                      & assign scalar to submatrix with submask for $\grbm{C}(\grba{i},\grba{j})$ & $\grbm{C}(\grba{i},\grba{j}) \grbmask{\grbm{M}} $                                                      & $\grbaccumeq{} \grbs{s}$                                                            \\
        %                                      & assign vector to subvector with submask for $\grbv{w}(\grba{i})$          & $\grbv{w}(\grba{i}) \grbmask{\grbv{m}} $                                                               & $\grbaccumeq{} \grbv{u}$                                                            \\
        %                                      & assign scalar to subvector with submask for $\grbv{w}(\grba{i})$          & $\grbv{w}(\grba{i}) \grbmask{\grbv{m}} $                                                               & $\grbaccumeq{} \grbs{s}$                                                            \\
        \midrule
        \multirow{2}{*}{\tt apply}               & \multirow{2}{*}{apply unary operator}               & $\grbm{C} \grbmask{\grbm{M}} $                                                                         & $\grbaccumeq{} \grbf{f}{\grbm{A}, \grbs{k}}$                                        & \multirow{2}{*}{$k$: thunk} \\
                                                 &                                                     & $\grbv{w} \grbmask{\grbv{m}} $                                                                         & $\grbaccumeq{} \grbf{f}{\grbv{u}, \grbs{k}}$                                                 \\
        \midrule
        \multirow{2}{*}{\tt select}              & \multirow{2}{*}{apply select operator}              & $\grbm{C} \grbmask{\grbm{M}} $                                                                         & $\grbaccumeq{} \grbm{A}\grbselect{\grbf{f}{\grbm{A}, \grbs{k}}}$                    & \multirow{2}{*}{$k$: thunk} \\
                                                 &                                                     & $\grbv{w} \grbmask{\grbv{m}} $                                                                         & $\grbaccumeq{} \grbv{u}\grbselect{\grbf{f}{\grbv{u}, \grbs{k}}}$                    \\
        \midrule
        \multirow{3}{*}{\tt reduce}              & reduce matrix to column vector                      & $\grbv{w} \grbmask{\grbv{m}} $                                                                         & $\grbaccumeq{} \grbreduce{\grbplus}{\grbs{j}}{\grbm{A}}{:,\grbs{j}}$                \\
                                                 & reduce matrix to scalar                             & $\grbs{s} $                                                                                            & $\grbaccumeq{} \grbreduce{\grbplus}{\grbs{i}, \grbs{j}}{\grbm{A}}{\grbs{i},\grbs{j}}$ \\
                                                 & reduce vector to scalar                             & $\grbs{s} $                                                                                            & $\grbaccumeq{} \grbreduce{\grbplus}{\grbs{i}}{\grbm{u}}{\grbs{i}}$                  \\
        \midrule
        \multirow{1}{*}{\tt transpose}           & transpose                                           & $\grbm{C} \grbmask{\grbm{M}} $                                                                         & $\grbaccumeq{} \grbm{A}\grbt$                                                       \\
        % \midrule
        % \tt kronecker                        & Kronecker multiplication                                                  & $\grbm{C} \grbmask{\grbm{M}}$                                                                          & $\grbaccumeq{} \grbkron{\grbm{A}, \grbm{B}}$                                        \\
        \midrule\midrule
        % </operations>
        % <methods>
        \multirow{2}{*}{\tt new}                 & new matrix                                          & \multicolumn{2}{l}{$\grbnewmatrix{\grbm{A}}{\grbplaceholder{TYPE}}{\grbplaceholder{PRECISION}}{n}{m}$}                                                                                       \\
                                                 & new vector                                          & \multicolumn{2}{l}{$\grbnewvector{\grbv{u}}{\grbplaceholder{TYPE}}{\grbplaceholder{PRECISION}}{n}$}                                                                                          \\
        \midrule
        \multirow{2}{*}{\tt build}               & matrix from tuples                                  & $\grbm{C}\ $                                                                                           & $\grbassign \left\{ \grba{i}, \grba{j}, \grba{x} \right\} $                         \\
                                                 & vector from tuples                                  & $\grbv{w}\ $                                                                                           & $\grbassign \left\{ \grba{i}, \grba{x} \right\} $                                   \\
        \midrule
        \multirow{2}{*}{\tt extractTuples}       & \multirow{2}{*}{extract index/value arrays}         & $ \left\{ \grba{i}, \grba{j}, \grba{x} \right\} $                                                      & $\grbassign \grbm{A} $                                                              \\
                                                 &                                                     & $ \left\{ \grba{i}, \grba{x} \right\} $                                                                & $\grbassign \grbv{u}   $                                                            \\
        \midrule
        \multirow{2}{*}{\tt dup}                 & duplicate matrix                                    & $\grbm{C} $                                                                                            & $\grbassign \grbm{A}$                                                               \\
                                                 & duplicate vector                                    & $\grbv{w} $                                                                                            & $\grbassign \grbv{u}$                                                               \\
        \midrule
        \multirow{2}{*}{\tt extractElement}      & \multirow{2}{*}{extract scalar element}             & $\grbs{s} $                                                                                            & $\ = \grbm{A}(\grbs{i}, \grbs{j})$                                                  \\
                                                 &                                                     & $\grbs{s} $                                                                                            & $\ = \grbv{u}(\grbs{i})$                                                            \\
        \midrule
        \multirow{2}{*}{\tt setElement}          & \multirow{2}{*}{set element}                        & $\grbm{C}(\grbs{i}, \grbs{j}) $                                                                        & $\ = \grbs{s}$                                                                      \\
                                                 &                                                     & $\grbv{w}(\grbs{i})$                                                                                   & $\ = \grbs{s}$                                                                      \\
        \bottomrule
        % </methods>
    \end{tabular}
    \caption{GraphBLAS operations and methods based on \cite{DBLP:journals/toms/Davis19}.
        \emph{Notation:}
        Matrices and vectors are typeset in bold, starting with uppercase ($\grbm{A}$) and lowercase ($\grbv{u}$) letters, respectively.
        Scalars including indices are lowercase italic ($\grbs{k}$, $\grbs{i}$, $\grbs{j}$) while arrays are lowercase bold italic ($\grba{x}$, $\grba{i}$, $\grba{j}$).
        $\grbplus$ and $\grbtimes$ are the addition and multiplication operators forming a semiring and default to conventional arithmetic $+$ and $\times$ operators.
        $\grbaccum$ is the accumulator operator.
        \todo{do we need everything, e.g. subassign, Kronecker?}
        \todo{matrices can be tranposed through descriptors, denoted with e.g. $\grbm{B}\grbt$}
        \todo{masks}
    }
    \label{tab:graphblas-notation}
\end{table*}


``guiding principles'': understandable (biggest probability of first guess to be correct), similar to existing notations~\cite{GraphBLASv13}

\todo{initial draft by Gabor}

\todo{explain masks}

\todo{explain replace/merge as per Scott's email}
% REPLACE: C :=              M .* (C + AB)
% MERGE:   C := (!M .* C) U [M .* (C + AB)]

\todo{revise Scott, Tim D, and Tim M}

\todo{consider adding semiring table}

what do masks mean?

the mask restricts the scope of the computation to the elements selected by the mask ($\grbm{m}$) or the complement of these elements ($\grbneg\grbm{m}$)

there are variations based on
(1)~how the elements are selected (based on the \emph{values} in the mask or the \emph{structure} of the mask)
(2)~how the elements outside the selected ones are treated (they are \emph{replaced} with implicit zeros or they are kept intact and \emph{merged} with the results of the computation)

by default, masks should use \emph{merge} semantics, i.e. the computation can only effect elements selected by the mask, elements outside the mask are unaffected
$\grbv{w}\grbmask{\grbv{m}}$ and $\grbv{w}\grbmask{\grbneg \grbv{m}}$

\emph{replace} semantics, i.e. they annihilate all elements outside the mask. this is denoted with
$\grbv{w}\grbmaskreplace{\grbv{m}}$ and $\grbv{w}\grbmaskreplace{\grbneg \grbv{m}}$

by default, element values in the mask are checked and elements with zero value are not used
to avoid this, we use \emph{structural masks}
$\grbv{w}\grbmask{\grbstr{\grbv{m}}}$ and $\grbv{w}\grbmask{\grbneg \grbstr{\grbv{m}}}$

Combining \emph{replace semantics} and \emph{structural masks} is possible:
$\grbv{w}\grbmaskreplace{\grbstr{\grbv{m}}}$
$\grbv{w}\grbmaskreplace{\grbneg \grbstr{\grbv{m}}}$


% \begin{tabular}{ll}
%     $\grbv{w}\grbmask{\grbv{m}}$                 & compute for nodes selected by the values of $\grbv{m}$ & keep       \\
%     $\grbv{w}\grbmaskreplace{\grbv{m}}$          & compute for nodes selected by the values in $\grbv{m}$ & discard  \\
%     $\grbv{w}\grbmask{\grbstr{\grbv{m}}}$        & compute for nodes selected by the pattern of $\grbv{m}$ & keep      \\
%     $\grbv{w}\grbmaskreplace{\grbstr{\grbv{m}}}$ & compute for nodes selected by the pattern of $\grbv{m}$ & discard \\
% \end{tabular}

Initializing scalars, vectors, and matrices (GraphBLAS methods):

\begin{itemize}
    \item $\grbnewscalar{\grbs{s}}{\grbfloat}{64}$
    \item $\grbnewvector{\grbv{w}}{\grbfloat}{32}{n}$
    \item $\grbnewmatrix{\grbm{A}}{\grbuint}{16}{m}{n}$
    \item $\grbnewmatrix{\grbm{A}}{\grbint}{64}{k}{m}$
\end{itemize}

\paragraph{Transposition}
separate operator + descriptor ($\grbt$)

vectors can be interpreted as row/column vectors, we do not transpose them manually

extensions

conversion table

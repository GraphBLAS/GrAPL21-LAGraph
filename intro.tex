\section{Introduction}
\label{sec:introduction}

LAGraph is a library of Graph Algorithms based on the GraphBLAS

Key contributions:

\begin{itemize}
    \item document design decisions for LAGraph
    \item present a concise notation for \grb algorihms
    \item algorithms of the GAP benchmark suite~\cite{DBLP:journals/corr/BeamerAP15} used in the IISWC benchmark paper~\cite{DBLP:conf/iiswc/AzadABBCDDDDFGG20}
    \item improve data ingestion performance, e.g. using SIMD techniques~\cite{DBLP:journals/vldb/LangdaleL19}
\end{itemize}

Recently, numerous graph-specific have targeted GPUs such as the \mbox{GraphBLAS Template Library (GBTL)}~\cite{7529957}, Gunrock~\cite{DBLP:journals/topc/WangPDWYWOYLRO17} and \mbox{GraphBLAST}~\cite{DBLP:journals/corr/abs-1908-01407}, and FPGAs~\cite{DBLP:journals/corr/abs-1903-06697}.

However, in the near future we expect even more heterogeneous hardware architectures including graph-specific hardware based on the Programmable Integrated Unified Memory Architecture (PIUMA)~\cite{DBLP:journals/corr/abs-2010-06277}.
Additionally, graph processing workloads can be offloaded to  machine learning accelerators, \eg
Tensor Processing Units (TPUs)~\cite{DBLP:conf/isca/JouppiYPPABBBBB17},
systolic arrays using reconfigurable dataflow architecture~\cite{SambaNova},
sparse linear algebra-based deep learning accelerators~\cite{Cerebras}.

Previous \grb design papers:
theory~\cite{DBLP:conf/hpec/MattsonBBBDFFGGHKLLPPRSWY13},
C API~\cite{DBLP:conf/hpec/MattsonYMBM17},
C++ API~\cite{DBLP:conf/ipps/BrockBMMM20},
distributed API~\cite{DBLP:conf/ipps/BrockBMMMPSS20},
LAGraph~\cite{DBLP:conf/ipps/MattsonDKBMMY19}

\begin{lstlisting}[language=C, label=lst:example, caption=Example]
int main() {
    return 0; // return zero
}
\end{lstlisting}



\footnote{A non peer-reviewed comparison of 6 popular graph algorithms libraries is available at
\url{https://www.timlrx.com/blog/benchmark-of-popular-graph-network-packages-v2}.}

\section{Introduction}
\label{sec:introduction}

Graphs represent networks of relationships. They play a key role in 
a wide range of applications.   Consequently, numerous libraries of graph algorithms exist 
such as igraph~\cite{igraph}, NetworkX~\cite{DBLP:reference/snam/X18xv}, and SNAP~\cite{DBLP:journals/tist/LeskovecS16}.
These libraries let programmers work with graphs without the need to master the art of crafting graph algorithms.

There are multiple ways to build libraries of graph algorithms.  One approach
views graphs in terms of sparse matrices and graph algorithms in terms of 
linear algebra. This perspective led to the 
GraphBLAS~\cite{DBLP:conf/hpec/MattsonBBBDFFGGHKLLPPRSWY13,DBLP:conf/hpec/MattsonYMBM17}; 
a community effort~\cite{GraphBLASforum} to define low level building blocks for graph algorithms as linear algebra.
The GraphBLAS, however, are for graph algorithm \emph{developers}.  They are too 
low level for graph algorithm \emph{users}.  To focus on users and the 
algorithms they require, researchers from the graphBLAS community launched the
LAGraph project~\cite{DBLP:conf/ipps/MattsonDKBMMY19}.  

A major goal for the LAGraph project is to produce a library 
of high quality, production-worthy algorithms constructed on top 
the GraphBLAS library.  In this paper, we describe the first release of the LAGraph library~\cite{LAGraphRepo}.
While LAGraph will eventually work with any implementation of the GraphBLAS, it is currently tied to the
the SuiteSparse GraphBLAS library~\cite{SuiteSparseGraphBLAS}.

In this first release of LAGraph, the range of algorithms is narrow (the algorithms found in the GAP benchmark suite).
With so few algorithms, we were able to focus our efforts on the key design decisions needed to establish a solid
foundation we can build on for the future.  Those design decisions 
and the rationale behind them are a key contribution of this paper.  We also establish a performance baseline for
our ongoing work with LAGraph with results for the GAP benchmark suite~\cite{DBLP:journals/corr/BeamerAP15}.

The LAGraph project is much more than a library project.   Another goal for the LAGraph project is to 
create a repository of algorithms based on the GraphBLAS and study them in order to advance the state of the art in 
Graph algorithms expressed as Linear algebra. To support this goal, we created a concise notation for expressing
graph algorithms in terms of the GraphBLAS.   As an example of the use of this notation, we use it to describe 
the algorithms used in the GAP benchmark suite.  We view this notation as a key contribution of this paper.  

%
%Save for Future work: improve data ingestion performance, e.g. using SIMD techniques~\cite{DBLP:journals/vldb/LangdaleL19}

%Previous \grb design papers:
%theory~\cite{DBLP:conf/hpec/MattsonBBBDFFGGHKLLPPRSWY13},
%C API~\cite{DBLP:conf/hpec/MattsonYMBM17},
%C++ API~\cite{DBLP:conf/ipps/BrockBMMM20},
%distributed API~\cite{DBLP:conf/ipps/BrockBMMMPSS20},
%LAGraph~\cite{DBLP:conf/ipps/MattsonDKBMMY19}


%\todo{add two paragraphs with a high-level overview of LAGraph}

%\footnote{A non peer-reviewed comparison of 6 popular graph algorithms libraries is available at
%\url{https://www.timlrx.com/blog/benchmark-of-popular-graph-network-packages-v2}.}

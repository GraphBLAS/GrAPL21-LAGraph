\section{Utility Fuctions}
\label{sec:utility}

LAGraph includes a set of utility functions that operate
on a graph.  All function names are prefixed with \verb'LAGraph_'
so we exclude that prefix in the names below, for brevity.

% Here is a rough categorization of the utilities (not all included yet)
\begin{itemize}

\item {\bf Graph Properties:}
    % If they operate on the LAGraph\_Graph cached properties consider a
    % consistent naming scheme like LAGraph\_Property\_XXX
    An \verb'LAGraph_Graph' includes cached properties which can be
    assigned by Basic methods, or which are required by Advanced methods.

    % \begin{itemize}
      % \item
      \verb'DeleteProperties' clear all properties,
      % \item
      \verb'Property_AT' computes the transpose of the adjacency matrix \verb'G->A',
      % \item
      \verb'Property_RowDegree' computes the row degrees of \verb'G->A',
      % \item
      \verb'Property_ColDegree' computes the column degrees of \verb'G->A',
      % \item
      and
      \verb'Property_ASymmetricPattern' determines if the pattern of \verb'G->A' is symmetric or unsymmetric.
       %   (consider Property\_ClearAll), Property\_AT, Property\_AssymetricPattern, Property\_RowDegree, Property\_ColDegree
    % \end{itemize}

\item {\bf Display and debug:}
    \verb'CheckGraph' checks the validity of a graph.
    Since the graph is not opaque, a user application is able to change a graph
    arbitrarily and thus might make it an invalid object.
    \verb'DisplayGraph' displays a graph and its properties.

\item {\bf Memory management:}
    Wrappers for \verb'malloc', \verb'calloc', \verb'realloc', and \verb'free',
    allowing a user application to select the memory manager to be used.
    These default to the ANSI C11 library functions.
    % malloc, calloc, realloc, free stuff.  This might be covered in the design
    % decisions section

% \item {\bf Threading:}
    % Get/SetNumThreads. Shouldn’t this be part of init, or an
    % LAGraph\_Context?  Should this also be discussed in design decisions.
    % GraphBLAS threading is one thing, but this seems to be LAGraph threading
    % (outside of GraphBLAS calls).  I could see this as algorithm specific and
    % not a general util.  The only way to leverage these inside algorithms is
    % to either set a global property that all algorithms have access to, or
    % creating a context that is passed to algorithms.

\item {\bf Graph I/O:}
    % \begin{itemize}
    % \item
    \verb'BinRead' and \verb'BinWrite' read/write a \verb'GrB_Matrix' in binary form.
    % \item
    \verb'MMRead' and \verb'MMWrite' read/write a \verb'GrB_Matrix' in Matrix Market form.
    % , MMRead (we are missing BinWrite, MMWrite), DisplayGraph (is this a
    % pretty print of the matrix or all the properties, this should also take
    % FILE*)
    % \end{itemize}

\item {\bf Matrix operations:}
    \verb'Pattern' returns a boolean matrix containing the pattern of a matrix.
    % Pattern (MakePattern, GetPattern).  Not sure how to categorize these
% \item {\bf Matrix comparison}:
    \verb'IsEqual' determines if two matrices are equal.  It selects the appropriate
    \verb'GrB_EQ_T' operator that matches the matrix type, and then calls \verb'IsAll'.
    \verb'IsAll' compares two matrices and returns false if
    the pattern of the two matrices differ.  It then uses a given comparator operator to
    compare all pairs of entries, and returns true if all comparisons return true.
    % IsEqual, IsAll. Are these graph or matrix utils? IsAll
    % is an ambiguous (maybe misleading) name because it seems to be a generic
    % comparator (consider CompareGraphs).

\item {\bf Degree operations:}
    % SortByDegree, SampleDegree  (maybe belongs grouped here as it is computing properties, but maybe not cached)
    \verb'SortByDegree' returns a permutation that sorts a graph by its row/column degrees, and
    \verb'SampleDegree' computes a quick estimate of the mean and median row/column degrees.

\item {\bf Error handling:}
    \verb'LAGraph_TRY' and \verb'GrB_TRY' are helper macros for a simple try/catch
    mechanism.  They require the user application to define \verb'LAGraph_CATCH'
    and \verb'GrB_CATCH'.
    % TRY/CATCH, MIN/MAX, tic/toc, TypeName, KindName

\item {\bf Other:}
    \verb'TypeName' returns a string with the name of a \verb'GrB_Type'.
    \verb'KindName' returns a string with of graph kind (directed or undirected).
    \verb'Tic' and \verb'Toc' provide a portable timer.
    \verb'Sort1', \verb'Sort2', and \verb'Sort3' sort 1, 2, or 3 integer arrays.

% \item {\bf Consider for removal} Things that may be implementation detail and could be buried:

% \begin{itemize}

%  \item Sort1/2/3 - currently only Sort2 is used.  I don't see a strong need to include these as part of the public API at this time.

%   \item Random15/60 - I see Random64 being the most widely usable. These can easily be buried as well, and I would still suggest Random64 (Random60 seems too SuiteSparse:GraphBLAS specific.

%   \item Test\_ReadProblem (move to the TestArea)

% \end{itemize}
\end{itemize}


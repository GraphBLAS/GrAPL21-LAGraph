\section{Evaluation}
\label{sec:evaluation}

\todo{Tim D}

\subsection{SuiteSparse Extensions}


In a prior paper (\cite{DBLP:conf/iiswc/AzadABBCDDDDFGG20}), an early draft of
SuiteSparse:GraphBLAS v4.0.0 (Aug 2020) was compared with the GAP benchmark
\cite{DBLP:conf/sc/BeamerAP12} and four other graph libraries.  This prior
version of SS:GrB included two primary data structures for its sparse matrices:
compressed sparse vector, and a hypersparse variant
\cite{DBLP:conf/ipps/BulucG08}, both held by row or by column.  It included a
draft implementation of a bitmap data structure that could only be used in a
prototype breadth-first search.  Since then, SuiteSparse:GraphBLAS v4.0.3 has
been released, with full support for bitmap and full matrices for all its
operations.  In an $m$-by-$n$ bitmap matrix, the values are held in a full
array of size $mn$, and another \verb'int8_t' array of size $mn$ holds the
sparsity pattern of the matrix.  A full matrix is a simple dense array of size
$mn$.

The bitmap format is particularly important for the ``pull'' phase of an
algorithm, as used in direction-optimizing breadth-first-search
\cite{DBLP:conf/sc/BeamerAP12}.  The GAP benchmark suite uses this method by
holding its frontier as a bitmap in the pull step and as a list in the push
step. The GAP BFS was shown to be typically the fastest BFS amongst the 6 graph
libraries compared in \cite{DBLP:conf/iiswc/AzadABBCDDDDFGG20} (for 4 of the 5
benchmark graphs).  With the addition of the bitmap format to SS:GrB,
LAGraph+SS:GrB is able to come within a factor of 2 or so of the performance of
the highly-tuned BFS GAP benchmark (see the results in the next section), for
those 4 graphs.  At the same time, however, the BFS is very easily expressed in
LAGraph as an easy-to-read and easy-to-write code.  This enables non-experts to
obtain a reasonably high level of performance with modest programming effort
when writing their own graph algorithms.

% TODO do I use the curly bracket notation since all masks are structural?

Direction-optimization is incredibly simple to add to an LAGraph algorithm.
For example, a batched direction-optimizing betweenness-centrality (BC)
algorithm in LAGraph only requires a simple heuristic to determine which
direction to use, followed by masked matrix-matrix multiplication with the
matrix or its transpose: $F \langle \neg P \rangle =FB'$ (the pull) or $F
\langle \neg P \rangle =FA$ (the push), where $A$ is the adjacency matrix of
the graph and $B=A'$ is its explicit transpose, $F$ is the frontier, and the
complemented mask $\neg P$ is the set of unvisited nodes.  The multiplication
$FB'$ relies on the descriptor to represent the transpose of $B$, which is not
explicitly transposed.  In the backward phase, the pull step is $W=WA'$ while
the push is $W=WB$, where $W$ is the 4-by-$n$ matrix in which centrality is
accumulated.

Additional optimizations added to SS:GrB in the past year include a {\em lazy
sort}.  Normally, SS:GrB keeps its vectors sorted (row vectors in a CSR matrix,
or column vectors if the matrix is held by column), with entries sorted in
ascending order of column or row index, respectively.  This simplifies the many
algorithms that operate on a \verb'GrB_Matrix'.  However, some algorithms
naturally produce a jumbled result (matrix multiply in particular), while many
algorithms are tolerant of jumbled input matrices.  We thus allow the sort to
be left pending.  The lazy sort joins two other kinds of pending work in
SS:GrB: {\em pending tuples}, and {\em zombies}.  A pending tuple is an entry
that is held inside a matrix in an unsorted list, awaiting insertion into the
CSR/CSC format of a \verb'GrB_Matrix'.  A zombie is the opposite: it is an
entry in the CSR/CSC format that has been marked for deletion, but has not yet
been deleted from the matrix.  With the lazy sort, the sort is postponed until
another algorithm requires sorted input matrices.  If the sort is lazy enough,
it might never occur, which is the case for the LAGraph BFS.

Another useful addition to SS:GrB is the new positional binary operators.
The BFS relies on the ANY-SECONDI semiring to compute a single step,
$q \langle \neg \pi \rangle = q'A$, where $q$ is the current frontier
$\pi$ is the parent vector, and $A$ is the adjacency matrix.

Consider a matrix multiply for conventional linear algebra, where the PLUS
monoid sums a set of $t$ entries to obtain a single scalar for computing
$c_{ij} = \sum a_{ik} b_{kj}$ in the matrix multiply $C=AB$.  The ANY monoid
performs the reduction of $t$ entries to a single number by merely selecting
any one of the $t$ entries as the result $c_{ij}$.  The selection is done
non-deterministically, allowing for a benign race condition.  In the BFS, this
corresponds to selecting any valid parent of a newly discovered node.  Indeed,
the creation of the ANY operator was inspired by Scott Beamer's \verb'bfs.cc'
method in the GAP benchmark, which has the same benign race condition.  The ANY
monoid translates the concept of this benign race condition to construct a
valid BFS tree into a linear algebraic operation, suitable for implementation
in GraphBLAS.

The SECONDI operator is the multiplicative operator in the ANY-SECONDI
semiring, where the result of $a_{ik} b_{kj}$ is simply the index $k$ in the
semiring for $C=AB$.  This gives the id of the parent node for a newly
discovered node in the next frontier.  The ANY monoid then selects any valid
parent $k$.

\subsection{Performance Results}

Our benchmark environment is an NVIDIA DGX Station (donated to Texas A\&M by
NVIDIA in support of this research).  It includes a 20-core Intel(R) Xeon(R)
CPU E5-2698 v4 @ 2.20GHz, with 40 threads.  All codes were compiled with gcc
5.4.0 (-O3).  All default settings were used, which means that hyperthreading
was enabled.  The system has 256GB of RAM in a single socket (no NUMA effects).
LAGraph (Feb 7, 2021) and SuiteSparse:GraphBLAS 4.0.4-draft (Feb 7, 2021) were
used.  The NVIDIA DGX Station includes four P100 GPUs, but no GPUs were used by
this experiment (a GPU-accelerated SS:GrB is in progress).
Table~\ref{table:results} lists the run time (in seconds) for the GAP benchmark
and LAGraph+SS:GrB for the 6 algorithms on the 5 benchmark matrices.
The benchmark matrices are listed in Table~\ref{table:matrices}.

\begin{table}
\begin{center}
\begin{tabular}{|l|rrrrr|}
\hline
Algorithm :    &   \multicolumn{5}{c|}{graph, with run time in seconds}  \\
 package       &   Kron    &   Urand   &   Twitter  &  Web    &    Road  \\
\hline
BC   : GAP     &  31.52    &  46.36    &  10.82     &  3.01   &    1.50  \\
BC   : Jan26   &  26.85    &  31.78    &  10.04     &  9.25   &   51.91  \\
BC   : Feb6    &  24.75    &  30.66    &   9.28     &  8.97   &   43.66  \\
BC   : Feb7    &  24.52    &  30.69    &   9.11     &  8.43   &   34.06  \\
\hline
BFS  : GAP     &    .31    &    .58    &    .22     &   .34   &     .25  \\
BFS  : Jan26   &    .52    &   1.31    &    .33     &   .67   &    3.33  \\
BFS  : Feb6    &    .52    &   1.20    &    .33     &   .66   &    3.36  \\
\hline
PR   : GAP     &  19.81    &  25.29    &  15.16     &  5.13   &    1.01  \\
PR   : Jan26   &  21.96    &  27.75    &  17.22     &  9.30   &    1.34  \\
PR   : Feb6    &  21.67    &  27.60    &  17.14     &  9.34   &    1.32  \\
\hline
CC   : GAP     &    .53    &   1.66    &    .23     &   .22   &     .05  \\
CC   : Jan26   &   3.42    &   4.59    &   1.48     &  1.97   &    1.00  \\
CC   : Feb6    &   3.35    &   4.56    &   1.47     &  1.96   &     .97  \\
\hline
SSSP : GAP     &   4.91    &   7.23    &   2.02     &   .81   &     .21  \\
SSSP : Jan26   &  17.62    &  25.62    &   8.44     &  9.67   &   48.49  \\
SSSP : Feb6    &  17.58    &  25.60    &   8.18     &  9.60   &   48.24  \\
\hline
TC   : GAP     & 374.08    &  21.83    &  79.58     & 22.18   &     .03  \\
TC   : Jan26   & 943.47    &  34.10    & 242.36     & 35.15   &     .29  \\
TC   : Feb6    & 922.35    &  33.97    & 238.71     & 34.67   &     .23  \\
\hline
\end{tabular}
\caption{Run time of GAP and LAGraph+SS:GrB
\label{table:results}}
\end{center}
\end{table}

\begin{table}
\begin{center}
\begin{tabular}{|l|rrr|}
\hline
graph   & nodes        & entries in $A$ & graph kind \\
\hline
Kron    & 134,217,726 &  4,223,264,644 &  undirected   \\
Urand   & 134,217,728 &  4,294,966,740 &  undirected   \\
Twitter &  61,578,415 &  1,468,364,884 &  directed     \\
Web     &  50,636,151 &  1,930,292,948 &  directed     \\
Road    &  23,947,347 &     57,708,624 &  directed     \\
\hline
\end{tabular}
\caption{Benchmark matrices\label{table:matrices}
(\url{https://sparse.tamu.edu/GAP})}
\end{center}
\end{table}

% Notes: TC in SS is SandiaDot method only

With the simple addition of the bitmap (needed for the pull step), the
push/pull optimization in BC resulted in a nearly 2x performance gain in the
GraphBLAS method for the largest matrices, as compared to the SS:GrB version
used for the results presented in \cite{DBLP:conf/iiswc/AzadABBCDDDDFGG20}.

With this change the BC method in LAGraph+SS:GrB is not only expressible in a
simple, elegant code, but it is also faster than the highly-tuned GAP benchmark
method, \verb'bc.cc', for the three largest matrices (1.3x for Kron, 1.5x for
Urand, and 1.2x for Twitter).

% Feb7 update:  the sort is completely lazy.

% For the Jan26 results:
% We expect the BC in LAGraph+SS:GrB to become faster still in the next
% release because we have not yet fully exploited the lazy sort.  The frontier
% matrix $F$ is left jumbled by the lazy sort, but it is sorted right away by a
% subsequent assignment.  Most uses of \verb'GrB_assign' are intolerant of
% jumbled input matrices, so it sorts them on input.  However, \verb'GrB_assign'
% includes about 40 internal variations, a few of which do not actually require
% sorted input matrices.  The particular method used in \verb'GrB_assign' in the
% LAGraph BC method is one of those methods, so this would be simple to exploit.
% With this change, the sort would be so lazy that it would {\em never} occur.
% The frontier would be computed, left jumbled, used in subsequent computations,
% and then recomputed (and thus discarded), just as we currently do in the
% LAGraph BFS.

With the addition of bitmap format (which makes push/pull optimization very
simple to express, and very fast) and the ANY-SECONDI semiring, the BFS of a
directed or undirected graph is easily expressed in GraphBLAS, and has a
performance that is only about 2x slower than the GAP benchmark.  We expect
the remaining 2x performance gap arises from two issues:

\begin{enumerate}
\item
The GAP assumes that the graph has fewer than $2^{32}$ nodes and edges, and
thus uses 32-bit integers throughout.  GraphBLAS is written for larger
problems, and thus relies solely on 64-bit integers.  This cannot be easily
changed in GraphBLAS, but rather than ``fixing'' GraphBLAS to use smaller
integers, it is the GAP algorithms that would need to be updated for larger
graphs in the future.  In the current GAP benchmark graphs, two graphs are
chosen with almost exactly 4 billion edges.  Graphs of current interest in
large data science can easily exceeed $2^{32}$ nodes and edges \cite{9286235}.

\item In GraphBLAS, the BFS must be expressed as two calls.  The first computes
$q \langle \neg \pi \rangle = q'A$, and the second updates the parent vector,
$ \pi \langle q \rangle = q$:

{\footnotesize
\begin{verbatim}
  GrB_vxm (q, pi, NULL, semiring, q, A, GrB_DESC_RSC) ;
  GrB_assign (pi, q, NULL, q, GrB_ALL, n, GrB_DESC_S) ; \end{verbatim}}

In Scott Beamer's \verb'bfs.cc', these two steps are fused, and the
matrix-vector multiplication can write its result directly into the vector
\verb'pi'.  This could be implemented in a future GraphBLAS library, since the
GraphBLAS API allows for a non-blocking mode where work is queued and done
later, thus enabling a fusion of these two steps.  SS:GrB exploits the
non-blocking mode (for its lazy sort, pending tuples, and zombies) but does not
{\em yet} exploit the fusion of \verb'GrB_vxm' and \verb'GrB_assign'.  We
intend to exploit this in the future.
\end{enumerate}

Note that LAGraph+SS:GrB is quite slow for many algorithms (all but PR) on the
Road graph.  The primary reason for this is the high diameter of the Road graph
(about 6000).  This requires 6000 iterations of GraphBLAS in the BFS, each with
a tiny amount of work.  Each call to GraphBLAS does several \verb'malloc' and
\verb'free's, and in some cases the workspace must be initialized.  A future
version of SS:GrB is planned that will eliminate this work entirely, by
implementing an internal memory pool.  There may be other overheads, but we
hope that a memory pool, fusion to fully exploit non-blocking mode, and other
optimization will address this large performance gap for the Road graph for
these algorithms.

LAGraph+SS:GrB is also up to 3x slower than the GAP for the triangle counting
problem (for all but the Road graph, where it is even slower).  This
performance gap can be eliminated entirely in the future, if the \verb'GrB_mxm'
and \verb'GrB_reduce' are combined in a single fused step, by a full
exploitation of the GraphBLAS non-blocking mode.  The current method computes
$C \langle L \rangle = LU'$, followed by the reduction of $C$ to a single
scalar.  The matrix $C$ is then discarded.  All that GraphBLAS needs is a fused
kernel that does not explicitly instantiate the temporary matrix $C$.
This is permitted by the GraphBLAS C API Specification, but not yet implemented
in SS:GrB.


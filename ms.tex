\documentclass[conference]{IEEEtran}
%\IEEEoverridecommandlockouts
% The preceding line is only needed to identify funding in the first footnote. If that is unneeded, please comment it out.

\usepackage[hyphens]{url}
\usepackage{hyperref} 
\usepackage{graphicx}
\usepackage{amsmath}
\usepackage{amssymb}

\usepackage{etoolbox}
\newbool{colored}
\IfFileExists{./colored}{\booltrue{colored}}{\boolfalse{colored}}
\newbool{ascii}
\IfFileExists{./ascii}{\booltrue{ascii}}{\boolfalse{ascii}}

\usepackage{subcaption}
\usepackage{xspace}
\usepackage{booktabs}
\usepackage{enumitem}
\usepackage{numprint}
\usepackage{multirow}
\usepackage{xcolor}
%\usepackage{todonotes}
% \setuptodonotes{inline}
\newcommand{\todo}[1]{{\bf TODO: #1}}

\usepackage{tikz}
\usetikzlibrary{positioning,fit,arrows.meta,backgrounds}

% (Gabor) I usually write with frenchspacing on but turned it off now.
% Caveat: With nonfrenchspacing, comments in algorithm2e use double-spacing after ':', '.', etc. which looks bad.
% \frenchspacing

% (Scott) I usually use the minted package for code blocks configured as follows
% You put code inside \begin{cplus} \end{cplus}
\usepackage{minted}
\setminted{fontsize=\footnotesize,baselinestretch=.97,linenos,frame=lines,xleftmargin=6pt,numbersep=3pt,mathescape=true,escapeinside=||,bgcolor=bg}
\usemintedstyle{default}
\definecolor{bg}{rgb}{0.97,0.97,0.97}
\newminted{cpp}{}
\newenvironment{cplus}{\VerbatimEnvironment\begin{cppcode}}{\end{cppcode}}
\newmintinline[cplusinl,mathescape]{cpp}{}



% TODO: remove in submitted version
\thispagestyle{plain}\pagestyle{plain}

\usepackage[ruled,vlined,linesnumbered]{algorithm2e}

\SetKwProg{Fn}{Function}{}{end}
\SetCommentSty{itshape}

\SetKwComment{Comment}{\color{green!100}// }{}
\renewcommand{\CommentSty}[1]{\tt \footnotesize \color{green!100}#1}

\DontPrintSemicolon
\SetKwRepeat{Do}{do}{while}
\SetKw{KwDownto}{downto}
\SetKw{Continue}{continue}
\SetKw{Break}{break}

% https://tex.stackexchange.com/a/260697
% \newcommand{\lstnumberautorefname}{Line} % unused as we use minted

\renewcommand{\algorithmautorefname}{Alg.}
% https://tex.stackexchange.com/a/351229
\makeatletter
\patchcmd{\ALG@step}{\addtocounter{ALG@line}{1}}{\refstepcounter{ALG@line}}{}{}
\newcommand{\ALG@lineautorefname}{Line}
\makeatother
% to comply with IEEEtran's recommendation
\renewcommand{\figureautorefname}{Fig.}
\renewcommand{\sectionautorefname}{Sec.}
\renewcommand{\subsectionautorefname}{Sec.}
\renewcommand{\subsubsectionautorefname}{Sec.}

\makeatletter
\algocf@newcommand{KwDataXX}[1]{%
  \sbox\algocf@inputbox{\hbox{\KwSty{Data}\algocf@typo: }}%
  \ifthenelse{\boolean{algocf@inoutnumbered}}{\relax}{\everypar={\relax}}%
  {\let\\\algocf@newinput\hspace{\wd\algocf@inputbox}\hangindent=\wd\algocf@inputbox\hangafter=\wd\algocf@inputbox#1\par}%
  \algocf@linesnumbered% reset the numbering of the lines
}
\makeatother


\input{commands}
\hyphenation{Suite-Sparse}
\hyphenation{Graph-BLAS}
\hyphenation{Suite-Sparse-Graph-BLAS}

\newcommand{\suitesparse}{SuiteSparse\xspace}
\newcommand{\grb}{GraphBLAS\xspace}
\newcommand{\ssgrb}{SuiteSparse:GraphBLAS\xspace}
\newcommand{\gxb}{\ssgrb}
\newcommand{\lagraph}{LAGraph\xspace}
\newcommand{\pygrb}{pygraphblas\xspace}
\newcommand{\grblas}{grblas\xspace}

% Define new boolean flags using etoolbox ('\newbool' is similar to '\newtoggle').
% This workaround is needed as simply putting the newcommands inside 'IfFileExists' did not do the job
% as it broke with 'Illegal parameter number in definition of \reserved@a', a symptom probably caused
% by the lack of protection (\protect). Anyways, the workaround is actually cleaner.

%\newcommand{\grbreduce}[2]{\left[{#1}_j \, {#2}(:, j) \right]}
% \ifbool{ascii}{ % ASCII mode ======================================================================
%     \newcommand{\grbm}[1]{{\ifbool{colored}{\color{brown}}{}{\mathtt{#1}}}}% matrix
%     \newcommand{\grbv}[1]{{\ifbool{colored}{\color{lilac}}{}{\mathtt{#1}}}}% vector
%     \newcommand{\grba}[1]{{\ifbool{colored}{\color{gray}}{}{\mathtt{#1}}}}% array
%     \newcommand{\grbs}[1]{{\ifbool{colored}{\color{blue}}{}{\mathtt{#1}}}}% scalar

%     \newcommand{\grbstr}[1]{{\{#1\}}}
%     \newcommand{\grbmask}[1]{<\! #1 \!>}
%     \newcommand{\grbmaskreplace}[1]{<\!<\! #1 \!>\!>}
%     \newcommand{\grbneg}{\texttt{!}}
%     \newcommand{\grbassign}{\mathrel{\texttt{<-}}}
%     \newcommand{\grbf}[2]{\grboperation{#1}(#2)}
%     \newcommand{\grbreduce}[4]{[ {#1 #3} ]} % omit the indices
%     \newcommand{\grbt}{\texttt{'}} % transpose
%     \newcommand{\grbdiv}{\grbbinaryop{DIV}}
%     \newcommand{\grbminus}{\grbbinaryop{MINUS}}
%     \newcommand{\grbaccum}{\texttt{\ensuremath{+}}}
%     \newcommand{\grbaccumeq}[1]{\mathbin{\texttt{\ensuremath{\ifstrempty{#1}{\grbaccum}{#1}=}}}}

%     \newcommand{\grbplus}{\grbbinaryop{+}}
%     \newcommand{\grbtimes}{\grbbinaryop{\times}}
%     \newcommand{\grbapply}{\grbbinaryop{\odot}}

%     \newcommand{\grbfrac}[2]{(#1)/(#2)}

%     \newcommand{\grbbool}{\mathtt{bool}} % booleans
%     \newcommand{\grbuint}{\mathtt{uint}} % unsigned integers
%     \newcommand{\grbint}{\mathtt{int}}   % integers
%     \newcommand{\grbfloat}{\mathtt{fp}}  % floats (?)

%     \newcommand{\grbplaceholder}[1]{\mathsf{#1}}

%     \newcommand{\grbscalartype}[2]{\mathtt{#1#2()}}
%     \newcommand{\grbvectortype}[3]{\mathtt{#1#2(#3)}}
%     \newcommand{\grbmatrixtype}[4]{\mathtt{#1#2(#3, #4)}}

%     \newcommand{\grbnewscalar}[3]{\mathtt{#1 = \grbscalartype{#2}{#3}}}
%     \newcommand{\grbnewvector}[4]{\mathtt{#1 = \grbvectortype{#2}{#3}{#4}}}
%     \newcommand{\grbnewmatrix}[5]{\mathtt{#1 = \grbmatrixtype{#2}{#3}{#4}{#5}}}

%     \newcommand{\grbalpha}{\mathtt{alpha}}
%     \newcommand{\grboperator}[1]{\mathtt{#1}}

%     \newcommand{\grbrange}[2]{#1:#2}
%     \newcommand{\grbdontcare}{\_}

%     \newcommand{\grboperationnoarg}[1]{\mathtt{#1}}

%     \newcommand{\grbewiseadd}[1]{\grbbinaryop{#1[intersection]}}
%     \newcommand{\grbewisemult}[1]{\grbbinaryop{#1[union]}}
% }{ % LaTeX mode ===================================================================================
    \newcommand{\grbm}[1]{{\ifbool{colored}{\color{brown}}{}{\mathbf{#1}}}}% matrix
    \newcommand{\grbv}[1]{{\ifbool{colored}{\color{lilac}}{}{\mathbf{#1}}}}% vector
    \newcommand{\grba}[1]{{\ifbool{colored}{\color{gray}}{}{\textbf{\textit{#1}}}}}% array
    \newcommand{\grbs}[1]{{\ifbool{colored}{\color{blue}}{}{\mathit{#1}}}}% scalar

    \newcommand{\grbmask}[1]{\langle #1 \rangle}
    %\newcommand{\grbstr}[1]{{\{#1\}}}
    \newcommand{\grbstr}[1]{s(#1)}
    \newcommand{\grbval}[1]{\mathit{val}(#1)}
    %\newcommand{\grbmaskreplace}[1]{\langle\!\langle #1 \rangle\!\rangle}
    \newcommand{\grbmaskreplace}[1]{\langle #1, \mathrm{r} \rangle}
    \newcommand{\grbneg}{\neg}

    % use the \mapsfrom symbol extracted from the stix package as suggested in https://tex.stackexchange.com/a/331899/71109
    \DeclareFontEncoding{LS1}{}{}
    \DeclareFontSubstitution{LS1}{stix}{m}{n}
    \DeclareSymbolFont{arrows1}{LS1}{stixsf}{m}{n}
    \global\let\mapsfrom\undefined % undefine \mapsfrom because some templates such as acmart already have it
    \DeclareMathSymbol{\mapsfrom}{\mathrel}{arrows1}{"AB}
    \newcommand{\grbassign}{\mapsfrom}

    \newcommand{\grbf}[2]{\mathit{#1}(#2)}
    \newcommand{\grbreduce}[4]{[ {#1}_{#2}\, #3(#4) ]}
    \newcommand{\grbtransposesymbol}{\mathsf{T}}
    \newcommand{\grbt}{^\grbtransposesymbol} %^{\top}} % transpose
    %\newcommand{\grbdiv}{\grbbinaryop{\oslash}}
    %\newcommand{\grbminus}{\grbbinaryop{\ominus}}
    
    \newcommand{\grbdiv}{\grbbinaryop{div}}
    \newcommand{\grbminus}{\grbbinaryop{minus}}
    
    \newcommand{\grbaccum}{\ensuremath{\odot}}
    \newcommand{\grbaccumeq}[1]{\mathbin{\ensuremath{\ifstrempty{#1}{\grbaccum}{#1}\!\!=}}}
    %\newcommand{\grbaccumeq}[1]{\mathbin{\ensuremath{\ifstrempty{#1}{\grbaccum}{#1}_\cup\!\!=}}}

    \newcommand{\grbplus}{\oplus}
    \newcommand{\grbtimes}{\otimes}
    \newcommand{\grbapply}{\odot}
    
    \newcommand{\grbfrac}[2]{\frac{#1}{#2}}

    \newcommand{\grbbool}{\mathbb{B}}  % booleans
    \newcommand{\grbuint}{\mathbb{N}}  % unsigned integers
    \newcommand{\grbint}{\mathbb{Z}}   % integers
    \newcommand{\grbfloat}{\mathbb{Q}} % floats (?)
    \newcommand{\grbdouble}{\mathtt{FP64}} % doubles
    
    \newcommand{\grbplaceholder}[1]{\mathsf{#1}}

    \newcommand{\grbscalartype}[2]{#1_{#2}}
    \newcommand{\grbvectortype}[3]{#1_{#2}^{#3}}
    \newcommand{\grbmatrixtype}[4]{#1_{#2}^{#3 \times #4}}

    \newcommand{\grbnewscalar}[3]{\text{let: } #1 \in \grbscalartype{#2}{#3}}
    \newcommand{\grbnewvector}[4]{\text{let: } #1 \in \grbvectortype{#2}{#3}{#4}}
    \newcommand{\grbnewmatrix}[5]{\text{let: } #1 \in \grbmatrixtype{#2}{#3}{#4}{#5}}

    %\newcommand{\grbalpha}{\alpha}
    \newcommand{\grboperator}[1]{\mathsf{#1}}

    \newcommand{\grbrange}[2]{#1 \! : \! #2}
    \newcommand{\grbdontcare}{\textvisiblespace}

    \newcommand{\grboperationnoarg}[1]{\mathrm{#1}}

    \newcommand{\grbewiseadd}[1]{\grbbinaryop{#1_\cup}}
    \newcommand{\grbewisemult}[1]{\grbbinaryop{#1_\cap}}
    \newcommand{\grbdim}[1]{\mathrm{dim}(#1)}
% }

% do not lange/rangle for tuples as it is already used for masks
% do not use grbtuple for the time being
%\newcommand{\grbtuple}[1]{( #1 )}
\newcommand{\tuple}[1]{(#1)}

% trying to avoid too much syntax (e.g. using wedge/vee symbols for LAND/LOR)
%\newcommand{\grblorland}{\lor\!.\!\land}

\newcommand{\grbsemiringops}[2]{\mathbin{\grboperator{#1.#2}}}
\newcommand{\grbplustimes}{\grbsemiringops{\grbplus}{\grbtimes}}
%\newcommand{\grbplustimes}{\grbsemiringops{\grbplus_\cup}{\grbtimes_\cap}}

\newcommand{\grbpluspair}{\grbsemiringops{plus}{pair}}
\newcommand{\grbplussecond}{\grbsemiringops{plus}{second}}
\newcommand{\grbplusfirst}{\grbsemiringops{plus}{first}}
\newcommand{\grbanypair}{\grbsemiringops{any}{pair}}
\newcommand{\grbanyfirst}{\grbsemiringops{any}{first}}
\newcommand{\grbanysecond}{\grbsemiringops{any}{second}}
\newcommand{\grbanyfirstj}{\grbsemiringops{any}{firstj}}
\newcommand{\grbanyfirstjone}{\grbsemiringops{any}{firstj1}}
\newcommand{\grbminfirstj}{\grbsemiringops{min}{firstj}}
\newcommand{\grbminfirstjone}{\grbsemiringops{min}{firstj1}}
\newcommand{\grbanysecondi}{\grbsemiringops{any}{secondi}}
\newcommand{\grbanysecondione}{\grbsemiringops{any}{secondi1}}
\newcommand{\grbminsecondi}{\grbsemiringops{min}{secondi}}
\newcommand{\grbminsecondione}{\grbsemiringops{min}{secondi1}}
\newcommand{\grblorland}{\grbsemiringops{lor}{land}}
\newcommand{\grbminplus}{\grbsemiringops{min}{plus}}
\newcommand{\grbmaxplus}{\grbsemiringops{max}{plus}}
\newcommand{\grbmaxfirst}{\grbsemiringops{max}{first}}
\newcommand{\grbminfirst}{\grbsemiringops{min}{first}}
\newcommand{\grbminsecond}{\grbsemiringops{min}{second}}
\newcommand{\grbmaxsecond}{\grbsemiringops{max}{second}}
\newcommand{\grbsecondmin}{\grbsemiringops{second}{min}}
\newcommand{\grbsecondmax}{\grbsemiringops{second}{max}}
\newcommand{\grbarithmeticplustimes}{\grbsemiringops{+}{\times}}
\newcommand{\grbarithmeticplustimestext}{\grbsemiringops{plus}{times}}

\newcommand{\grbbinaryop}[1]{\mathbin{\grboperator{#1}}}
\newcommand{\grbany}{\grbbinaryop{any}}
\newcommand{\grbpair}{\grbbinaryop{pair}}
\newcommand{\grbland}{\grbbinaryop{land}}
\newcommand{\grblor}{\grbbinaryop{lor}}
\newcommand{\grbband}{\grbbinaryop{band}}
\newcommand{\grbbor}{\grbbinaryop{bor}}
\newcommand{\grbmin}{\grbbinaryop{min}}
\newcommand{\grbmax}{\grbbinaryop{max}}
\newcommand{\grbfirst}{\grbbinaryop{first}}
\newcommand{\grbsecond}{\grbbinaryop{second}}
\newcommand{\grbfirsti}{\grbbinaryop{firsti}}
\newcommand{\grbsecondi}{\grbbinaryop{secondi}}
\newcommand{\grbfirstione}{\grbbinaryop{firsti1}}
\newcommand{\grbsecondione}{\grbbinaryop{secondi1}}
\newcommand{\grbfirstj}{\grbbinaryop{firstj}}
\newcommand{\grbsecondj}{\grbbinaryop{secondj}}
\newcommand{\grbfirstjone}{\grbbinaryop{firstj1}}
\newcommand{\grbsecondjone}{\grbbinaryop{secondj1}}
\newcommand{\grbarithmeticplus}{\grbbinaryop{+}}
\newcommand{\grbarithmetictimes}{\grbbinaryop{\times}}
\newcommand{\grbarithmeticplustext}{\grbbinaryop{plus}}
\newcommand{\grbarithmetictimestext}{\grbbinaryop{times}}
\newcommand{\grbisne}{\grbbinaryop{isne}}
\newcommand{\grbplustext}{\grbbinaryop{plus}}
\newcommand{\grbtimestext}{\grbbinaryop{times}}

\newcommand{\grbgenericop}{\grboperator{op}}

% boolean values
\newcommand{\grbbooleanvalue}[1]{\mathtt{#1}}
\newcommand{\grbtrue}{\grbbooleanvalue{TRUE}}
\newcommand{\grbfalse}{\grbbooleanvalue{FALSE}}
\newcommand{\grbT}{\grbbooleanvalue{T}}
\newcommand{\grbF}{\grbbooleanvalue{F}}
\newcommand{\grbstring}{\textrm{String}}
\newcommand{\grbdate}{\textrm{Date}}

% cardinality / count
\newcommand{\grbcnt}[1]{| #1 |}

% operations
\newcommand{\grboperation}[2]{\grboperationnoarg{#1}(#2)}
\newcommand{\grbnrows}[1]{\grboperation{nrows}{#1}}
\newcommand{\grbncols}[1]{\grboperation{ncols}{#1}}
\newcommand{\grbnvals}[1]{\grboperation{nvals}{#1}}
\newcommand{\grbclear}[1]{\grboperation{clear}{#1}}
\newcommand{\grbdiag}[1]{\grboperation{diag}{#1}}
\newcommand{\grbselect}[1]{\grbmask{#1}}
\newcommand{\grbkron}[1]{\grboperation{kron}{#1}}
\newcommand{\grbtril}[1]{\grboperation{tril}{#1}}
\newcommand{\grbtriu}[1]{\grboperation{triu}{#1}}
\newcommand{\grbondiag}[1]{\grboperation{ondiag}{#1}}
\newcommand{\grboffdiag}[1]{\grboperation{offdiag}{#1}}

\newcommand{\grbind}[1]{\mathrm{ind}(#1)}

\usepackage{microtype}
\renewcommand*\ttdefault{txtt}

\usepackage{listings}

\usepackage{textcomp}
\lstset{upquote=true}


\begin{document}

%\title{LAGraph: A Graph Algorithm and Network Analysis Library for GraphBLAS}
\title{LAGraph: Linear Algebra, Network Analysis Libraries, and the Study of Graph Algorithms}
\author{ % Tim D modified the ordering: Gabor, then alphabetical
\IEEEauthorblockN{
    G\'abor Sz\'arnyas\IEEEauthorrefmark{5},
    David A. Bader\IEEEauthorrefmark{6},
    Timothy A. Davis\IEEEauthorrefmark{1},
    James Kitchen\IEEEauthorrefmark{3}, \\
    Timothy G. Mattson\IEEEauthorrefmark{2},
    Scott McMillan\IEEEauthorrefmark{4},
    Erik Welch\IEEEauthorrefmark{3}
} \\
\IEEEauthorblockA{
    \IEEEauthorrefmark{1}Texas A\&M University
}
\IEEEauthorblockA{
    \IEEEauthorrefmark{2}Parallel Computing Labs, Intel Corporation, Ocean Park, WA
}
\IEEEauthorblockA{
    \IEEEauthorrefmark{3}Anaconda, Inc.
}
\IEEEauthorblockA{
    \IEEEauthorrefmark{4}
    Software Engineering Institute, Carnegie Mellon University, Pittsburgh, PA
}
\IEEEauthorblockA{
    \IEEEauthorrefmark{5}CWI Amsterdam, The Netherlands
}
\IEEEauthorblockA{
    \IEEEauthorrefmark{6}New Jersey Institute of Technology
}
}

\maketitle

\begin{abstract}

Graphs and graph algorithms can be expressed in terms of 
linear algebra. The GraphBLAS are a library of low level building blocks for such algorithms.
The GraphBLAS target algorithm \emph{developers}.  The LAGraph project
targets graph algorithm \emph{users} with high-level algorithms common in network
analysis.   In this paper, we describe 
the first release of the LAGraph library.  We describe the design decisions behind the
library, the contents of the library, and performance data using the GAP benchmark suite.
LAGraph, however, is much more than a library development project.  It is also a
project to document and analyze the full range of algorithms enabled by the GraphBLAS.  To that end
we have developed a compact (and hopefully intuitive) notation for describing
these algorithms.  In this paper, we present that 
notation with examples from the GAP benchmark suite.  

\end{abstract}

\begin{IEEEkeywords}
Graph Processing, Graph Algorithms, Graph Analytics, Linear Algebra, GraphBLAS
\end{IEEEkeywords}

\section{Introduction}
\label{sec:introduction}

Graphs represent networks of relationships. They play a key role in 
a wide range of applications.   Consequently, numerous graph libraries exist 
such as igraph~\cite{igraph}, NetworkX~\cite{DBLP:reference/snam/X18xv}, and SNAP~\cite{DBLP:journals/tist/LeskovecS16}.
These libraries let programmers work with graphs without the need to master the art of crafting graph algorithms.

There are multiple ways to build libraries of graph algorithms.  One approach
views graphs in terms of sparse matrices and graph algorithms in terms of 
linear algebra. This perspective led to the 
GraphBLAS~\cite{DBLP:conf/hpec/MattsonBBBDFFGGHKLLPPRSWY13,DBLP:conf/hpec/MattsonYMBM17}; 
a community effort~\cite{GraphBLASforum} to define low-level building blocks for graph algorithms as linear algebra.
The GraphBLAS are for graph algorithm \emph{developers}.  They are too 
low-level for graph algorithm \emph{users}.  To focus on users and the 
algorithms they require, we launched the
LAGraph project~\cite{DBLP:conf/ipps/MattsonDKBMMY19}.  

The LAGraph project will produce a library 
of high quality, production-worthy algorithms constructed on top of
the GraphBLAS.  In this paper, we describe the first release of LAGraph~\cite{LAGraphRepo}.
While LAGraph will eventually work with any implementation of the GraphBLAS, it is currently tied to
the \ssgrb library~\cite{SuiteSparseGraphBLAS} (SS:GrB).

In this release of LAGraph, we restricted ourselves to versions of the algorithms found in the GAP benchmarks.
This restricted scope allowed us to focus on the key design decisions needed to establish a solid
foundation for the future.  Those design decisions, the rationale behind them, and a performance baseline 
using the GAP benchmark suite~\cite{DBLP:journals/corr/BeamerAP15} are key contributions of this paper.   

The LAGraph project is much more than a library project.   Another goal for the project is to 
create a repository of algorithms based on the GraphBLAS and study them to advance the state of the art in 
Graph algorithms expressed as Linear algebra. To support this goal, we created a concise notation for expressing
graph algorithms in terms of the GraphBLAS.   As an example of this notation in action, we use it to describe 
the algorithms used in the GAP benchmark suite.  We view this notation as a key contribution of this paper.  

%
%Save for Future work: improve data ingestion performance, e.g. using SIMD techniques~\cite{DBLP:journals/vldb/LangdaleL19}

%Previous \grb design papers:
%theory~\cite{DBLP:conf/hpec/MattsonBBBDFFGGHKLLPPRSWY13},
%C API~\cite{DBLP:conf/hpec/MattsonYMBM17},
%C++ API~\cite{DBLP:conf/ipps/BrockBMMM20},
%distributed API~\cite{DBLP:conf/ipps/BrockBMMMPSS20},
%LAGraph~\cite{DBLP:conf/ipps/MattsonDKBMMY19}


%\todo{add two paragraphs with a high-level overview of LAGraph}

%\footnote{A non peer-reviewed comparison of 6 popular graph algorithms libraries is available at
%\url{https://www.timlrx.com/blog/benchmark-of-popular-graph-network-packages-v2}.}


%TimM removed this figure.  We are short on space and this doesn't add enough to justify the space it consumes.
%\tikzset{
    module/.style={%
        draw,
        minimum width=50mm,
        minimum height=8mm,
        font=\sffamily,
        node distance=0mm,
        },
    module/.default=0.5cm,
    >=LaTeX
}
\begin{figure}[htb]
    \centering
    \begin{tikzpicture}
        \node[module, minimum height=16mm, text depth=8mm] (I1) {Graph application};
        \node[module, below=of I1, minimum width=30mm, xshift=10mm, yshift=8mm] (I2) {LAGraph};
        \node[module, below=of I1, minimum width=60mm] (I3) {GraphBLAS API};
        \node[module, below=of I3] (I4) {GraphBLAS implementation};
        \node[module, below=of I4] (I5) {Hardware architecture};
    \end{tikzpicture}
    \caption{Separation of concerns using the \grb API.}
    \label{fig:architecture}
\end{figure}


\section{Design Decisions}
\label{sec:decisions}

LAGraph is for users who want to use graph algorithms that run on top of the GraphBLAS.  
Our overarching design goal is ease of use with flexibility to handle advanced use-cases.
We don't wish to compromise performance, but when the tradeoff between convenience and performance
is unavoidable, we offer both and let the user choose.
LAGraph includes a set of data structures and utility functions 
that make it convenient for developers to write algorithms on top of GraphBLAS with 
an approachable API and consistent user experience.

\subsection{Core data structure}

\begin{listing}
\begin{cplus}
typedef struct LAGraph_Graph_struct
{
    GrB_Matrix   A;      // adjacency matrix of the graph
    LAGraph_Kind kind;   // kind of graph: directed, etc.
    
    // cached properties
    GrB_Matrix   AT;     // transpose of A
    GrB_Vector   row_degree;
    GrB_Vector   col_degree;
    LAGraph_BooleanProperty A_pattern_is_symmetric;
    int64_t      ndiag;  // -1 if unknown
} *LAGraph_Graph;

typedef struct LAGraph_Graph_struct *LAGraph_Graph; |$\label{line:LAGraph_Graph}$|

// creating a graph
GrB_Matrix M;
// ...construction of M omitted

LAGraph_Graph G;
LAGraph_New(&G, &M, LAGRAPH_DIRECTED_ADJACENCY); |$\label{line:CreateGraphObject}$|

// operating on properties
LAGraph_Property_AT(G, msg);  |$\label{line:ComputeTranspose}$|// compute/cache
\end{cplus}
\caption{{\tt LAGraph\_Graph} data structure and methods.}
\label{Lst.graph}
\end{listing}

The primary data structure in LAGraph is the \verb'LAGraph_Graph' which consists of primary components
and cached properties. The data structure is not opaque, providing the user with full ability to access and
modify all internal components. This contrasts with the opaque objects in the GraphBLAS.  This data structure is 
shown at the top of Listing~\ref{Lst.graph} and defined ultimately on Line~\ref{line:LAGraph_Graph}.

The primary components of this struct are a GraphBLAS matrix named \verb'A' and an enumeration \verb'kind'.
The kind indicates how the matrix should be interpreted.
Currently, the only kinds defined are \verb'LAGRAPH_ADJACENCY_UNDIRECTED' and 
\verb'LAGRAPH_ADJACENCY_DIRECTED', but more options will be added in the future.  Creating the 
Graph object is performed on Line~\ref{line:CreateGraphObject} of Listing~\ref{Lst.graph}.
Following this call, \verb'M' will be \verb'NULL'. The matrix previously pointed to by \verb'M' now
lives at \verb'G->A'. This ``move'' constructor helps avoid memory-freeing errors.

Cached properties include the transpose of \verb'A', the row degrees, column degrees, etc.
They can be computed from the primary components, but doing so repeatedly for each algorithm
utilizing \verb'A' would be wasteful. Having them live inside the Graph object simplifies
algorithm call signatures. Utility functions exist to compute each cached property.  For example,
Line~\ref{line:ComputeTranspose} of Listing~\ref{Lst.graph} will compute the transpose of \verb'G->A' and store it as \verb'G->AT'.
Following this call, any algorithm which is given \verb'G' will have access to both \verb'A' and its transpose.

Because the Graph object is not opaque, any piece of code may set the transpose as well. For instance, if an algorithm
computes the transpose as part of its normal logic, it could directly set \verb'G->AT'.
The expectation is that the Graph object will always remain consistent.
If \verb'G->A' is modified, all cached properties must be either be set as unknown or modified to reflect the change.
Properties which are not known are set to \verb'NULL' or \verb'LAGRAPH_BOOLEAN_UNKNOWN' in the case of Boolean properties.
This expectation is a convention that all LAGraph algorithm implementers are expected to follow.



\subsection{User modes}

Algorithms in LAGraph target two user modes: Basic and Advanced. The Basic user mode is for those who want
things to ``just work'', are less concerned about performance, and may be less experienced with graph libraries.
The Advanced user mode is for those whose primary concern is performance and are willing to conform to stricter
requirements to achieve that goal.

Algorithms targeting the Basic mode typically have limited options. Often, there will only be one function for
a given algorithm. Under the hood, that single algorithm might take different paths depending on the shape or
size of the input graph. The idea is that a basic user wants to compute PageRank or Betweenness Centrality,
but doesn't want to have to understand the five different ways to compute them. They simply want the correct answer.

Algorithms targeting the Advanced mode are often highly specialized implementations of an algorithm. The Advanced
mode user is expected to understand details such as push-pull and batch mode and why different techniques are
better for each graph. Advanced mode algorithms are very strict in their input. If the input doesn't match the
expected kind, an error will be raised.

Advanced mode algorithms will also raise an error if a cached property is needed by an algorithm, but is not
currently available on the Graph object. While Basic mode algorithms are free to compute and cache properties
on the Graph object, Advanced mode algorithms never will. The idea is to never surprise the user with unexpected
additional computation. An Advanced mode user must opt-in to all computations.

Often, Basic mode algorithms will inspect the input, possibly compute properties or transform the data,
and finally call one of the Advanced mode algorithms to do the actual work on the graph. Having these two user
modes allows LAGraph to target a wider range of users who vary in their experience with graph algorithms.


\subsection{Algorithm calling conventions}

Algorithms in LAGraph follow a general calling convention.

\begin{cplus}
int algorithm
(
    // outputs:
    TYPE *out1,
    TYPE *out2,
    ...
    // input/output
    TYPE inout,
    ...
    // inputs
    TYPE input1,
    TYPE input2,
    ...
    // error message holder
    char *msg
)
\end{cplus}

The return value is always an int with the following meaning:

\begin{itemize}
    \item \verb'=0 ->' success
    \item \verb'<0 ->' error
    \item \verb'>0 ->' warning
\end{itemize}

The meaning of a given error or warning value is algorithm-specific
and should be listed in the documentation for the algorithm.

Outputs appear first and are passed by reference. A pointer should be created by the caller, but
memory will be allocated by the algorithm. If the output is not needed, a \verb'NULL' is passed and the
algorithm will not return that output.

Input/Output arguments are passed by value. The expectation is that the object will be modified.
This supports things like batch mode in which a frontier is updated and returned to the caller.
It also supports Basic mode algorithms which may modify a Graph object by adding cached properties.

Inputs are passed by value and should never be changed by the algorithm.

The final argument of any LAGraph algorithm holds the error message. This must be \verb'char[]' of
size \verb'LAGRAPH_MSG_LEN'. When the algorithm returns an error or a warning, a message may be placed in
this array as additional information. Because the caller creates this array, the caller must free
the memory or reuse it as appropriate. If the algorithm is successful, it should fill the message array
with an empty string to clear any previous message.


\subsection{Error handling}

Because every algorithm in LAGraph can return an error, the return value of every call should be
checked before proceeding. To make this less burdensome for a C-based library, LAGraph provides a
convenience macro which works similar to try/catch in other languages.

\begin{cplus}
#define LAGraph_TRY(LAGraph_method)
{
    int LAGraph_status = LAGraph_method;
    if (LAGraph_status < 0)
    {
        LAGraph_CATCH (LAGraph_status);
    }
}
\end{cplus}


\verb'LAGraph_CATCH' can be defined before an algorithm and will be called in the event of an error.
This allows for proper freeing of memory and other necessary tasks.

A similar macro, \verb'GrB_TRY', will call \verb'GrB_CATCH' when making GraphBLAS calls which return
a \verb'GrB_Info' value other than \verb'GrB_SUCCESS' or \verb'GrB_NO_VALUE'.

\verb'LAGraph_TRY' and \verb'GrB_TRY' provide an easy to use and easy to read method for dealing with
error checking while writing graph algorithms.


\subsection{Contributing algorithms}

The LAGraph project welcomes contributions from graph practitioners who understand the GraphBLAS vision
of using the language of linear algebra to express graph computations. However, as a matter of
practical concern, many users want a stable experience when using LAGraph for doing real work. To balance
these, the LAGraph repository will have both a stable and an experimental folder.

New algorithms or modifications of existing algorithms will first be added to the experimental folder.
The release schedule of experimental algorithms will generally be much faster than the stable release,
and there is no expectation of a bug-free experience.
The goal is to generate lots of ideas and allow uninhibited contributions to
push the boundary of what is possible with the GraphBLAS. The stable release will be fully tested and
will move much slower, targeting the needs of those who want to use LAGraph as a complete, production-grade
library rather than as a research project.


\section{Notation}
\label{sec:notation}

\setlength{\tabcolsep}{1.9pt}
%\renewcommand\arraystretch{0.85}

\begin{table*}[htbp]
    \centering
    \begin{tabular}{llr@{}l}
        \toprule
        \multicolumn{1}{c}{\bf op./method}   & \multicolumn{1}{c}{\bf name}                                              & \multicolumn{2}{c}{\bf notation}                                                                                                                                                             \\
        % <operations>
        \midrule
        \tt mxm                              & matrix-matrix multiplication                                              & $\grbm{C} \grbmask{\grbm{M}}        $                                                                  & $\grbaccumeq{} \grbm{A} \grbplustimes \grbm{B}$                                     \\
        \tt vxm                              & vector-matrix multiplication                                              & $\grbv{\grbv{w}} \grbmask{\grbv{m}} $                                                                  & $\grbaccumeq{} \grbv{u} \grbplustimes \grbm{A}$                                     \\
        \tt mxv                              & matrix-vector multiplication                                              & $\grbv{w} \grbmask{\grbv{m}}        $                                                                  & $\grbaccumeq{} \grbm{A} \grbplustimes \grbv{u}$                                     \\
        \midrule
        \multirow{2}{*}{\tt eWiseAdd}        & element-wise addition                                                     & $\grbm{C} \grbmask{\grbm{M}} $                                                                         & $\grbaccumeq{} \grbm{A} \grbewiseadd{\grbgenericop} \grbm{B}$                       \\
                                             & set union of patterns                                                     & $\grbv{w} \grbmask{\grbv{m}} $                                                                         & $\grbaccumeq{} \grbv{u} \grbewiseadd{\grbgenericop} \grbv{v}$                       \\
        \midrule
        \multirow{2}{*}{\tt eWiseMult}       & element-wise multiplication                                               & $\grbm{C} \grbmask{\grbm{M}} $                                                                         & $\grbaccumeq{} \grbm{A} \grbewisemult{\grbgenericop} \grbm{B}$                      \\
                                             & set intersection of patterns                                              & $\grbv{w} \grbmask{\grbv{m}} $                                                                         & $\grbaccumeq{} \grbv{u} \grbewisemult{\grbgenericop} \grbv{v}$                      \\
        \midrule
        \multirow{4}{*}{\tt extract}         & extract submatrix                                                         & $\grbm{C} \grbmask{\grbm{M}} $                                                                         & $\grbaccumeq{} \grbm{A}(\grba{i}, \grba{j})$                                        \\
                                             & extract column vector                                                     & $\grbv{w} \grbmask{\grbv{m}} $                                                                         & $\grbaccumeq{} \grbv{A}(:, \grbs{j})$                                               \\
                                             & extract row vector                                                        & $\grbv{w} \grbmask{\grbv{m}} $                                                                         & $\grbaccumeq{} \grbv{A}(\grbs{i}, :)$                                               \\
                                             & extract subvector                                                         & $\grbv{w} \grbmask{\grbv{m}} $                                                                         & $\grbaccumeq{} \grbv{u}(\grba{i})$                                                  \\
        \midrule
        \multirow{4}{*}{\tt assign}          & assign matrix to submatrix with mask for $\grbm{C}$                       & $\grbm{C} \grbmask{\grbm{M}} (\grba{i},\grba{j}) $                                                     & $\grbaccumeq{} \grbm{A}$                                                            \\
                                             & assign scalar to submatrix with mask for $\grbm{C}$                       & $\grbm{C} \grbmask{\grbm{M}} (\grba{i},\grba{j}) $                                                     & $\grbaccumeq{} \grbs{s}$                                                            \\
                                             & assign vector to subvector with mask for $\grbv{w}$                       & $\grbv{w} \grbmask{\grbv{m}} (\grba{i}) $                                                              & $\grbaccumeq{} \grbv{u}$                                                            \\
                                             & assign scalar to subvector with mask for $\grbv{w}$                       & $\grbv{w} \grbmask{\grbv{m}} (\grba{i}) $                                                              & $\grbaccumeq{} \grbs{s}$                                                            \\
        \midrule
        \multirow{4}{*}{\tt subassign (GxB)} & assign matrix to submatrix with submask for $\grbm{C}(\grba{i},\grba{j})$ & $\grbm{C}(\grba{i},\grba{j}) \grbmask{\grbm{M}} $                                                      & $\grbaccumeq{} \grbm{A}$                                                            \\
                                             & assign scalar to submatrix with submask for $\grbm{C}(\grba{i},\grba{j})$ & $\grbm{C}(\grba{i},\grba{j}) \grbmask{\grbm{M}} $                                                      & $\grbaccumeq{} \grbs{s}$                                                            \\
                                             & assign vector to subvector with submask for $\grbv{w}(\grba{i})$          & $\grbv{w}(\grba{i}) \grbmask{\grbv{m}} $                                                               & $\grbaccumeq{} \grbv{u}$                                                            \\
                                             & assign scalar to subvector with submask for $\grbv{w}(\grba{i})$          & $\grbv{w}(\grba{i}) \grbmask{\grbv{m}} $                                                               & $\grbaccumeq{} \grbs{s}$                                                            \\
        \midrule
        \multirow{2}{*}{\tt apply}           & \multirow{2}{*}{apply unary operator}                                     & $\grbm{C} \grbmask{\grbm{M}} $                                                                         & $\grbaccumeq{} \grbf{f}{\grbm{A}}$                                                  \\
                                             &                                                                           & $\grbv{w} \grbmask{\grbv{m}} $                                                                         & $\grbaccumeq{} \grbf{f}{\grbv{u}}$                                                  \\
        \midrule
        \multirow{2}{*}{\tt select}          & \multirow{2}{*}{apply select operator}                                    & $\grbm{C} \grbmask{\grbm{M}} $                                                                         & $\grbaccumeq{} \grbm{A}\grbselect{\grbf{f}{\grbm{A}, \grbs{k}}}$                    \\
                                             &                                                                           & $\grbv{w} \grbmask{\grbv{m}} $                                                                         & $\grbaccumeq{} \grbv{u}\grbselect{\grbf{f}{\grbv{u}, \grbs{k}}}$                    \\
        \midrule
        \multirow{3}{*}{\tt reduce}          & reduce matrix to column vector                                            & $\grbv{w} \grbmask{\grbv{m}} $                                                                         & $\grbaccumeq{} \grbreduce{\grbplus}{\grbs{j}}{\grbm{A}}{:,\grbs{j}}$                \\
                                             & reduce matrix to scalar                                                   & $\grbs{s} $                                                                                            & $\grbaccumeq{} \grbreduce{\grbplus}{\grbs{i}\grbs{j}}{\grbm{A}}{\grbs{i},\grbs{j}}$ \\
                                             & reduce vector to scalar                                                   & $\grbs{s} $                                                                                            & $\grbaccumeq{} \grbreduce{\grbplus}{\grbs{i}}{\grbm{u}}{\grbs{i}}$                  \\
        \midrule
        \multirow{1}{*}{\tt transpose}       & transpose                                                                 & $\grbm{C} \grbmask{\grbm{M}} $                                                                         & $\grbaccumeq{} \grbm{A}\grbt$                                                       \\
        \midrule
        \tt kronecker                        & Kronecker multiplication                                                  & $\grbm{C} \grbmask{\grbm{M}}$                                                                          & $\grbaccumeq{} \grbkron{\grbm{A}, \grbm{B}}$                                        \\
        \midrule\midrule
        % </operations>
        % <methods>
        \multirow{2}{*}{\tt new}             & new matrix                                                                & \multicolumn{2}{l}{$\grbnewmatrix{\grbm{A}}{\grbplaceholder{TYPE}}{\grbplaceholder{PRECISION}}{n}{m}$}                                                                                       \\
                                             & new vector                                                                & \multicolumn{2}{l}{$\grbnewvector{\grbv{u}}{\grbplaceholder{TYPE}}{\grbplaceholder{PRECISION}}{n}$}                                                                                          \\
        \midrule
        \multirow{2}{*}{\tt build}           & matrix from tuples                                                        & $\grbm{C}\ $                                                                                           & $\grbassign \left\{ \grba{i}, \grba{j}, \grba{x} \right\} $                         \\
                                             & vector from tuples                                                        & $\grbv{w}\ $                                                                                           & $\grbassign \left\{ \grba{i}, \grba{x} \right\} $                                   \\
        \midrule
        \multirow{2}{*}{\tt extractTuples}   & \multirow{2}{*}{extract index/value arrays}                               & $ \left\{ \grba{i}, \grba{j}, \grba{x} \right\} $                                                      & $\grbassign \grbm{A} $                                                              \\
                                             &                                                                           & $ \left\{ \grba{i}, \grba{x} \right\} $                                                                & $\grbassign \grbv{u}   $                                                            \\
        \midrule
        \multirow{2}{*}{\tt dup}             & duplicate matrix                                                          & $\grbm{C} $                                                                                            & $\grbassign \grbm{A}$                                                               \\
                                             & duplicate vector                                                          & $\grbv{w} $                                                                                            & $\grbassign \grbv{u}$                                                               \\
        \midrule
        \multirow{2}{*}{\tt extractElement}  & \multirow{2}{*}{extract scalar element}                                   & $\grbs{s} $                                                                                            & $\ = \grbm{A}(\grbs{i}, \grbs{j})$                                                  \\
                                             &                                                                           & $\grbs{s} $                                                                                            & $\ = \grbv{u}(\grbs{i})$                                                            \\
        \midrule
        \multirow{2}{*}{\tt setElement}      & \multirow{2}{*}{set element}                                              & $\grbm{C}(\grbs{i}, \grbs{j}) $                                                                        & $\ = \grbs{s}$                                                                      \\
                                             &                                                                           & $\grbv{w}(\grbs{i})$                                                                                   & $\ = \grbs{s}$                                                                      \\
        \bottomrule
        % </methods>
    \end{tabular}
    \caption{GraphBLAS operations and methods based on \cite{DBLP:journals/toms/Davis19}.
        \emph{Notation:}
        Matrices and vectors are typeset in bold, starting with uppercase ($\grbm{A}$) and lowercase ($\grbv{u}$) letters, respectively.
        Scalars including indices are lowercase italic ($\grbs{k}$, $\grbs{i}$, $\grbs{j}$) while arrays are lowercase bold italic ($\grba{x}$, $\grba{i}$, $\grba{j}$).
        $\grbplus$ and $\grbtimes$ are the addition and multiplication operators forming a semiring and default to conventional arithmetic $+$ and $\times$ operators.
        $\grbaccum$ is the accumulator operator.
    }
    \label{tab:graphblas-notation}
\end{table*}


``guiding principles'': understandable (biggest probability of first guess to be correct), similar to existing notations~\cite{GraphBLASv13}

\todo{initial draft by Gabor}

\todo{explain masks}

\todo{explain replace/merge as per Scott's email}
% REPLACE: C :=              M .* (C + AB)
% MERGE:   C := (!M .* C) U [M .* (C + AB)]

\todo{revise Scott, Tim D, and Tim M}

\todo{consider adding semiring table}


\section{Algorithms}
\label{sec:algorithms}

%-------------------------------------------------------------------------------
\subsection{Breadth-First Search (BFS)}
%-------------------------------------------------------------------------------
\label{sec:bfs}

The breadth-first search (BFS)
builds on the observation that a vector-matrix multiplication $\grbv{f}\grbt\grbm{A}$ expresses
the navigation from the nodes selected by vector $\grbv{f}$ in the graph represented
by $\grbm{A}$.
The direction-optimizing push/pull BFS \cite{DBLP:conf/sc/BeamerAP12} is simple
to express in GraphBLAS \cite{DBLP:conf/icpp/YangBO18}.  If $\grbm{A}$ is held by row,
then $\grbv{f}\grbt\grbm{A}$ is a push step, while $\grbm{B}\grbv{f}$ is a pull step, where
$\grbm{B}=\grbm{A}\grbt$ is the explicit transpose of $\grbm{A}$, also held by row.
Other \grb libraries, \eg GraphBLAST, store both directions and perform
direction-optimization automatically~\cite{DBLP:journals/corr/abs-1908-01407}.
The push-only BFS is shown in
\autoref{alg:bfs-parents}, while the push/pull BFS is \autoref{alg:bfs-parents-do}.

The GraphBLAS BFS relies on the $\grbanysecondi$ %ANY-SECONDI
semiring to compute a single step,
$\grbv{q} \grbmask{\grbneg \grbstr{\grbv{p}}} = \grbv{q}\grbt\grbm{A}$, where $\grbv{q}$ is the current frontier
(using $\grbv{q}$ as short for queue),
$\grbv{p}$ is the parent vector, and $\grbm{A}$ is the adjacency matrix.
This step assigns the parents of newly nodes, which do not yet have a parent,
using the complemented structural mask $\grbmask{\grbneg \grbstr{\grbv{p}}}$.

Consider a matrix multiply for conventional linear algebra, where the $\grbplus$ %PLUS
monoid sums a set of $t$ entries to obtain a single scalar for computing
$c_{ij} = \sum a_{ik} b_{kj}$ in the matrix multiply $\grbm{C} = \grbm{A}\grbm{B}$.  The $\grbany$ %ANY
monoid performs the reduction of $t$ entries to a single number by merely selecting
any one of the $t$ entries as the result $c_{ij}$.  The selection is done
non-deterministically, allowing for a benign race condition.  In the BFS, this
corresponds to selecting any valid parent of a newly discovered node.  Indeed,
the creation of the $\grbany$ %ANY
operator was inspired by Scott Beamer's \verb'bfs.cc'
method in the GAP benchmark, which has the same benign race condition.  The $\grbany$ %ANY
monoid translates the concept of this benign race condition to construct a
valid BFS tree into a linear algebraic operation, suitable for implementation
in GraphBLAS.

The $\grbsecondi$ %SECONDI
operator is the multiplicative operator in the $\grbanysecondi$ %ANY-SECONDI
semiring, where the result of $a_{ik} b_{kj}$ is simply the index $k$ in the
semiring for $\grbm{C} = \grbm{A}\grbm{B}$.  This gives the id of the parent node for a newly
discovered node in the next frontier.  The $\grbany$ %ANY
monoid then selects any valid
parent $k$.
%
\begin{algorithm}[htb]
    \caption{Parents BFS.}
    \label{alg:bfs-parents}
    \DontPrintSemicolon
    \KwIn{$\grbm{A}, \grbs{startVertex}$}
    \Fn{ParentsBFS}{
        $\grbv{p}(\grbs{startVertex}) = \grbs{startVertex}$ \;
        $\grbv{q}(\grbs{startVertex}) = \grbs{startVertex}$ \;
        \For{$\grbs{level} = 1$ \KwTo $\grbnrows{\grbm{A}}-1$}{
            $\grbv{q}\grbt \grbmaskreplace{\grbneg \grbstr{\grbv{p}\grbt}} = \grbv{q}\grbt \grbanysecondi \grbm{A}$ \;
            $\grbv{p} \grbmask{\grbstr{\grbv{q}}} = \grbv{q}$ \;
            \If{$\grbnvals{\grbv{q}} = 0$}{return}
        }
    }
\end{algorithm}

\begin{algorithm}[htb]
    \caption{Direction Optimizing Parent BFS.}
    \label{alg:bfs-parents-do}
    \DontPrintSemicolon
    \KwIn{$\grbm{A}, \grbm{A}\grbt, \grbs{startVertex}$}
    \Fn{DirectionOptimizingBFS}{
        $\grbv{q}(\grbs{startVertex}) = 0$ \;
        \For{$\grbs{level} = 1$ \KwTo $\grbnrows{\grbm{A}}-1$}{
            \If{$\mathit{Push}(\grbm{A}, \grbv{q})$}{ %\Comment{Decide to push/pull}
                $\grbv{q}\grbt \grbmaskreplace{\grbneg \grbstr{\grbv{p}\grbt}} = \grbv{q}\grbt \grbanysecondi \grbm{A}$
            }
            \Else{
                $\grbv{q} \grbmaskreplace{\grbneg \grbstr{\grbv{p}}} = \grbm{A}\grbt \grbanysecondi \grbv{q}$
            }
            $\grbv{p} \grbmask{\grbstr{\grbv{q}}} = \grbv{q}$ \;
            \If{$\grbnvals{\grbv{q}} = 0$}{return}
        }
    }
\end{algorithm}


%-------------------------------------------------------------------------------
\subsection{Betweenness Centrality (BC)}
%-------------------------------------------------------------------------------
\label{sec:bc}

\begin{algorithm}[htb]
	\caption{Betweenness centrality.}
	\label{alg:bc}
	%\KwData{...}
	%\KwResult{...}
	\Fn{BrandesBC}{
		\Comment{The $\grbm{NumSp}$ structure holds the number of shortest paths for each node and starting vertex discovered so far.}
		\Comment{Initialized to source vertices.}
		$\grbm{NumSp} \grbassign \{ \grba{s}, [1, 1, \ldots, 1] \}$ \;
		\Comment{The $\grbm{Frontier}$ holds the number of shortest paths for each node and starting vertex discovered so far.}
		$\grbm{Frontier} \grbassign \{ \grba{s}, [1, 1, \ldots, 1] \}$ \;
                $\grbm{Frontier} \grbmaskreplace{\grbneg\grbstr{\grbm{NumSp}}} =\grbm{Frontier} \grbplusfirst \grbm{A}$ 
		\Comment{The $\grbm{Sigmas}$ matrices store frontier information for each level of the BFS phase.}
		\Comment{BFS phase (forward sweep)}
		\For{$\grbs{d} = 0$ \KwTo $\grbnrows{\grbm{A}}$}{
			\Comment{$\grbm{Sigmas}[\grbs{d}](:, \grbs{s}) = \grbs{d}^{\textrm{th}}$ level frontier from source vertex $\grbs{s}$}
			$\grbnewmatrix{\grbm{Sigmas}[\grbs{d}]}{\grbbool}{}{\grbs{n}}{\grbs{nsver}} $ \;
                        % the '(:, :)' is there to express 'assign' instead of 'dup':
			$\grbm{Sigmas}[\grbs{d}](:,:) = \grbm{Frontier} $ \Comment*{Convert matrix to Boolean}
			$\grbm{NumSp} \grbaccumeq{+} \grbm{Frontier}$ \Comment*{Accumulate path counts}
			$\grbm{Frontier}\grbmaskreplace{\grbneg\grbstr{\grbm{NumSp}}} = \grbm{Frontier} \grbplusfirst \grbm{A}$ \Comment*{Update frontier}
                        \If{$\grbnvals{\grbm{Frontier}} = 0$}{break}
		}

		$\grbnewmatrix{\grbm{BCU}}{\grbfloat}{32}{\grbs{n}}{\grbs{nsver}}$ \;
		$\grbm{BCU}(:) = 1.0$ \Comment*{Make $\grbm{BCU}$ dense, initialize all elements to $1.0$}
		$\grbnewmatrix{\grbm{W}}{\grbfloat}{32}{\grbs{n}}{\grbs{nsver}}$ \;

		\Comment{Tally phase (backward sweep)}
		\For{$\grbs{i} = \grbs{d} - 1$ \KwDownto $0$}{
			$\grbm{W}\grbmaskreplace{\grbstr{\grbm{Sigmas}[\grbs{i}]}} = \grbm{BCU}  \grbdiv_\cap \grbm{NumSp}$ \;
			$\grbm{W}\grbmaskreplace{\grbstr{\grbm{Sigmas}[\grbs{i} - 1]}} = \grbm{W} \grbplusfirst \grbm{A}\grbt$
                        \Comment*{Add contributions by successors and mask with that BFS level's frontier.}
			$\grbm{BCU} \grbaccumeq{+} \grbm{W} \times_\cap \grbm{NumSp}$ \;
		}

		\Comment{Row reduce $\grbm{BCU}$ and subtract $\grbs{nsver}$ from every entry to account for 1 extra value per $\grbm{BCU}$ row element}
		$\grbv{delta}(:) = -\grbs{nsver}$ \;
		$\grbv{delta} \grbaccumeq{+} \grbreduce{+}{\grbs{j}}{\grbm{BCU}}{:, \grbs{j}} $ \;
	}
\end{algorithm}

%
The vertex betweenness-centrality metric is a weighted measure of the number of
shorted paths that go through any given node,
\[ \sum_{s \ne i \ne t} \frac{\sigma (s, t|i)}{\sigma{s,t}}, \]
where $\sigma(s,t)$ is the total number of shortest paths from node $s$ to $t$,
and $\sigma(s,t|i)$ is the total number of shortest paths from node $s$ to $t$
that pass through node $i$.  This is expensive to compute, so in practice,
a subset of source nodes $s$ are chosen at random (a {\em batch}).

Like the BFS, direction-optimization is incredibly simple to add to the LAGraph
algorithm for batched betweenness-centrality (BC).
It only requires a simple heuristic to determine which
direction to use, followed by masked matrix-matrix multiplication with the
matrix or its transpose: $\grbm{F} \grbmask{\grbneg \grbstr{\grbm{P}}} = \grbm{F}\grbm{B}\grbt$ (the pull) or $\grbm{F}
\grbmask{\grbneg \grbstr{\grbm{P}}} = \grbm{F} \grbm{A}$ (the push), where $\grbm{A}$ is the adjacency matrix of
the graph and $\grbm{B} = \grbm{A}\grbt$ is its explicit transpose, $\grbm{F}$ is the frontier, and the
complemented structural mask $\grbneg \grbstr{\grbm{P}}$ is the set of unvisited nodes.  The multiplication
$\grbm{F} \grbm{B}\grbt$ relies on the descriptor to represent the transpose of $\grbm{B}$, which is not
explicitly transposed.  In the backward phase, the pull step is $\grbm{W} = \grbm{W} \grbm{A}\grbt$ while
the push is $\grbm{W} = \grbm{W} \grbm{B}$, where $\grbm{W}$ is the 4-by-$n$ matrix in which centrality is
accumulated.

To simplify the presentation of the entire BC algorithm, \autoref{alg:bc} does
not show the direction-optimization.  It is the same transformation as the pair
of BFS algorithms, where the push-only step (line 5 of
\autoref{alg:bfs-parents}), is expanded to a push/pull hueristic (lines 4-7 of
\autoref{alg:bfs-parents-do}).

%-------------------------------------------------------------------------------
\subsection{PageRank (PR)}
%-------------------------------------------------------------------------------
\label{sec:pagerank}

PageRank (PR) computes the importance of each node as a recursively-defined
metric: a web page is important if important pages link to it.
\autoref{alg:pagerank} shows the GraphBLAS implemenation of PR as specified in
the GAP Benchmark.  It uses the $\grbplussecond$ semiring, where
$\grboperator{second}(x,y)=y$, so it can ignore any edge weights in the input
matrix.  The PR in GAP does not properly handle dangling vertices in the graph.
The Graphalytics benchmark has a PageRank variant which avoids this
problem~\cite{DBLP:journals/corr/abs-2011-15028}.  We have included this
version to compare its performance with the GAP benchmark algorithm
\verb'pr.cc'.
%
\begin{algorithm}[htb]
    \caption{PageRank (as specified in the GAPBS).}
    \label{alg:pagerank}
    \KwData{$\grbm{A} \in \grbbool^{n \times n}$ \Comment*{adjacency matrix}}
    \KwDataXX{$\grbs{damping}$ \Comment*{damping factor}}
    \KwDataXX{$\grbs{tol}$ \Comment*{stopping tolerance}}
    \KwDataXX{$\grbs{itermax}$\Comment*{maximum number of iterations}}
    \KwResult{$\grbv{r} \in \grbfloat^n$}
    \Fn{PageRank}{
        % $\grbv{pr}(:) = 1 / n$ \;
        % $\grbv{outdegrees} = [\grbplus_{\grbs{j}} \grbm{A}(:, \grbs{j})] $ \;
        % %$\grbv{nondangling} = [\grblor_{\grbs{j}} \grbm{A}(:, \grbs{j})] $ \;

        % \For{$\grbs{k} = 1$ \KwTo $\grbs{numIterations}$}{
        %     $\grbv{importance} = \grbv{pr} \grbdiv \grbv{outdegrees}$ \;
        %     $\grbv{importance} = \grbf{times}{\grbv{importance}, \grbs{\grbalpha}} $ \Comment*{apply the $\grbf{times}{x, s} = x \cdot s$ operator}
        %     $\grbv{importance} = \grbv{importance} \grbplustimes \grbm{A} $ \;

        %     $\grbv{danglingVertexRanks} \grbmask{\grbneg{\grbv{outdegrees}}} = \grbv{pr}(:) $ \;
        %     $\grbs{totalDanglingRank} = \grbfrac{\grbs{\grbalpha}}{\grbs{n}} \grbtimes \grbreduce{\grbplus}{\grbs{i}}{\grbv{danglingVertexRanks}}{\grbs{i}} $ \;

        %     $\grbv{pr} = \grbfrac{1-\grbs{\grbalpha}}{\grbs{n}} \grbplus \grbs{totalDanglingRank} $ \;
        %     $\grbv{pr} = \grbv{pr} \grbplus \grbv{importance} $ \;
        % }
        $\grbs{teleport} = \frac{1 - \alpha}{n}$ \;
        $\grbv{r}(0:n-1) = \frac{1}{n}$, $\grbv{t} = \grbfloat^n$ \;
        $\grbv{d_{out}} = \grbreduce{+}{j}{\grbm{A}}{:, j}$ \Comment{precomputed rowdegree}
        $\grbv{d} = \grbv{d_{out}} \grbewisemult{\grbdiv} \grbs{damping}  $ \Comment{prescale with damping}

        %$\grbv{pr}(:) = 1 / n$ \;
        
        \For{$\grbs{k} = 1$ \KwTo $\grbs{numIterations}$}{
            swap $\grbv{t}$ and $\grbv{r}$ \Comment{$\grbv{t}$ is now the prior rank}
            $\grbv{w} = \grbv{t} \grbewisemult{\grbdiv} \grbv{d}$ \;
            $\grbv{r}(0:n-1) = \grbs{teleport}$ \;
            $\grbv{r} += \grbm{A}\grbt \grbv{w}$ \;
            $\grbv{t} -= \grbv{r}$ \;
            $\grbv{t} = abs(\grbv{t})$ \;
            \If{$\grbreduce{+}{j}{\grbv{t}}{:} < \grbs{tol}$}{return}
        }
    }
\end{algorithm}


%-------------------------------------------------------------------------------
\subsection{Single-Source Shortest-Paths (SSSP)}
%-------------------------------------------------------------------------------
\label{sec:sssp}

A Delta-Stepping Single-Source-Shortest-Path algorithm in GraphBLAS is shown in
\autoref{alg:sssp-delta-stepping}.  It relies on the $\grbminplus$ semiring.
Since it is a fairly complex algorithm, refer to
\cite{DBLP:conf/ipps/SridharBMSLM19} for a description of the method.
%
\begin{algorithm}[htb]
	\caption{SSSP (delta-stepping).}
	\label{alg:sssp-delta-stepping}
	\KwData{\;
		$\quad \grbm{A}, \grbm{A_H}, \grbm{A_L} \in \grbmatrixtype{\grbfloat}{}{\grbcnt{V}}{\grbcnt{V}} $ \;
		$\quad \grbs{s}, \grbs{i} \in \grbscalartype{\grbuint}{} $ \;
		$\quad \Delta \in \grbscalartype{\grbfloat}{} $ \;
		$\quad \grbv{t}, \grbv{t_{Req}} \in \grbvectortype{\grbfloat}{}{\grbcnt{V}} $ \;
		$\quad \grbv{t_{B_i}}, \grbv{e} \in \grbvectortype{\grbuint}{}{\grbcnt{V}} $ \; % e was S in the original paper, changed to lowercase for consistency
	}
	%\KwResult{...}
	\Fn{DeltaStepping}{
		$\grbm{A_L} = \grbm{A}\grbselect{0 < \grbm{A} \leq \Delta} $ \;
		$\grbm{A_H} = \grbm{A}\grbselect{\Delta < \grbm{A}} $ \;
		$\grbv{t}(:) = \infty $ \;
		$\grbv{t}(\grbs{s}) = 0 $ \;
		\While{$\grbnvals{ \grbv{t}\grbselect{\grbs{i} \Delta \leq \grbv{t}} } \neq 0$}{
			$\grbs{s} = 0 $ \;
			$\grbv{t_{B_i}} = \grbv{t} \grbselect{\grbs{i} \Delta \leq \grbv{t} < (\grbs{i} + 1) \Delta}$ \;
			\While{$\grbv{t_{B_i}} \neq 0$}{
				$\grbv{t_{Req}} = \grbv{t} \grbewisemult{\times} \grbv{t_{B_i}}$ \;
				$\grbv{t_{Req}} = \grbm{A_L\grbt} \grbminplus \grbv{t_{Req}}$ \;
				$\grbv{e} = \grbv{t} \grbselect{0 < \grbv{e} \grbplus \grbv{t_{B_i}} }$ \;
				$\grbv{t_{B_i}} = \grbv{t} \grbselect{\grbs{i} \Delta \leq \grbv{t_{Req}} < (\grbs{i} + 1) \Delta}$ \;
				$\grbv{t_{B_i}} = \grbv{t_{B_i}} \grbselect{\grbv{t_{Req}} < \grbs{t}}$ \;
				$\grbv{t} = \grbv{t} \grbewiseadd{\grbmin} \grbv{t_{Req}}$ \;
			}
			$\grbv{t_{Req}} = \grbm{A_H\grbt} \grbminplus (\grbv{t} \grbewisemult{\times} \grbv{e})$ \;
			$\grbv{t} = \grbv{t} \grbewiseadd{\grbmin} \grbv{t_{Req}}$ \;
			$\grbs{i} = \grbs{i} + 1 $ \;
	}
	}
\end{algorithm}


%-------------------------------------------------------------------------------
\subsection{Triangle Counting (TC)}
%-------------------------------------------------------------------------------
\label{sec:triangle-count}

The triangle counting (TC) problem is to compute the number of unique cliques
of size 3 in a graph.  The TC algorithm is shown in
\autoref{alg:triangle-count-sandiadot}, based on \cite{8091043}.
%
\begin{algorithm}[htb]
	\caption{Triangle count (``SandiaDot'' variant).}
	\label{alg:triangle-count-sandiadot}
	%\KwData{...}
	%\KwResult{...}
	\Fn{TriangleCount}{
		$\grbm{L} = \grbtril{\grbm{A}}$ \;
		$\grbm{C}\grbmask{\grbm{L}} = \grbm{L} \grbplustimes \grbm{L}$ \;
		$\grbs{t} = \grbreduce{\grbplus}{\grbs{i}\grbs{j}}{\grbm{C}}{\grbs{i}, \grbs{j}}$ \;
	}
\end{algorithm}

%
It starts with a heuristic that decides when
to sort the input graph by ascending degree.  Next, it constructs the lower and
upper triangular part and computes a masked matrix multiply using the
$\grbpluspair$ semiring.  Internally, a dot product method is used in SS:GrB,
because $\grbm{U}$ is transposed via the descriptor.  The $\grboperator{pair}$
is the simple function $\grboperator{pair}(x,y)=1$.  When used in a semiring,
it acts like the $\grboperator{times}$ operator of the conventional semiring,
except that it can ignore the values of its inputs and treat them both as 1.
This semiring is useful for structural computations, such as triangle counting,
when the edge weights of a graph may be present but should be ignored in a
particular algorithm.

%-------------------------------------------------------------------------------
\subsection{Connected Components}
%-------------------------------------------------------------------------------
\label{sec:connected-components}

The connected components algorithm in LAGraph (\autoref{alg:fastsv})
is written by Zhang, Azad, and Bulu{\c{c}}
\cite{ZHANG202014,DBLP:conf/ppsc/ZhangAH20}.  The method maintains a forest of
trees represented by a parent vector, and interatively merges trees until no
more merging is possible.  The method as shown in \autoref{alg:fastsv} is a
simplified variant that it operates on the entire graph.  In the LAGraph
version, a subgraph is constructed first, and the method finds the connected
components of the subgraph, and then operates on the entire graph.
\begin{algorithm}[htb]
	\caption{Connected components (FastSV).}
	\label{alg:fastsv}
	%\KwData{...}
	%KwResult{...}
	\Fn{FastSV}{
        $\grbs{n} = \grbnrows{\grbm{A}}$ \;
        $\grbv{gf} = \grbv{f}$ \;
        $\grbv{dup} = \grbv{gf}$ \;
        $\grbv{mngf} = \grbv{gf}$ \;
        $\{ \grba{i}, \grba{x} \} \grbassign \grbv{f}$ \;
        \Repeat{$\grbs{sum} == 0$}{
            \Comment{Step 1: Stochastic hooking}
            $\grbv{mngf} = \grbv{mngf} \grbmin \grbm{A} $ \;
            $\grbv{mngf} = \grbv{mngf} \grbsecondmin \grbv{gf}$ \;
            $\grbv{f}(\grba{x}) = \grbv{f} \grbmin \grbv{mngf} $ \;
            \Comment{Step 2: Aggressive hooking}
            $\grbv{f} = \grbv{f} \grbmin \grbv{mngf} $ \;
            \Comment{Step 3: Shortcutting}
            $\grbv{f} = \grbv{f} \grbmin \grbv{gf} $ \;
            \Comment{Step 4: Calculate grandparents}
            $\{ \grba{i},  \grba{x} \} \grbassign \grbv{f}$ \;
            $\grbv{gf} = \grbv{f}(\grba{x})$ \;
            \Comment{Step 5: Check termination}
            $\grbv{diff} = \grbv{dup} \neq \grbv{gf} $ \; % \neq or 'isne'?
            $\grbs{sum} = [+_{\grbs{i}} \grbv{diff}(\grbs{i}) ] $ \;
            $\grbv{dup} = \grbv{gf}$ \; %LAGRAPH_OK (GrB_assign (dup, 0, 0, gp, GrB_ALL, 0, 0));
        }
	}
\end{algorithm}




\section{Utility Fuctions}
\label{sec:utility}

LAGraph includes a set of utility functions that operate
on a graph.  All function names are prefixed with \verb'LAGraph_'
so we exclude that prefix in the names below, for brevity.

% Here is a rough categorization of the utilities (not all included yet)
\begin{itemize}

\item {\bf Graph Properties:}
    % If they operate on the LAGraph\_Graph cached properties consider a
    % consistent naming scheme like LAGraph\_Property\_XXX
    An \verb'LAGraph_Graph' includes cached properties which can be
    assigned by Basic methods, or which are required by Advanced methods.

    % \begin{itemize}
      % \item
      \verb'DeleteProperties' clear all properies,
      % \item
      \verb'Property_AT' computes the transpose of the adjacency matrix \verb'G->A',
      % \item
      \verb'Property_RowDegree' computes the row degrees of \verb'G->A',
      % \item
      \verb'Property_ColDegree' computes the column degrees of \verb'G->A',
      % \item
      and
      \verb'Property_ASymmetricPattern' determines if the pattern of \verb'G->A' is symmetric or unsymmetric.
       %   (consider Property\_ClearAll), Property\_AT, Property\_AssymetricPattern, Property\_RowDegree, Property\_ColDegree
    % \end{itemize}

\item {\bf Display and debug:}
    \verb'CheckGraph' checks the validity of a graph.
    Since the graph is not opaque, a user application is able to change a graph
    arbitrarily and thus might make it an invalid object.
    \verb'DisplayGraph' displays a graph and its properties.

\item {\bf Memory management:}
    Wrappers for \verb'malloc', \verb'calloc', \verb'realloc', and \verb'free',
    allowing a user application to select the memory manager to be used.
    These default to the ANSI C11 library functions.
    % malloc, calloc, realloc, free stuff.  This might be covered in the design
    % decisions section

% \item {\bf Threading:}
    % Get/SetNumThreads. Shouldn’t this be part of init, or an
    % LAGraph\_Context?  Should this also be discussed in design decisions.
    % GraphBLAS threading is one thing, but this seems to be LAGraph threading
    % (outside of GraphBLAS calls).  I could see this as algorithm specific and
    % not a general util.  The only way to leverage these inside algorithms is
    % to either set a global property that all algorithms have access to, or
    % creating a context that is passed to algorithms.

\item {\bf Graph I/O:}
    % \begin{itemize}
    % \item
    \verb'BinRead' and \verb'BinWrite' read/write a \verb'GrB_Matrix' in binary form.
    % \item
    \verb'MMRead' and \verb'MMWrite' read/write a \verb'GrB_Matrix' in Matrix Market form.
    % , MMRead (we are missing BinWrite, MMWrite), DisplayGraph (is this a
    % pretty print of the matrix or all the properties, this should also take
    % FILE*)
    % \end{itemize}

\item {\bf Matrix operations:}
    \verb'Pattern' returns a boolean matrix containing the pattern of a matrix.
    % Pattern (MakePattern, GetPattern).  Not sure how to categorize these
% \item {\bf Matrix comparison}:
    \verb'IsEqual' determines if two matrices are equal.  It selects the appropriate
    \verb'GrB_EQ_T' operator that matches the matrix type, and then calls \verb'IsAll'.
    \verb'IsAll' compares two matrices and returns false if
    the pattern of the two matrices differ.  It then uses a given comparator operator to
    compare all pairs of entries, and returns true if all comparisons return true.
    % IsEqual, IsAll. Are these graph or matrix utils? IsAll
    % is an ambiguous (maybe misleading) name because it seems to be a generic
    % comparator (consider CompareGraphs).

\item {\bf Degree operations:}
    % SortByDegree, SampleDegree  (maybe belongs grouped here as it is computing properties, but maybe not cached)
    \verb'SortByDegree' returns a permutation that sorts a graph by its row/column degrees, and
    \verb'SampleDegree' computes a quick estimate of the mean and median row/column degrees.

\item {\bf Error handling:}
    \verb'LAGraph_TRY' and \verb'GrB_TRY' are helper macros for a simple try/catch
    mechanism.  They require the user application to define \verb'LAGraph_CATCH'
    and \verb'GrB_CATCH'.
    % TRY/CATCH, MIN/MAX, tic/toc, TypeName, KindName

\item {\bf Other:}
    \verb'TypeName' returns a string with the name of a \verb'GrB_Type'.
    \verb'KindName' returns a string with of graph kind (directed or undirected).
    \verb'Tic' and \verb'Toc' provide a portable timer.
    \verb'Sort1', \verb'Sort2', and \verb'Sort3' sort 1, 2, or 3 integer arrays.

% \item {\bf Consider for removal} Things that may be implementation detail and could be buried:

% \begin{itemize}

%  \item Sort1/2/3 - currently only Sort2 is used.  I don't see a strong need to include these as part of the public API at this time.

%   \item Random15/60 - I see Random64 being the most widely usable. These can easily be buried as well, and I would still suggest Random64 (Random60 seems too SuiteSparse:GraphBLAS specific.

%   \item Test\_ReadProblem (move to the TestArea)

% \end{itemize}
\end{itemize}



\section{Evaluation}
\label{sec:evaluation}

The performance of LAGraph can only be considered in context of an
implementation of the underlying GraphBLAS library.  This is discussed in
Section~\ref{sec:extensions}, followed by performance results of the new
LAGraph API on the 6 algorithms in the GAP Benchmark
\cite{DBLP:conf/sc/BeamerAP12}.

\subsection{SuiteSparse Extensions}
\label{sec:extensions}

In a prior paper (\cite{DBLP:conf/iiswc/AzadABBCDDDDFGG20}), an early draft of
SS:GrB, (SuiteSparse:GraphBLAS v4.0.0, Aug 2020), was compared with the GAP benchmark
\cite{DBLP:conf/sc/BeamerAP12} and four other graph libraries.  This prior
version of SS:GrB included two primary data structures for its sparse matrices:
compressed sparse vector, and a hypersparse variant
\cite{DBLP:conf/ipps/BulucG08}, both held by row or by column.  It included a
draft implementation of a bitmap data structure that could only be used in a
prototype breadth-first search.  Since then, SuiteSparse:GraphBLAS v4.0.3 has
been released, with full support for bitmap and full matrices for all its
operations.  In an $m$-by-$n$ bitmap matrix, the values are held in a full
array of size $mn$, and another \verb'int8_t' array of size $mn$ holds the
sparsity pattern of the matrix.  A full matrix is a simple dense array of size
$mn$.

The bitmap format is particularly important for the ``pull'' phase of an
algorithm, as used in direction-optimizing breadth-first-search
\cite{DBLP:conf/sc/BeamerAP12,DBLP:conf/icpp/YangBO18}.  The GAP benchmark suite uses this method by
holding its frontier as a bitmap in the pull step and as a list in the push
step. The GAP BFS was typically the fastest BFS amongst the 6 graph
libraries compared in \cite{DBLP:conf/iiswc/AzadABBCDDDDFGG20} (for 4 of the 5
benchmark graphs).  With the addition of the bitmap format to SS:GrB,
LAGraph+SS:GrB is able to come within a factor of 2 or so of the performance of
the highly-tuned BFS GAP benchmark (see the results in the next section), for
those 4 graphs.  At the same time, however, the BFS is very easily expressed in
LAGraph as easy-to-read and easy-to-write code.  This enables non-experts to
obtain a reasonably high level of performance with modest programming effort
when writing graph algorithms.

Additional optimizations added to SS:GrB in the past year include a {\em lazy
sort}.  Normally, SS:GrB keeps its vectors sorted (row vectors in a CSR matrix,
or column vectors if the matrix is held by column), with entries sorted in
ascending order of column or row index, respectively.  This simplifies \
algorithms that operate on a \verb'GrB_Matrix'.  However, some algorithms
naturally produce a jumbled result (matrix multiply in particular), while others
are tolerant of jumbled input matrices.  We thus allow the sort to
be left pending.  The lazy sort joins two other kinds of pending work in
SS:GrB: {\em pending tuples} and {\em zombies}~\cite{DBLP:journals/toms/Davis19}.
A pending tuple is an entry
that is held inside a matrix in an unsorted list, awaiting insertion into the
CSR/CSC format of a \verb'GrB_Matrix'.  A zombie is the opposite: it is an
entry in the CSR/CSC format that has been marked for deletion, but has not yet
been deleted from the matrix.  With  lazy sort, the sort is postponed until
another algorithm requires sorted input matrices.  If the sort is lazy enough,
it might never occur, which is the case for the LAGraph BFS and BC.

Positional binary operators have also been added,
such as the $\grbanysecondi$ semiring, %ANY-SECONDI
which makes the BFS much faster.

\subsection{Performance Results}

We ran our benchmarks on an NVIDIA DGX Station (donated to Texas A\&M by
NVIDIA).  It includes a 20-core Intel(R) Xeon(R)
CPU E5-2698 v4 @ 2.20GHz, with 40 threads.  All codes were compiled with gcc
5.4.0 (-O3).  All default settings were used, which means  hyperthreading
was enabled.  The system has 256GB of RAM in a single socket.
%  There are NUMA effects in a single socket system with 20 cores .... so I deleted this parenthetical comment (no NUMA effects).
LAGraph (Feb 15, 2021) and SuiteSparse:GraphBLAS 4.0.4-draft (Feb 15, 2021) were
used.  The NVIDIA DGX Station includes four P100 GPUs, but no GPUs were used by
this experiment (a GPU-accelerated SS:GrB is in progress).
Table~\ref{table:results} lists the run time (in seconds) for the GAP benchmark
and LAGraph+SS:GrB for the 6 algorithms on the 5 benchmark matrices.
The benchmark matrices are listed in Table~\ref{table:matrices}.

\begin{table}
\begin{center}
\begin{tabular}{|l|rrrrr|}
\hline
Algorithm :    &   \multicolumn{5}{c|}{graph, with run time in seconds}  \\
 package       &   Kron    &   Urand   &   Twitter  &  Web    &    Road  \\
\hline
BC   : GAP     &  31.52    &  46.36    &  10.82     &  3.01   &    1.50  \\
% BC : Jan26   &  26.85    &  31.78    &  10.04     &  9.25   &   51.91  \\
% BC : Feb6    &  24.75    &  30.66    &   9.28     &  8.97   &   43.66  \\
% BC : Feb7    &  24.52    &  30.69    &   9.11     &  8.43   &   34.06  \\
BC   : SS      &  23.61    &  32.69    &   9.25     &  8.20   &   34.40  \\     % Feb14
\hline
BFS  : GAP     &    .31    &    .58    &    .22     &   .34   &     .25  \\
%BFS : Jan26   &    .52    &   1.31    &    .33     &   .67   &    3.33  \\
%BFS : Feb6    &    .52    &   1.20    &    .33     &   .66   &    3.36  \\
BFS  : SS      &    .52    &   1.22    &    .33     &   .66   &    3.32  \\     % Feb7
\hline
PR   : GAP     &  19.81    &  25.29    &  15.16     &  5.13   &    1.01  \\
%PR  : Jan26   &  21.96    &  27.75    &  17.22     &  9.30   &    1.34  \\
%PR  : Feb6    &  21.67    &  27.60    &  17.14     &  9.34   &    1.32  \\
PR   : SS      &  22.17    &  27.71    &  17.21     &  9.30   &    1.34  \\     % Feb7
\hline
CC   : GAP     &    .53    &   1.66    &    .23     &   .22   &     .05  \\
%CC  : Jan26   &   3.42    &   4.59    &   1.48     &  1.97   &    1.00  \\
%CC  : Feb6    &   3.35    &   4.56    &   1.47     &  1.96   &     .97  \\
CC   : SS      &   3.36    &   4.47    &   1.47     &  1.97   &     .98  \\     % Feb7
\hline
SSSP : GAP     &   4.91    &   7.23    &   2.02     &   .81   &     .21  \\
%SSSP: Jan26   &  17.62    &  25.62    &   8.44     &  9.67   &   48.49  \\
%SSSP: Feb6    &  17.58    &  25.60    &   8.18     &  9.60   &   48.24  \\
SSSP : SS      &  17.37    &  25.54    &   8.54     &  9.61   &   46.79  \\     % Feb7
\hline
TC   : GAP     & 374.08    &  21.83    &  79.58     & 22.18   &     .03  \\
%TC  : Jan26   & 943.47    &  34.10    & 242.36     & 35.15   &     .29  \\
%TC  : Feb6    & 922.35    &  33.97    & 238.71     & 34.67   &     .23  \\
TC   : SS      & 917.99    &  34.01    & 239.58     & 34.65   &     .23  \\     % Feb7
\hline
\end{tabular}
\caption{Run time of GAP and LAGraph+SS:GrB
\vspace{-2.5ex}
\label{table:results}}
\end{center}
\end{table}

\begin{table}
\begin{center}
\begin{tabular}{|l|rrr|}
\hline
graph   & nodes        & entries in $A$ & graph kind \\
\hline
Kron    & 134,217,726 &  4,223,264,644 &  undirected   \\
Urand   & 134,217,728 &  4,294,966,740 &  undirected   \\
Twitter &  61,578,415 &  1,468,364,884 &  directed     \\
Web     &  50,636,151 &  1,930,292,948 &  directed     \\
Road    &  23,947,347 &     57,708,624 &  directed     \\
\hline
\end{tabular}
\caption{Benchmark matrices\label{table:matrices}
(\url{https://sparse.tamu.edu/GAP})}
\vspace{-2.5ex}
\end{center}
\end{table}

% Notes: TC in SS is SandiaDot method only

With the addition of the bitmap (needed for the pull step), the
push/pull optimization in BC resulted in a nearly 2x performance gain in the
GraphBLAS method for the largest matrices, as compared to the SS:GrB version
used for the results presented in \cite{DBLP:conf/iiswc/AzadABBCDDDDFGG20}.

With this change, the BC method in LAGraph+SS:GrB is not only expressible in a
simple, elegant code, but it is also faster than the highly-tuned GAP benchmark
method, \verb'bc.cc', for the three largest matrices (1.3x for Kron, 1.5x for
Urand, and 1.2x for Twitter).

% Feb7 update:  the sort is completely lazy in BC.

% Feb6 results: the sort partially lazy in BC.

% For the Jan26 results:
% We expect the BC in LAGraph+SS:GrB to become faster still in the next
% release because we have not yet fully exploited the lazy sort.  The frontier
% matrix $\grbm{F}$ is left jumbled by the lazy sort, but it is sorted right away by a
% subsequent assignment.  Most uses of \verb'GrB_assign' are intolerant of
% jumbled input matrices, so it sorts them on input.  However, \verb'GrB_assign'
% includes about 40 internal variations, a few of which do not actually require
% sorted input matrices.  The particular method used in \verb'GrB_assign' in the
% LAGraph BC method is one of those methods, so this would be simple to exploit.
% With this change, the sort would be so lazy that it would {\em never} occur.
% The frontier would be computed, left jumbled, used in subsequent computations,
% and then recomputed (and thus discarded), just as we currently do in the
% LAGraph BFS.

The bitmap format (which makes push/pull optimization
simple to express, and fast) and the $\grbanysecondi$ %ANY-SECONDI
semiring, the BFS of a
directed or undirected graph is easily expressed in GraphBLAS, and has a
performance that is only about 1.5x to 2x slower than the GAP benchmark.  We expect
the remaining performance gap arises from two issues:

\begin{enumerate}
\item
GAP assumes that the graph has fewer than $2^{32}$ nodes and edges, and
thus uses 32-bit integers throughout.  GraphBLAS is written for larger
problems, and thus relies solely on 64-bit integers.  This cannot be easily
changed in GraphBLAS, but rather than ``fixing'' GraphBLAS to use smaller
integers, the GAP benchmark suite should be updated for larger
graphs.  In the current GAP benchmark graphs, two graphs are
chosen with almost exactly 4 billion edges.  Graphs of current interest in
large data science can easily exceed $2^{32}$ nodes and edges \cite{9286235}.

\item In GraphBLAS, the BFS must be expressed as two calls.  The first computes
$\grbv{q} \grbmask{\grbneg \grbv{p}} = \grbv{q}\grbt \grbm{A}$, and the second updates the parent vector,
$\grbv{p} \grbmask{\grbstr{\grbv{q}}} = \grbv{q}$:

{\footnotesize
\begin{verbatim}
  GrB_vxm (q, p, NULL, semiring, q, A, GrB_DESC_RSC) ;
  GrB_assign (p, q, NULL, q, GrB_ALL, n, GrB_DESC_S) ; \end{verbatim}}

In GAP's \verb'bfs.cc', these two steps are fused, and the
matrix-vector multiplication can write its result directly into the parent vector
\verb'p'.  This could be implemented in a future GraphBLAS library, since the
GraphBLAS API allows for a non-blocking mode where work is queued and done
later, thus enabling a fusion of these two steps.  SS:GrB exploits the
non-blocking mode (for its lazy sort, pending tuples, and zombies) but does not
{\em yet} exploit the fusion of \verb'GrB_vxm' and \verb'GrB_assign'.  We
intend to exploit this in the future.
\end{enumerate}

Note that for the Road graph,
LAGraph+SS:GrB is quite slow for all but PageRank (PR).
The primary reason for this is the high diameter of the Road graph
(about 6980).  This requires 6980 iterations of GraphBLAS in the BFS, each with
a tiny amount of work.  Each call to GraphBLAS does several \verb'malloc' and
\verb'free's, and in some cases the workspace must be initialized.  A future
version of SS:GrB is planned that will eliminate this work entirely, by
implementing an internal memory pool.  There may be other overheads, but we
hope that a memory pool, fusion to fully exploit non-blocking mode, and other
optimizations will address this large performance gap for the Road graph for
these algorithms.

LAGraph+SS:GrB is also up to 3x slower than the GAP for the triangle counting
problem (for all but the Road graph, where it is even slower).  This
performance gap can be eliminated entirely in the future, if the \verb'GrB_mxm'
and \verb'GrB_reduce' are combined in a single fused step, by a full
exploitation of the GraphBLAS non-blocking mode.  The current method computes
$\grbm{C} \grbmask{\grbstr{\grbm{L}}} = \grbm{L}\grbm{U}\grbt$, followed by the reduction of $\grbm{C}$ to a single
scalar.  The matrix $\grbm{C}$ is then discarded.  All that GraphBLAS needs is a fused
kernel that does not explicitly instantiate the temporary matrix $\grbm{C}$.
This is permitted by the GraphBLAS C API Specification, but not yet implemented
in SS:GrB.



\section{Conclusion}
\label{sec:conclusion}

In this paper we introduced the LAGraph library, the rationale behind our design,  
and a performance baseline based on the GAP benchmark suite.   We also introduced
a notation for graph algorithms expressed in terms of linear algebra.  We hope this
notation will lead to a consensus-notation the 
larger ``Graphs as Linear Algebra'' community might adopt.

This paper defines the foundation for our future work on the LAGraph project.  
We plan to explore Python wrappers for LAGraph that work well for data analytics workflows.  
In addition to the GAP benchmarks, which focus on graph algorithms, we will  
investigate end-to-end workflows based on the LDBC Graphalytics benchmark~\cite{DBLP:journals/pvldb/IosupHNHPMCCSAT16}.

Algorithmically we see a number of research directions to pursue.   With end-to-end workflows, the performance
of data ingestion heavily impacts performance.  We are interested in improving data ingestion performance
by exploiting a CPU's SIMD instructions~\cite{DBLP:journals/vldb/LangdaleL19}.  We are also interested in how  
LAGraph maps onto GPUs using versions of the GraphBLAS optimized for GPUs.

\section*{Acknowledgements}

D.~Bader was supported in part by NSF CCF-2109988 and NVIDIA (NVAIL Award).
G.~Sz\'arnyas was partially supported by the SQIREL-GRAPHS NWO project.
T.~Davis was supported by NSF CNS-1514406, NVIDIA, Intel, MIT Lincoln Lab,
Redis Labs, and IBM.
This material is also based upon work funded and supported by the Department of
Defense under Contract No.~FA8702-15-D-0002 with Carnegie Mellon University for
the operation of the Software Engineering Institute, a federally funded research
and development center [DM21-0298].




%\clearpage
\bibliographystyle{IEEEtranS}
\bibliography{ms}

\end{document}

% TODOs for the camera-ready
% - make sure page numbering is turned off
% - check whether GraphBLAS v2.0 is out (maybe cross-cite other GrAPL paper)
% - check whether cited CoRR papers (PIUMA, GraphBLAST, FPGA survey, ...) have been accepted

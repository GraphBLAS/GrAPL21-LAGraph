\documentclass[conference]{IEEEtran}
%\IEEEoverridecommandlockouts
% The preceding line is only needed to identify funding in the first footnote. If that is unneeded, please comment it out.

\usepackage[hyphens]{url}
\usepackage{hyperref} 
\usepackage{graphicx}
\usepackage{amsmath}
\usepackage{amssymb}

\usepackage{etoolbox}
\newbool{colored}
\IfFileExists{./colored}{\booltrue{colored}}{\boolfalse{colored}}
\newbool{ascii}
\IfFileExists{./ascii}{\booltrue{ascii}}{\boolfalse{ascii}}

\usepackage{subcaption}
\usepackage{xspace}
\usepackage{booktabs}
\usepackage{enumitem}
\usepackage{numprint}
\usepackage{multirow}
\usepackage[dvipsnames]{xcolor}
%\usepackage{todonotes}
% \setuptodonotes{inline}
\newcommand{\todo}[1]{{\bf TODO: #1}}

\usepackage{tikz}
\usetikzlibrary{positioning,fit,arrows.meta,backgrounds}

% (Gabor) I usually write with frenchspacing on but turned it off now.
% Caveat: With nonfrenchspacing, comments in algorithm2e use double-spacing after ':', '.', etc. which looks bad.
% \frenchspacing

% (Scott) I usually use the minted package for code blocks configured as follows
% You put code inside \begin{cplus} \end{cplus}
\usepackage{minted}
\setminted{fontsize=\footnotesize,baselinestretch=.97,linenos,frame=lines,xleftmargin=6pt,numbersep=3pt,mathescape=true,escapeinside=||,bgcolor=bg}
\usemintedstyle{default}
\definecolor{bg}{rgb}{0.97,0.97,0.97}
\newminted{cpp}{}
\newenvironment{cplus}{\VerbatimEnvironment\begin{cppcode}}{\end{cppcode}}
\newmintinline[cplusinl,mathescape]{cpp}{}



% TODO: remove in submitted version
\thispagestyle{plain}\pagestyle{plain}

\usepackage[ruled,vlined,linesnumbered]{algorithm2e}

\SetKwProg{Fn}{Function}{}{end}
\SetCommentSty{itshape}

\SetKwComment{Comment}{\color{green!100}// }{}
\renewcommand{\CommentSty}[1]{\tt \color{green!100}#1}

\DontPrintSemicolon
\SetKwRepeat{Do}{do}{while}
\SetKw{KwDownto}{downto}
\SetKw{Continue}{continue}
\SetKw{Break}{break}

% https://tex.stackexchange.com/a/260697
\newcommand{\lstnumberautorefname}{Line}


\newcommand{\ie}{i.e.,\@\xspace}
\newcommand{\Ie}{I.e.,\@\xspace}
\newcommand{\eg}{e.g.,\@\xspace}
\newcommand{\Eg}{E.g.,\@\xspace}
\newcommand{\etal}{et al.\@\xspace}
\newcommand{\etc}{etc.\@\xspace}
\newcommand{\vs}{vs.\@\xspace}
\newcommand{\viz}{viz.\@\xspace} % videlicet
\newcommand{\cf}{cf.\@\xspace} % confer
\newcommand{\Cf}{Cf.\@\xspace}
\newcommand{\wrt}{w.r.t.\@\xspace} % with respect to
\newcommand{\approximately}{approx.\@\xspace}

\newcommand{\yedscale}{0.445}

\definecolor{blue}{HTML}{365E7D}
\definecolor{green}{HTML}{697D36}
\definecolor{lilac}{HTML}{7D3662}
\definecolor{orange}{HTML}{ff7f00}
\definecolor{brown}{HTML}{a65628}
\definecolor{pink}{HTML}{f781bf}

\newcommand{\algo}[1]{\textcolor{blue}{\textit{#1}}}

\hyphenation{Suite-Sparse}
\hyphenation{Graph-BLAS}
\hyphenation{Suite-Sparse-Graph-BLAS}

\newcommand{\suitesparse}{SuiteSparse\xspace}
\newcommand{\grb}{GraphBLAS\xspace}
\newcommand{\ssgrb}{SuiteSparse:GraphBLAS\xspace}
\newcommand{\gxb}{\ssgrb}
\newcommand{\lagraph}{LAGraph\xspace}
\newcommand{\pygrb}{pygraphblas\xspace}
\newcommand{\grblas}{grblas\xspace}

% Define new boolean flags using etoolbox ('\newbool' is similar to '\newtoggle').
% This workaround is needed as simply putting the newcommands inside 'IfFileExists' did not do the job
% as it broke with 'Illegal parameter number in definition of \reserved@a', a symptom probably caused
% by the lack of protection (\protect). Anyways, the workaround is actually cleaner.

%\newcommand{\grbreduce}[2]{\left[{#1}_j \, {#2}(:, j) \right]}
\ifbool{ascii}{ % ASCII mode ======================================================================
    \newcommand{\grbm}[1]{{\ifbool{colored}{\color{brown}}{}{\mathtt{#1}}}}% matrix
    \newcommand{\grbv}[1]{{\ifbool{colored}{\color{lilac}}{}{\mathtt{#1}}}}% vector
    \newcommand{\grba}[1]{{\ifbool{colored}{\color{gray}}{}{\mathtt{#1}}}}% array
    \newcommand{\grbs}[1]{{\ifbool{colored}{\color{blue}}{}{\mathtt{#1}}}}% scalar

    \newcommand{\grbstr}[1]{{\{#1\}}}
    \newcommand{\grbmask}[1]{<\! #1 \!>}
    \newcommand{\grbmaskreplace}[1]{<\!<\! #1 \!>\!>}
    \newcommand{\grbneg}{\texttt{!}}
    \newcommand{\grbassign}{\mathrel{\texttt{<-}}}
    \newcommand{\grbf}[2]{\grboperation{#1}(#2)}
    \newcommand{\grbreduce}[4]{[ {#1 #3} ]} % omit the indices
    \newcommand{\grbt}{\texttt{'}} % transpose
    \newcommand{\grbdiv}{\grbbinaryop{DIV}}
    \newcommand{\grbminus}{\grbbinaryop{MINUS}}
    \newcommand{\grbaccum}{\texttt{\ensuremath{+}}}
    \newcommand{\grbaccumeq}[1]{\mathbin{\texttt{\ensuremath{\ifstrempty{#1}{\grbaccum}{#1}=}}}}

    \newcommand{\grbplus}{\grbbinaryop{+}}
    \newcommand{\grbtimes}{\grbbinaryop{\times}}
    \newcommand{\grbapply}{\grbbinaryop{\odot}}

    \newcommand{\grbfrac}[2]{(#1)/(#2)}

    \newcommand{\grbbool}{\mathtt{bool}} % booleans
    \newcommand{\grbuint}{\mathtt{uint}} % unsigned integers
    \newcommand{\grbint}{\mathtt{int}}   % integers
    \newcommand{\grbfloat}{\mathtt{fp}}  % floats (?)

    \newcommand{\grbplaceholder}[1]{\mathsf{#1}}

    \newcommand{\grbscalartype}[2]{\mathtt{#1#2()}}
    \newcommand{\grbvectortype}[3]{\mathtt{#1#2(#3)}}
    \newcommand{\grbmatrixtype}[4]{\mathtt{#1#2(#3, #4)}}

    \newcommand{\grbnewscalar}[3]{\mathtt{#1 = \grbscalartype{#2}{#3}}}
    \newcommand{\grbnewvector}[4]{\mathtt{#1 = \grbvectortype{#2}{#3}{#4}}}
    \newcommand{\grbnewmatrix}[5]{\mathtt{#1 = \grbmatrixtype{#2}{#3}{#4}{#5}}}

    \newcommand{\grbalpha}{\mathtt{alpha}}
    \newcommand{\grboperator}[1]{\mathtt{#1}}

    \newcommand{\grbrange}[2]{#1:#2}
    \newcommand{\grbdontcare}{\_}

    \newcommand{\grboperationnoarg}[1]{\mathtt{#1}}

    \newcommand{\grbewiseadd}[1]{\ #1[intersection]\ }
    \newcommand{\grbewisemult}[1]{\ #1[union]\ }
}{ % LaTeX mode ===================================================================================
    \newcommand{\grbm}[1]{{\ifbool{colored}{\color{brown}}{}{\mathbf{#1}}}}% matrix
    \newcommand{\grbv}[1]{{\ifbool{colored}{\color{lilac}}{}{\mathbf{#1}}}}% vector
    %\newcommand{\grba}[1]{{\ifbool{colored}{\color{gray}}{}{\boldsymbol{#1}}}}% array
    \newcommand{\grba}[1]{{\ifbool{colored}{\color{gray}}{}{\textbf{\textit{#1}}}}}% array
    \newcommand{\grbs}[1]{{\ifbool{colored}{\color{blue}}{}{\mathit{#1}}}}% scalar

    \newcommand{\grbmask}[1]{\langle #1 \rangle}
    %\newcommand{\grbstr}[1]{{\{#1\}}}
    \newcommand{\grbstr}[1]{s(#1)}
    %\newcommand{\grbmaskreplace}[1]{\langle\!\langle #1 \rangle\!\rangle}
    \newcommand{\grbmaskreplace}[1]{\langle #1, \mathrm{r} \rangle}
    \newcommand{\grbneg}{\neg}

    % use the \mapsfrom symbol extracted from the stix package as suggested in https://tex.stackexchange.com/a/331899/71109
    \DeclareFontEncoding{LS1}{}{}
    \DeclareFontSubstitution{LS1}{stix}{m}{n}
    \DeclareSymbolFont{arrows1}{LS1}{stixsf}{m}{n}
    \global\let\mapsfrom\undefined % undefine \mapsfrom because some templates such as acmart already have it
    \DeclareMathSymbol{\mapsfrom}{\mathrel}{arrows1}{"AB}
    \newcommand{\grbassign}{\mapsfrom}

    \newcommand{\grbf}[2]{\grboperation{#1}{#2}}
    \newcommand{\grbreduce}[4]{[ {#1}_{#2}\, #3(#4) ]}
    \newcommand{\grbt}{^\mathsf{T}} %^{\top}} % transpose
    \newcommand{\grbdiv}{\grbbinaryop{\oslash}}
    \newcommand{\grbminus}{\grbbinaryop{\ominus}}
    \newcommand{\grbaccum}{\texttt{\ensuremath{\odot}}}
    \newcommand{\grbaccumeq}[1]{\mathbin{\ensuremath{\ifstrempty{#1}{\grbaccum}{#1}\!\!=}}}

    \newcommand{\grbplus}{\oplus}
    \newcommand{\grbtimes}{\otimes}
    \newcommand{\grbapply}{\odot}
    
    \newcommand{\grbfrac}[2]{\frac{#1}{#2}}

    \newcommand{\grbbool}{\mathbb{B}}  % booleans
    \newcommand{\grbuint}{\mathbb{N}}  % unsigned integers
    \newcommand{\grbint}{\mathbb{Z}}   % integers
    \newcommand{\grbfloat}{\mathbb{Q}} % floats (?)
    
    \newcommand{\grbplaceholder}[1]{\mathsf{#1}}

    \newcommand{\grbscalartype}[2]{#1_{#2}}
    \newcommand{\grbvectortype}[3]{#1_{#2}^{#3}}
    \newcommand{\grbmatrixtype}[4]{#1_{#2}^{#3 \times #4}}

    \newcommand{\grbnewscalar}[3]{\text{let: } #1 \in \grbscalartype{#2}{#3}}
    \newcommand{\grbnewvector}[4]{\text{let: } #1 \in \grbvectortype{#2}{#3}{#4}}
    \newcommand{\grbnewmatrix}[5]{\text{let: } #1 \in \grbmatrixtype{#2}{#3}{#4}{#5}}

    \newcommand{\grbalpha}{\alpha}
    \newcommand{\grboperator}[1]{\mathsf{#1}}

    \newcommand{\grbrange}[2]{#1 \! : \! #2}
    \newcommand{\grbdontcare}{\textvisiblespace}

    \newcommand{\grboperationnoarg}[1]{\mathrm{#1}}

    \newcommand{\grbewiseadd}[1]{#1_\cup}
    \newcommand{\grbewisemult}[1]{#1_\cap}
}

% do not lange/rangle for tuples as it is already used for masks
% do not use grbtuple for the time being
%\newcommand{\grbtuple}[1]{( #1 )}
\newcommand{\tuple}[1]{(#1)}

% trying to avoid too much syntax (e.g. using wedge/vee symbols for LAND/LOR)
%\newcommand{\grblorland}{\lor\!.\!\land}

\newcommand{\grbsemiringops}[2]{\mathbin{\grboperator{#1.#2}}}
\newcommand{\grbplustimes}{\grbsemiringops{\grbplus}{\grbtimes}}

\newcommand{\grbanypair}{\grbsemiringops{any}{pair}}
\newcommand{\grbanyfirst}{\grbsemiringops{any}{first}}
\newcommand{\grbanysecond}{\grbsemiringops{any}{second}}
\newcommand{\grbanyfirstj}{\grbsemiringops{any}{firstj}}
\newcommand{\grbanyfirstjone}{\grbsemiringops{any}{firstj1}}
\newcommand{\grbminfirstj}{\grbsemiringops{min}{firstj}}
\newcommand{\grbminfirstjone}{\grbsemiringops{min}{firstj1}}
\newcommand{\grbanysecondi}{\grbsemiringops{any}{secondi}}
\newcommand{\grbanysecondione}{\grbsemiringops{any}{secondi1}}
\newcommand{\grbminsecondi}{\grbsemiringops{min}{secondi}}
\newcommand{\grbminsecondione}{\grbsemiringops{min}{secondi1}}
\newcommand{\grblorland}{\grbsemiringops{lor}{land}}
\newcommand{\grbminplus}{\grbsemiringops{min}{plus}}
\newcommand{\grbmaxplus}{\grbsemiringops{max}{plus}}
\newcommand{\grbmaxfirst}{\grbsemiringops{max}{first}}
\newcommand{\grbminfirst}{\grbsemiringops{min}{first}}
\newcommand{\grbminsecond}{\grbsemiringops{min}{second}}
\newcommand{\grbmaxsecond}{\grbsemiringops{max}{second}}
\newcommand{\grbsecondmin}{\grbsemiringops{second}{min}}
\newcommand{\grbsecondmax}{\grbsemiringops{second}{max}}
\newcommand{\grbarithmeticplustimes}{\grbsemiringops{plus}{times}} % not necessary because this is the default

\newcommand{\grbbinaryop}[1]{\mathop{\grboperator{#1}}}
\newcommand{\grbany}{\grbbinaryop{any}}
\newcommand{\grbpair}{\grbbinaryop{pair}}
\newcommand{\grbland}{\grbbinaryop{land}}
\newcommand{\grblor}{\grbbinaryop{lor}}
\newcommand{\grbband}{\grbbinaryop{band}}
\newcommand{\grbbor}{\grbbinaryop{bor}}
\newcommand{\grbmin}{\grbbinaryop{min}}
\newcommand{\grbmax}{\grbbinaryop{max}}
\newcommand{\grbfirst}{\grbbinaryop{first}}
\newcommand{\grbsecond}{\grbbinaryop{second}}
\newcommand{\grbfirsti}{\grbbinaryop{firsti}}
\newcommand{\grbsecondi}{\grbbinaryop{secondi}}
\newcommand{\grbfirstione}{\grbbinaryop{firsti1}}
\newcommand{\grbsecondione}{\grbbinaryop{secondi1}}
\newcommand{\grbfirstj}{\grbbinaryop{firstj}}
\newcommand{\grbsecondj}{\grbbinaryop{secondj}}
\newcommand{\grbfirstjone}{\grbbinaryop{firstj1}}
\newcommand{\grbsecondjone}{\grbbinaryop{secondj1}}
\newcommand{\grbarithmeticplus}{\grbbinaryop{plus}} % usually not necessary because this is the default addition
\newcommand{\grbarithmetictimes}{\grbbinaryop{times}} % usually not necessary because this is the default multiplication
\newcommand{\grbisne}{\grbbinaryop{isne}}
\newcommand{\grbplustext}{\grbbinaryop{plus}}
\newcommand{\grbtimestext}{\grbbinaryop{times}}

\newcommand{\grbgenericop}{\grboperator{op}}

% boolean values
\newcommand{\grbbooleanvalue}[1]{\mathtt{#1}}
\newcommand{\grbtrue}{\grbbooleanvalue{TRUE}}
\newcommand{\grbfalse}{\grbbooleanvalue{FALSE}}
\newcommand{\grbT}{\grbbooleanvalue{T}}
\newcommand{\grbF}{\grbbooleanvalue{F}}
\newcommand{\grbstring}{\textrm{String}}
\newcommand{\grbdate}{\textrm{Date}}

% cardinality / count
\newcommand{\grbcnt}[1]{| #1 |}

% operations
\newcommand{\grboperation}[2]{\grboperationnoarg{#1}(#2)}
\newcommand{\grbnrows}[1]{\grboperation{nrows}{#1}}
\newcommand{\grbncols}[1]{\grboperation{ncols}{#1}}
\newcommand{\grbnvals}[1]{\grboperation{nvals}{#1}}
\newcommand{\grbclear}[1]{\grboperation{clear}{#1}}
\newcommand{\grbdiag}[1]{\grboperation{diag}{#1}}
\newcommand{\grbselect}[1]{\grbmask{#1}}
\newcommand{\grbkron}[1]{\grboperation{kron}{#1}}
\newcommand{\grbtril}[1]{\grboperation{tril}{#1}}
\newcommand{\grbtriu}[1]{\grboperation{triu}{#1}}
\newcommand{\grbondiag}[1]{\grboperation{ondiag}{#1}}
\newcommand{\grboffdiag}[1]{\grboperation{offdiag}{#1}}

\newcommand{\grbind}[1]{\mathrm{ind}(#1)}

\usepackage{microtype}
\renewcommand*\ttdefault{txtt}

\usepackage{listings}

\usepackage{textcomp}
\lstset{upquote=true}


\begin{document}

%\title{LAGraph: A Graph Algorithm and Network Analysis Library for GraphBLAS}
\title{LAGraph: Linear Algebra, Network Analysis Libraries, and the Study of Graph Algorithms}
\author{ % Tim D modified the ordering: Gabor, then alphabetical
\IEEEauthorblockN{
    G\'abor Sz\'arnyas\IEEEauthorrefmark{5},
    David A. Bader\IEEEauthorrefmark{6},
    Timothy A. Davis\IEEEauthorrefmark{1},
    James Kitchen\IEEEauthorrefmark{3}, \\
    Timothy G. Mattson\IEEEauthorrefmark{2},
    Scott McMillan\IEEEauthorrefmark{4},
    Erik Welch\IEEEauthorrefmark{3}
} \\
\IEEEauthorblockA{
    \IEEEauthorrefmark{1}Texas A\&M University
}
\IEEEauthorblockA{
    \IEEEauthorrefmark{2}Parallel Computing Labs, Intel Corporation, Ocean Park, WA
}
\IEEEauthorblockA{
    \IEEEauthorrefmark{3}Anaconda, Inc.
}
\IEEEauthorblockA{
    \IEEEauthorrefmark{4}
    Software Engineering Institute, Carnegie Mellon University, Pittsburgh, PA
}
\IEEEauthorblockA{
    \IEEEauthorrefmark{5}CWI Amsterdam, The Netherlands
}
\IEEEauthorblockA{
    \IEEEauthorrefmark{6}New Jersey Institute of Technology
}
}

\maketitle

\begin{abstract}

Graphs and graph algorithms can be expressed in terms of 
linear algebra. The GraphBLAS are a library of low level building blocks for such algorithms.
The GraphBLAS target algorithm \emph{developers}.  The LAGraph project
targets graph algorithm \emph{users} with high-level algorithms common in network
analysis.   In this paper, we describe 
the first release of the LAGraph library.  We describe the design decisions behind the
library, the contents of the library, and performance data using the GAP benchmark suite.
LAGraph, however, is much more than a library development project.  It is also a
project to document and analyze the full range of algorithms enabled by the GraphBLAS.  To that end
we have developed a compact (and hopefully intuitive) notation for describing
these algorithms.  In this paper, we present that 
notation with examples from the GAP benchmark suite.  

\end{abstract}

\begin{IEEEkeywords}
Graph Processing, Graph Algorithms, Graph Analytics, Linear Algebra, GraphBLAS
\end{IEEEkeywords}

\section{Introduction}
\label{sec:introduction}

LAGraph is a library of Graph Algorithms based on the GraphBLAS

Key contributions:

\begin{itemize}
    \item document design decisions for LAGraph
    \item present a concise notation for \grb algorihms
    \item algorithms of the GAP benchmark suite~\cite{DBLP:journals/corr/BeamerAP15} used in the IISWC benchmark paper~\cite{DBLP:conf/iiswc/AzadABBCDDDDFGG20}
    \item improve data ingestion performance, e.g. using SIMD techniques~\cite{DBLP:journals/vldb/LangdaleL19}
\end{itemize}

Recently, numerous graph-specific have targeted GPUs such as the \mbox{GraphBLAS Template Library (GBTL)}~\cite{7529957}, Gunrock~\cite{DBLP:journals/topc/WangPDWYWOYLRO17} and \mbox{GraphBLAST}~\cite{DBLP:journals/corr/abs-1908-01407}, and FPGAs~\cite{DBLP:journals/corr/abs-1903-06697}.

However, in the near future we expect even more heterogeneous hardware architectures including graph-specific hardware based on the Programmable Integrated Unified Memory Architecture (PIUMA)~\cite{DBLP:journals/corr/abs-2010-06277}.
Additionally, graph processing workloads can be offloaded to  machine learning accelerators, \eg
Tensor Processing Units (TPUs)~\cite{DBLP:conf/isca/JouppiYPPABBBBB17},
systolic arrays using reconfigurable dataflow architecture~\cite{SambaNova},
sparse linear algebra-based deep learning accelerators~\cite{Cerebras}.

Previous \grb design papers:
theory~\cite{DBLP:conf/hpec/MattsonBBBDFFGGHKLLPPRSWY13},
C API~\cite{DBLP:conf/hpec/MattsonYMBM17},
C++ API~\cite{DBLP:conf/ipps/BrockBMMM20},
distributed API~\cite{DBLP:conf/ipps/BrockBMMMPSS20},
LAGraph~\cite{DBLP:conf/ipps/MattsonDKBMMY19}

\begin{listing}[h]
\begin{cplus}
// This is a minted code block inside a float region
int main() {
    return 0; // return zero
}
\end{cplus}
\caption{Example 2}
\end{listing}



\footnote{A non peer-reviewed comparison of 6 popular graph algorithms libraries is available at
\url{https://www.timlrx.com/blog/benchmark-of-popular-graph-network-packages-v2}.}


%TimM removed this figure.  We are short on space and this doesn't add enough to justify the space it consumes.
%\tikzset{
    module/.style={%
        draw,
        minimum width=50mm,
        minimum height=8mm,
        font=\sffamily,
        node distance=0mm,
        },
    module/.default=0.5cm,
    >=LaTeX
}
\begin{figure}[htb]
    \centering
    \begin{tikzpicture}
        \node[module, minimum height=16mm, text depth=8mm] (I1) {Graph application};
        \node[module, below=of I1, minimum width=30mm, xshift=10mm, yshift=8mm] (I2) {LAGraph};
        \node[module, below=of I1, minimum width=60mm] (I3) {GraphBLAS API};
        \node[module, below=of I3] (I4) {GraphBLAS implementation};
        \node[module, below=of I4] (I5) {Hardware architecture};
    \end{tikzpicture}
    \caption{Separation of concerns using the \grb API.}
    \label{fig:architecture}
\end{figure}


\section{Design decisions}
\label{sec:decisions}

\todo{Jim}

We investigate the following design questions:

\begin{itemize}
    \item easy mode, expert mode
    \item multi-threadable
    \item data structure for representing a graph/matrix
    \item error handling
    \item opacity of LAGraph
\end{itemize}


\section{GraphBLAS Theory and Notation}
\label{sec:notation}

%\setlength{\tabcolsep}{1.9pt}
%\renewcommand\arraystretch{0.85}

\begin{table*}[htbp]
    \centering
    \begin{tabular}{llr@{}ll}
        \toprule
        \multicolumn{1}{c}{\bf operation/method} & \multicolumn{1}{c}{\bf name}                        & \multicolumn{2}{c}{\bf notation}                                                                       & \multicolumn{1}{c}{\bf comment}                                                     \\
        % <operations>
        \midrule
        \tt mxm                                  & matrix-matrix multiplication                        & $\grbm{C} \grbmask{\grbm{M}}        $                                                                  & $\grbaccumeq{} \grbm{A} \grbplustimes \grbm{B}$                                     \\
        \tt vxm                                  & vector-matrix multiplication                        & $\grbv{\grbv{w}} \grbmask{\grbv{m}} $                                                                  & $\grbaccumeq{} \grbv{u} \grbplustimes \grbm{A}$                                     \\
        \tt mxv                                  & matrix-vector multiplication                        & $\grbv{w} \grbmask{\grbv{m}}        $                                                                  & $\grbaccumeq{} \grbm{A} \grbplustimes \grbv{u}$                                     \\
        \midrule
        \multirow{2}{*}{\tt eWiseAdd}            & element-wise addition                               & $\grbm{C} \grbmask{\grbm{M}} $                                                                         & $\grbaccumeq{} \grbm{A} \grbewiseadd{\grbgenericop} \grbm{B}$                       \\
                                                 & set union of patterns                               & $\grbv{w} \grbmask{\grbv{m}} $                                                                         & $\grbaccumeq{} \grbv{u} \grbewiseadd{\grbgenericop} \grbv{v}$                       \\
        \midrule
        \multirow{2}{*}{\tt eWiseMult}           & element-wise multiplication                         & $\grbm{C} \grbmask{\grbm{M}} $                                                                         & $\grbaccumeq{} \grbm{A} \grbewisemult{\grbgenericop} \grbm{B}$                      \\
                                                 & set intersection of patterns                        & $\grbv{w} \grbmask{\grbv{m}} $                                                                         & $\grbaccumeq{} \grbv{u} \grbewisemult{\grbgenericop} \grbv{v}$                      \\
        \midrule
        \multirow{4}{*}{\tt extract}             & extract submatrix                                   & $\grbm{C} \grbmask{\grbm{M}} $                                                                         & $\grbaccumeq{} \grbm{A}(\grba{i}, \grba{j})$                                        \\
                                                 & extract column vector                               & $\grbv{w} \grbmask{\grbv{m}} $                                                                         & $\grbaccumeq{} \grbv{A}(:, \grbs{j})$                                               \\
                                                 & extract row vector                                  & $\grbv{w} \grbmask{\grbv{m}} $                                                                         & $\grbaccumeq{} \grbv{A}(\grbs{i}, :)$                                               \\
                                                 & extract subvector                                   & $\grbv{w} \grbmask{\grbv{m}} $                                                                         & $\grbaccumeq{} \grbv{u}(\grba{i})$                                                  \\
        \midrule
        \multirow{4}{*}{\tt assign}              & assign matrix to submatrix with mask for $\grbm{C}$ & $\grbm{C} \grbmask{\grbm{M}} (\grba{i},\grba{j}) $                                                     & $\grbaccumeq{} \grbm{A}$                                                            \\
                                                 & assign scalar to submatrix with mask for $\grbm{C}$ & $\grbm{C} \grbmask{\grbm{M}} (\grba{i},\grba{j}) $                                                     & $\grbaccumeq{} \grbs{s}$                                                            \\
                                                 & assign vector to subvector with mask for $\grbv{w}$ & $\grbv{w} \grbmask{\grbv{m}} (\grba{i}) $                                                              & $\grbaccumeq{} \grbv{u}$                                                            \\
                                                 & assign scalar to subvector with mask for $\grbv{w}$ & $\grbv{w} \grbmask{\grbv{m}} (\grba{i}) $                                                              & $\grbaccumeq{} \grbs{s}$                                                            \\
        % \midrule
        % \multirow{4}{*}{\tt subassign (GxB)} & assign matrix to submatrix with submask for $\grbm{C}(\grba{i},\grba{j})$ & $\grbm{C}(\grba{i},\grba{j}) \grbmask{\grbm{M}} $                                                      & $\grbaccumeq{} \grbm{A}$                                                            \\
        %                                      & assign scalar to submatrix with submask for $\grbm{C}(\grba{i},\grba{j})$ & $\grbm{C}(\grba{i},\grba{j}) \grbmask{\grbm{M}} $                                                      & $\grbaccumeq{} \grbs{s}$                                                            \\
        %                                      & assign vector to subvector with submask for $\grbv{w}(\grba{i})$          & $\grbv{w}(\grba{i}) \grbmask{\grbv{m}} $                                                               & $\grbaccumeq{} \grbv{u}$                                                            \\
        %                                      & assign scalar to subvector with submask for $\grbv{w}(\grba{i})$          & $\grbv{w}(\grba{i}) \grbmask{\grbv{m}} $                                                               & $\grbaccumeq{} \grbs{s}$                                                            \\
        \midrule
        \multirow{2}{*}{\tt apply}               & \multirow{2}{*}{apply unary operator}               & $\grbm{C} \grbmask{\grbm{M}} $                                                                         & $\grbaccumeq{} \grbf{f}{\grbm{A}, \grbs{k}}$                                        & \multirow{2}{*}{$k$: thunk} \\
                                                 &                                                     & $\grbv{w} \grbmask{\grbv{m}} $                                                                         & $\grbaccumeq{} \grbf{f}{\grbv{u}, \grbs{k}}$                                                 \\
        \midrule
        \multirow{2}{*}{\tt select}              & \multirow{2}{*}{apply select operator}              & $\grbm{C} \grbmask{\grbm{M}} $                                                                         & $\grbaccumeq{} \grbm{A}\grbselect{\grbf{f}{\grbm{A}, \grbs{k}}}$                    & \multirow{2}{*}{$k$: thunk} \\
                                                 &                                                     & $\grbv{w} \grbmask{\grbv{m}} $                                                                         & $\grbaccumeq{} \grbv{u}\grbselect{\grbf{f}{\grbv{u}, \grbs{k}}}$                    \\
        \midrule
        \multirow{3}{*}{\tt reduce}              & reduce matrix to column vector                      & $\grbv{w} \grbmask{\grbv{m}} $                                                                         & $\grbaccumeq{} \grbreduce{\grbplus}{\grbs{j}}{\grbm{A}}{:,\grbs{j}}$                \\
                                                 & reduce matrix to scalar                             & $\grbs{s} $                                                                                            & $\grbaccumeq{} \grbreduce{\grbplus}{\grbs{i}, \grbs{j}}{\grbm{A}}{\grbs{i},\grbs{j}}$ \\
                                                 & reduce vector to scalar                             & $\grbs{s} $                                                                                            & $\grbaccumeq{} \grbreduce{\grbplus}{\grbs{i}}{\grbm{u}}{\grbs{i}}$                  \\
        \midrule
        \multirow{1}{*}{\tt transpose}           & transpose                                           & $\grbm{C} \grbmask{\grbm{M}} $                                                                         & $\grbaccumeq{} \grbm{A}\grbt$                                                       \\
        % \midrule
        % \tt kronecker                        & Kronecker multiplication                                                  & $\grbm{C} \grbmask{\grbm{M}}$                                                                          & $\grbaccumeq{} \grbkron{\grbm{A}, \grbm{B}}$                                        \\
        \midrule\midrule
        % </operations>
        % <methods>
        \multirow{2}{*}{\tt new}                 & new matrix                                          & \multicolumn{2}{l}{$\grbnewmatrix{\grbm{A}}{\grbplaceholder{TYPE}}{\grbplaceholder{PRECISION}}{n}{m}$}                                                                                       \\
                                                 & new vector                                          & \multicolumn{2}{l}{$\grbnewvector{\grbv{u}}{\grbplaceholder{TYPE}}{\grbplaceholder{PRECISION}}{n}$}                                                                                          \\
        \midrule
        \multirow{2}{*}{\tt build}               & matrix from tuples                                  & $\grbm{C}\ $                                                                                           & $\grbassign \left\{ \grba{i}, \grba{j}, \grba{x} \right\} $                         \\
                                                 & vector from tuples                                  & $\grbv{w}\ $                                                                                           & $\grbassign \left\{ \grba{i}, \grba{x} \right\} $                                   \\
        \midrule
        \multirow{2}{*}{\tt extractTuples}       & \multirow{2}{*}{extract index/value arrays}         & $ \left\{ \grba{i}, \grba{j}, \grba{x} \right\} $                                                      & $\grbassign \grbm{A} $                                                              \\
                                                 &                                                     & $ \left\{ \grba{i}, \grba{x} \right\} $                                                                & $\grbassign \grbv{u}   $                                                            \\
        \midrule
        \multirow{2}{*}{\tt dup}                 & duplicate matrix                                    & $\grbm{C} $                                                                                            & $\grbassign \grbm{A}$                                                               \\
                                                 & duplicate vector                                    & $\grbv{w} $                                                                                            & $\grbassign \grbv{u}$                                                               \\
        \midrule
        \multirow{2}{*}{\tt extractElement}      & \multirow{2}{*}{extract scalar element}             & $\grbs{s} $                                                                                            & $\ = \grbm{A}(\grbs{i}, \grbs{j})$                                                  \\
                                                 &                                                     & $\grbs{s} $                                                                                            & $\ = \grbv{u}(\grbs{i})$                                                            \\
        \midrule
        \multirow{2}{*}{\tt setElement}          & \multirow{2}{*}{set element}                        & $\grbm{C}(\grbs{i}, \grbs{j}) $                                                                        & $\ = \grbs{s}$                                                                      \\
                                                 &                                                     & $\grbv{w}(\grbs{i})$                                                                                   & $\ = \grbs{s}$                                                                      \\
        \bottomrule
        % </methods>
    \end{tabular}
    \caption{GraphBLAS operations and methods based on \cite{DBLP:journals/toms/Davis19}.
        \emph{Notation:}
        Matrices and vectors are typeset in bold, starting with uppercase ($\grbm{A}$) and lowercase ($\grbv{u}$) letters, respectively.
        Scalars including indices are lowercase italic ($\grbs{k}$, $\grbs{i}$, $\grbs{j}$) while arrays are lowercase bold italic ($\grba{x}$, $\grba{i}$, $\grba{j}$).
        $\grbplus$ and $\grbtimes$ are the addition and multiplication operators forming a semiring and default to conventional arithmetic $+$ and $\times$ operators.
        $\grbaccum$ is the accumulator operator.
        \todo{do we need everything, e.g. subassign, Kronecker?}
        \todo{matrices can be tranposed through descriptors, denoted with e.g. $\grbm{B}\grbt$}
        \todo{masks}
    }
    \label{tab:graphblas-notation}
\end{table*}


In this section, we summarize the key concepts in \grb, then present a concise notation for the operations and methods defined in the \grb standard.
Additionally, we demonstrate how the operations can be interpreted as graph processing primitives if vectors are used to encode nodes in a graph and the matrices are adjacency matrices.\footnote{We use these vectors/matrices for illustration purposes. The \grb standard allows the definition of arbitrary vectors/matrices.}

\subsection{Overview}

We first give a brief overview of the theoretical aspects of the \grb. For more details, we refer the reader to tutorials~\cite{gabor_szarnyas_2020_4318870} and the specification documents~\cite{GraphBLASv13,GxBUserGuide}.

\paragraph{Data structures}
\grb builds on the duality between the graph and matrix data structures.
Namely, a directed graph $G = (V, E)$ can be represented with an \emph{adjacency matrix} $\grbm{A} \in \grbbool^{|V| \times |V|}$ where $\grbm{A}_{i,j} = \grbtrue$ iff $(v_i, v_j) \in E$.
The adjacency matrices used in \grb algorithms are not necessarily square: \eg induced subgraphs, where source nodes are selected from $V_1$ and target nodes are selected from $V_2$ ($V_1, V_2 \subseteq V$), can be represented with $\grbm{A} \in \grbbool^{|V_1| \times |V_2|}$.
\emph{Vectors} are used to encode data for nodes, \eg $\grbv{u} \in \grbbool^{|V|}$ can be used to select a subset of nodes.
Vectors and matrices can be defined over different types, \eg a nonnegative integer ($\grbuint$) matrix can encode the number of paths between two nodes, while a floating point ($\grbdouble$) matrix can encode edge weights.

The transposition of~$\grbm{A} \in D^{n \times m}$ is denoted with $\grbm{A}\grbt \in D^{m \times n}$ where $\grbm{A}\grbt(i, j) = \grbm{A}(j, i)$.  Compared to $\grbm{A}$, matrix~$\grbm{A}\grbt$ contains the edges in the reverse direction.
For vectors, $\grbm{u}$ denotes a column vector and $\grbm{u}\grbt$ denotes a row vector.

In practice, the adjacency matrices representing graphs are sparse, \ie most of their elements are \emph{zeros}, lending themselves to compressed representations such as CSR/CSC.
The \emph{zero elements} take their values during operations based on the identity value of the semiring's $\grbplus$ operation (see below).

%We refer to the graph represented by adjacency matrix~$\grbm{A}$ as ``the graph of $\grbm{A}$''.

\paragraph{Semirings}
\grb uses matrix operations %(\autoref{sec:operations})
to express graph processing primitives, \eg a matrix-vector multiplication $\grbm{A} \grbplustimes \grbm{u}$ finds incoming neighbors of the set of nodes selected by vector~$\grbv{u}$ in the graph represented by adjacency matrix~$\grbm{A}$.
\grb allows users to perform the multiplication operations over an arbitrary \emph{semiring}.
In general, the multiplication operator $\grbtimes$ is used for combining the values of matching input elements, while the addition operator $\grbplus$ defines how the results should be summarized.
For example, the $\grbminplus$ semiring uses $\grbplustext$ as the multiplication operator to compute the path length and $\grbmin$ as the addition operator to determine the length of the shortest path.
The algorithms presented in this paper use a number of non-conventional semirings such as $\grbanysecondi$, $\grbplusfirst$, and $\grbpluspair$. These are summarized in \autoref{tab:semirings} and defined in \autoref{sec:evaluation}.

\begin{table}
    \centering
    \begin{tabular}{llllr}
        \toprule
        \multicolumn{1}{c}{name} & \multicolumn{1}{c}{$\grbplus$} & \multicolumn{1}{c}{$\grbtimes$} & \multicolumn{1}{c}{$D$} & \multicolumn{1}{c}{zero} \\ \midrule
        conventional             & $\grbplustext$                 & $\grbtimestext$                 & $\grbuint$              & $0$                      \\
        $\grbanysecondi$         & $\grbany$                      & $\grbsecondi$                   & $\grbuint$              & $0$                      \\
        $\grbminplus$            & $\grbmin$                      & $\grbplustext$                  & $\grbdouble$            & $-\infty$                \\
        $\grbplusfirst$          & $\grbplustext$                 & $\grbfirst$                     & $\grbuint$              & $0$                      \\
        $\grbplussecond$         & $\grbplustext$                 & $\grbsecond$                    & $\grbuint$              & $0$                      \\
        $\grbpluspair$           & $\grbplustext$                 & $\grbpair$                      & $\grbuint$              & $0$                      \\
        \bottomrule
    \end{tabular}
    \caption{Semirings used in this paper}
    \label{tab:semirings}
\end{table}


\paragraph{Masks and accumulators}
All \grb operations whose output is a vector or a matrix allow the use of masks to limit the scope of the computation. %? ~\cite{DBLP:conf/ipps/AzadBG15}
The semantics of the masks are that the computation should be performed
on a given set of nodes (for vector masks) or
on a given set of edges (for matrix masks).
The accumulator operator $\grbaccum$ is a binary operator that determines how the result of an operation should be applied to its output.
The interplay of masks and the accumulators is discussed in the specifications~\cite{GraphBLASv13,GxBUserGuide}.

\paragraph{Notation}
To present our algorithms, we use the mathematical notation given in \autoref{tab:graphblas-notation}.
Matrices and vectors are typeset in bold, starting with uppercase ($\grbm{A}$) and lowercase ($\grbv{u}$) letters, respectively.
Scalars including indices are lowercase italic ($\grbs{k}$, $\grbs{i}$, $\grbs{j}$) while arrays are lowercase bold italic ($\grba{x}$, $\grba{i}$, $\grba{j}$).

\subsection{Operations}
\label{sec:operations}

\paragraph{Matrix multiplication}
\label{sec:mxm}

The \emph{matrix-matrix multiplication} operation $\grbm{A} \grbplustimes \grbm{B}$ expresses a navigation step that starts
in the edges of $\grbm{A}$ and traverses from their endpoints
using the edges of $\grbm{B}$.
The result matrix~$\grbm{C}$ contains paths with $\grbm{C}_{i,j}$ containing the summarized paths with start node $i$ in the graph of $\grbm{A}$ and end node $j$ in the graph of $\grbm{B}$.

The \emph{vector-matrix multiplication} operation $\grbv{u}\grbt \grbplustimes \grbm{A}$ performs navigation starting from the nodes selected in vector~$\grbv{u}$ among the edges of matrix~$\grbm{A}$.
The result vector~$\grbm{w}$ contains the set of reached nodes with the values computed on the semiring (combining the source node values with the outgoing edge values using $\grbtimes$ then summarizing these for each target node using $\grbplus$).
The \emph{matrix-vector multiplication} operation $\grbm{A} \grbplustimes \grbv{u}$ performs navigation in the reverse direction of the edges of $\grbm{A}$.


\paragraph{Element-wise addition}

The \emph{element-wise addition} operation
$\grbm{A} \grbewiseadd{\grbgenericop} \grbm{B}$
applies the operator $\grbgenericop$ on the elements selected by the \emph{union of the structures of its inputs},
\ie nodes/edges which are present in at least one of the input matrices.
Operation $\grbm{u} \grbewiseadd{\grbgenericop} \grbm{v}$ works similarly for vectors.
%For matrices, the operation is denoted with $\grbm{A} \grbewiseadd{\grbgenericop} \grbm{B}$,
%for vectors, it is denoted with $\grbv{u} \grbewiseadd{\grbgenericop} \grbv{v}$.

\paragraph{Element-wise multiplication}

The \emph{element-wise multiplication} operation $\grbm{A} \grbewisemult{\grbgenericop} \grbm{B}$ applies the operator $\grbgenericop$ on the elements selected by the \emph{intersection of the structures of its inputs},
\ie nodes/edges which are present in both inputs.
Operation $\grbm{u} \grbewisemult{\grbgenericop} \grbm{v}$ works similarly for vectors.

\paragraph{Extract}
For adjacency matrix~$\grbm{A}$,
the \emph{extract submatrix} operation $\grbm{A}(\grba{i}, \grba{j})$ returns a matrix containing the elements from $\grbm{A}$ with
row indices in $\grba{i}$ and
column indices in $\grba{j}$.
In graph terms, the submatrix represents an the induced subgraph where
the source nodes of the edges are in array $\grba{i}$ and
the target nodes of the edges are in array $\grba{j}$.
The \emph{extract vector} operation $\grbv{A}(\grbs{i}, :)$ selects a column vector containing node $\grbs{i}$'s neighbors along incoming edges.
The \emph{extract subvector} operation $\grbv{u}(\grba{i})$ selects the nodes with indices in array $\grba{i}$.

%with an optional mask $\grbmask{\grbm{M}}$ to limit the computation to certain edges.

\paragraph{Assign}
The \emph{assign} operation has multiple variants.
The first one assigns a matrix to a submatrix selected by row indices $\grba{i}$ and column indices $\grba{j}$:
$\grbm{C} \grbmask{\grbm{M}} (\grba{i},\grba{j}) \grbaccumeq{} \grbm{A}$.
This operator is useful e.g. to ``project'' an induced subgraph back to the original graph.
The second one assigns a vector to a selected subvector selected by indices $\grba{i}$:
$\grbv{w} \grbmask{\grbv{m}} (\grba{i}) \grbaccumeq{} \grbv{u}$.
Finally, both the selected submatrix/subvector can be assigned with a scalar value:
$\grbm{C} \grbmask{\grbm{M}} (\grba{i},\grba{j}) \grbaccumeq{} \grbs{s}$ and
$\grbv{w} \grbmask{\grbv{m}} (\grba{i}) \grbaccumeq{} \grbs{s}$.
In all cases, the scope of the assignment can be further constrained using masks (see \autoref{sec:masks}).

%\paragraph{Subassign} ?

\paragraph{Apply and select}
The \emph{apply} and \emph{select} operations evaluate a unary operator $f$ with an optional input (thunk) $k$ on all elements of the input matrix/vector. When evaluated on a given element, function $\mathit{f}$ can access the index or indices of the element, allowing the operation to be constrained on e.g. the lower triangle of a matrix.
In the case of \emph{apply}, denoted with $\grbf{f}{\grbm{A}, \grbs{k}}$ and $\grbf{f}{\grbv{u}, \grbs{k}}$, the resulting elements are returned as part of the output.
The \emph{select} operation requires $f$ to be a boolean function and zeros out elements that return $\grbfalse$.
$\grbm{A}\grbselect{\grbf{f}{\grbm{A}, \grbs{k}}}$ and $\grbv{u}\grbselect{\grbf{f}{\grbv{u}, \grbs{k}}}$ filter on the edges in matrix $\grbm{A}$ and the nodes in vector $\grbv{u}$, respectively.

\paragraph{Reduce}
For adjacency matrix~$\grbm{A}$,
the \emph{row-wise reduction} $\grbv{w} \grbmask{\grbv{m}} \grbaccumeq{} \grbreduce{\grbplus}{\grbs{j}}{\grbm{A}}{:,\grbs{j}}$ represents a summarization of the values on outgoing edges for each node (represented by row vector~$\grbm{A}(:, \grbs{j})$) to vector~$\grbv{w}$ %with an optional mask $\grbmask{\grbv{m}}$ to limit the computation to certain nodes.
For matrix~$\grbm{A}$, the \emph{reduction to scalar} $\grbs{s} \grbaccumeq{} \grbreduce{\grbplus}{\grbs{i}, \grbs{j}}{\grbm{A}}{\grbs{i},\grbs{j}}$ represents a summarization of all edge values.
For vector~$\grbv{u}$, the \emph{reduction to scalar} $\grbs{s} \grbaccumeq{} \grbreduce{\grbplus}{\grbs{i}}{\grbm{u}}{\grbs{i}}$ represents a summarization of all node values.

\paragraph{Transposition}
Transposition can be applied as a standalone \grb operation $\grbm{C} \grbmask{\grbm{M}} \grbaccumeq{} \grbm{A}\grbt$ %with an optional mask $\grbmask{\grbm{M}}$
and also to the input/output matrices of operations, for example:
$$\grbm{C}^{[\grbtransposesymbol]} \grbmask{\grbm{M}} \grbaccumeq{} \grbm{A}^{[\grbtransposesymbol]} \grbplustimes \grbm{B}^{[\grbtransposesymbol]}$$

%\paragraph{Kronecker}
%omitted to keep the paper concise

\subsection{Masks}
\label{sec:masks}

% \todo{explain replace/merge as per Scott's email}
% REPLACE: C :=              M .* (C + AB)
% MERGE:   C := (!M .* C) U [M .* (C + AB)]
% \todo{revise Scott, Tim D, and Tim M}

Masks are used by all \grb operations to limit the scope of the computation \wrt the output of the operation.
Namely, the mask prescribe that the computation only needs to be performed on the elements selected by the mask ($\grbm{m}$) or the complement of these elements ($\grbneg\grbm{m}$).

There are variations based on
(1)~how the elements are selected (\emph{valued} masks take the element values into consideration, while \emph{structural} masks only use the structure (pattern) of the mask
(2)~how the elements outside the selected ones are treated (they are \emph{replaced} with implicit zeros or they are kept intact and \emph{merged} with the results of the computation).

By default, masks use \emph{merge} semantics, \ie the computation can only effect elements selected by the mask, elements outside the mask are unaffected
$\grbv{w}\grbmask{\grbv{m}}$.
%$\grbv{w}\grbmask{\grbneg \grbv{m}}$

If \emph{replace} semantics is actived, masks annihilate all elements outside the mask. This is denoted with
$\grbv{w}\grbmaskreplace{\grbv{m}}$.
%$\grbv{w}\grbmaskreplace{\grbneg \grbv{m}}$

By default, masks are \emph{valued}, \ie values in the mask are checked and elements with zero values are not used considered to be part of the mask.
To only consider the pattern of the mask, \emph{structural masks}: $\grbv{w}\grbmask{\grbstr{\grbv{m}}}$
%$\grbv{w}\grbmask{\grbneg \grbstr{\grbv{m}}}$

Combining \emph{replace semantics} and \emph{structural masks} is possible and is denoted as $\grbv{w}\grbmaskreplace{\grbstr{\grbv{m}}}$
%$\grbv{w}\grbmaskreplace{\grbneg \grbstr{\grbv{m}}}$




\subsection{Methods}

% methods don't use masks

% Initializing scalars, vectors, and matrices (GraphBLAS methods):

% \begin{itemize}
%     \item $\grbnewscalar{\grbs{s}}{\grbfloat}{64}$
%     \item $\grbnewvector{\grbv{w}}{\grbfloat}{32}{n}$
%     \item $\grbnewmatrix{\grbm{A}}{\grbuint}{16}{m}{n}$
%     \item $\grbnewmatrix{\grbm{A}}{\grbint}{64}{k}{m}$
% \end{itemize}

% \begin{tabular}{ll}
%     $\grbv{w}\grbmask{\grbv{m}}$                 & compute for nodes selected by the values of $\grbv{m}$ & keep       \\
%     $\grbv{w}\grbmaskreplace{\grbv{m}}$          & compute for nodes selected by the values in $\grbv{m}$ & discard  \\
%     $\grbv{w}\grbmask{\grbstr{\grbv{m}}}$        & compute for nodes selected by the pattern of $\grbv{m}$ & keep      \\
%     $\grbv{w}\grbmaskreplace{\grbstr{\grbv{m}}}$ & compute for nodes selected by the pattern of $\grbv{m}$ & discard \\
% \end{tabular}


\section{Algorithms}
\label{sec:algorithms}

GAP algorithms: BFS, SSSP, TC, BC, PR. Not sure whether CC should be included.

% The PR in GAP does not work well if there are dangling vertices in the graph.
% The Graphalytics benchmark has a PageRank variant which avoids this problem~\cite{DBLP:journals/corr/abs-2011-15028}.


\section{Utility Fuctions}
\label{sec:utility}

LAGraph includes a set of utility functions that operate
on a graph.  All function names are prefixed with \verb'LAGraph_'
so we exclude that prefix in the names below, for brevity.

% Here is a rough categorization of the utilities (not all included yet)
\begin{itemize}

\item {\bf Graph Properties:}
    % If they operate on the LAGraph\_Graph cached properties consider a
    % consistent naming scheme like LAGraph\_Property\_XXX
    An \verb'LAGraph_Graph' includes cached properties which can be
    assigned by Basic methods, or which are required by Advanced methods.

    % \begin{itemize}
      % \item
      \verb'DeleteProperties' clear all properties,
      % \item
      \verb'Property_AT' computes the transpose of the adjacency matrix \verb'G->A',
      % \item
      \verb'Property_RowDegree' computes the row degrees of \verb'G->A',
      % \item
      \verb'Property_ColDegree' computes the column degrees of \verb'G->A',
      % \item
      and
      \verb'Property_ASymmetricPattern' determines if the pattern of \verb'G->A' is symmetric or unsymmetric.
       %   (consider Property\_ClearAll), Property\_AT, Property\_AssymetricPattern, Property\_RowDegree, Property\_ColDegree
    % \end{itemize}

\item {\bf Display and debug:}
    \verb'CheckGraph' checks the validity of a graph.
    Since the graph is not opaque, a user application is able to change a graph
    arbitrarily and thus might make it an invalid object.
    \verb'DisplayGraph' displays a graph and its properties.

\item {\bf Memory management:}
    Wrappers for \verb'malloc', \verb'calloc', \verb'realloc', and \verb'free',
    allowing a user application to select the memory manager to be used.
    These default to the ANSI C11 library functions.
    % malloc, calloc, realloc, free stuff.  This might be covered in the design
    % decisions section

% \item {\bf Threading:}
    % Get/SetNumThreads. Shouldn’t this be part of init, or an
    % LAGraph\_Context?  Should this also be discussed in design decisions.
    % GraphBLAS threading is one thing, but this seems to be LAGraph threading
    % (outside of GraphBLAS calls).  I could see this as algorithm specific and
    % not a general util.  The only way to leverage these inside algorithms is
    % to either set a global property that all algorithms have access to, or
    % creating a context that is passed to algorithms.

\item {\bf Graph I/O:}
    % \begin{itemize}
    % \item
    \verb'BinRead' and \verb'BinWrite' read/write a \verb'GrB_Matrix' in binary form.
    % \item
    \verb'MMRead' and \verb'MMWrite' read/write a \verb'GrB_Matrix' in Matrix Market form.
    % , MMRead (we are missing BinWrite, MMWrite), DisplayGraph (is this a
    % pretty print of the matrix or all the properties, this should also take
    % FILE*)
    % \end{itemize}

\item {\bf Matrix operations:}
    \verb'Pattern' returns a boolean matrix containing the pattern of a matrix.
    % Pattern (MakePattern, GetPattern).  Not sure how to categorize these
% \item {\bf Matrix comparison}:
    \verb'IsEqual' determines if two matrices are equal.  It selects the appropriate
    \verb'GrB_EQ_T' operator that matches the matrix type, and then calls \verb'IsAll'.
    \verb'IsAll' compares two matrices and returns false if
    the pattern of the two matrices differ.  It then uses a given comparator operator to
    compare all pairs of entries, and returns true if all comparisons return true.
    % IsEqual, IsAll. Are these graph or matrix utils? IsAll
    % is an ambiguous (maybe misleading) name because it seems to be a generic
    % comparator (consider CompareGraphs).

\item {\bf Degree operations:}
    % SortByDegree, SampleDegree  (maybe belongs grouped here as it is computing properties, but maybe not cached)
    \verb'SortByDegree' returns a permutation that sorts a graph by its row/column degrees, and
    \verb'SampleDegree' computes a quick estimate of the mean and median row/column degrees.

\item {\bf Error handling:}
    \verb'LAGraph_TRY' and \verb'GrB_TRY' are helper macros for a simple try/catch
    mechanism.  They require the user application to define \verb'LAGraph_CATCH'
    and \verb'GrB_CATCH'.
    % TRY/CATCH, MIN/MAX, tic/toc, TypeName, KindName

\item {\bf Other:}
    \verb'TypeName' returns a string with the name of a \verb'GrB_Type'.
    \verb'KindName' returns a string with of graph kind (directed or undirected).
    \verb'Tic' and \verb'Toc' provide a portable timer.
    \verb'Sort1', \verb'Sort2', and \verb'Sort3' sort 1, 2, or 3 integer arrays.

% \item {\bf Consider for removal} Things that may be implementation detail and could be buried:

% \begin{itemize}

%  \item Sort1/2/3 - currently only Sort2 is used.  I don't see a strong need to include these as part of the public API at this time.

%   \item Random15/60 - I see Random64 being the most widely usable. These can easily be buried as well, and I would still suggest Random64 (Random60 seems too SuiteSparse:GraphBLAS specific.

%   \item Test\_ReadProblem (move to the TestArea)

% \end{itemize}
\end{itemize}



\section{Evaluation}
\label{sec:evaluation}

The performance of LAGraph can only be considered in context of an
implementation of the underlying GraphBLAS library.  This is discussed in
Section~\ref{sec:extensions}, followed by performance results of the new
LAGraph API on the 6 algorithms in the GAP Benchmark
\cite{DBLP:conf/sc/BeamerAP12}.

\subsection{SuiteSparse Extensions}
\label{sec:extensions}

In a prior paper (\cite{DBLP:conf/iiswc/AzadABBCDDDDFGG20}), an early draft of
SuiteSparse:GraphBLAS v4.0.0 (Aug 2020) was compared with the GAP benchmark
\cite{DBLP:conf/sc/BeamerAP12} and four other graph libraries.  This prior
version of SS:GrB included two primary data structures for its sparse matrices:
compressed sparse vector, and a hypersparse variant
\cite{DBLP:conf/ipps/BulucG08}, both held by row or by column.  It included a
draft implementation of a bitmap data structure that could only be used in a
prototype breadth-first search.  Since then, SuiteSparse:GraphBLAS v4.0.3 has
been released, with full support for bitmap and full matrices for all its
operations.  In an $m$-by-$n$ bitmap matrix, the values are held in a full
array of size $mn$, and another \verb'int8_t' array of size $mn$ holds the
sparsity pattern of the matrix.  A full matrix is a simple dense array of size
$mn$.

The bitmap format is particularly important for the ``pull'' phase of an
algorithm, as used in direction-optimizing breadth-first-search
\cite{DBLP:conf/sc/BeamerAP12,DBLP:conf/icpp/YangBO18}.  The GAP benchmark suite uses this method by
holding its frontier as a bitmap in the pull step and as a list in the push
step. The GAP BFS was shown to be typically the fastest BFS amongst the 6 graph
libraries compared in \cite{DBLP:conf/iiswc/AzadABBCDDDDFGG20} (for 4 of the 5
benchmark graphs).  With the addition of the bitmap format to SS:GrB,
LAGraph+SS:GrB is able to come within a factor of 2 or so of the performance of
the highly-tuned BFS GAP benchmark (see the results in the next section), for
those 4 graphs.  At the same time, however, the BFS is very easily expressed in
LAGraph as an easy-to-read and easy-to-write code.  This enables non-experts to
obtain a reasonably high level of performance with modest programming effort
when writing their own graph algorithms.

Direction-optimization is incredibly simple to add to an LAGraph algorithm.
For example, a batched direction-optimizing betweenness-centrality (BC)
algorithm in LAGraph only requires a simple heuristic to determine which
direction to use, followed by masked matrix-matrix multiplication with the
matrix or its transpose: $\grbm{F} \grbmask{\grbneg \grbstr{\grbm{P}}} = \grbm{F}\grbm{B}\grbt$ (the pull) or $\grbm{F}
\grbmask{\grbneg \grbstr{\grbm{P}}} = \grbm{F} \grbm{A}$ (the push), where $A$ is the adjacency matrix of
the graph and $\grbm{B} = \grbm{A}\grbt$ is its explicit transpose, $\grbm{F}$ is the frontier, and the
complemented structural mask $\grbneg \grbstr{\grbm{P}}$ is the set of unvisited nodes.  The multiplication
$\grbm{F} \grbm{B}\grbt$ relies on the descriptor to represent the transpose of $\grbm{B}$, which is not
explicitly transposed.  In the backward phase, the pull step is $\grbm{W} = \grbm{W} \grbm{A}\grbt$ while
the push is $\grbm{W} = \grbm{W} \grbm{B}$, where $\grbm{W}$ is the 4-by-$n$ matrix in which centrality is
accumulated.

Additional optimizations added to SS:GrB in the past year include a {\em lazy
sort}.  Normally, SS:GrB keeps its vectors sorted (row vectors in a CSR matrix,
or column vectors if the matrix is held by column), with entries sorted in
ascending order of column or row index, respectively.  This simplifies the many
algorithms that operate on a \verb'GrB_Matrix'.  However, some algorithms
naturally produce a jumbled result (matrix multiply in particular), while many
algorithms are tolerant of jumbled input matrices.  We thus allow the sort to
be left pending.  The lazy sort joins two other kinds of pending work in
SS:GrB: {\em pending tuples} and {\em zombies}~\cite{DBLP:journals/toms/Davis19}.
A pending tuple is an entry
that is held inside a matrix in an unsorted list, awaiting insertion into the
CSR/CSC format of a \verb'GrB_Matrix'.  A zombie is the opposite: it is an
entry in the CSR/CSC format that has been marked for deletion, but has not yet
been deleted from the matrix.  With the lazy sort, the sort is postponed until
another algorithm requires sorted input matrices.  If the sort is lazy enough,
it might never occur, which is the case for the LAGraph BFS and BC.

Another useful addition to SS:GrB is the new positional binary operators,
such as the $\grbanysecondi$ %ANY-SECONDI
used in the BFS.

\subsection{Performance Results}

Our benchmark environment is an NVIDIA DGX Station (donated to Texas A\&M by
NVIDIA in support of this research).  It includes a 20-core Intel(R) Xeon(R)
CPU E5-2698 v4 @ 2.20GHz, with 40 threads.  All codes were compiled with gcc
5.4.0 (-O3).  All default settings were used, which means that hyperthreading
was enabled.  The system has 256GB of RAM in a single socket (no NUMA effects).
LAGraph (Feb 7, 2021) and SuiteSparse:GraphBLAS 4.0.4-draft (Feb 7, 2021) were
used.  The NVIDIA DGX Station includes four P100 GPUs, but no GPUs were used by
this experiment (a GPU-accelerated SS:GrB is in progress).
Table~\ref{table:results} lists the run time (in seconds) for the GAP benchmark
and LAGraph+SS:GrB for the 6 algorithms on the 5 benchmark matrices.
The benchmark matrices are listed in Table~\ref{table:matrices}.

\begin{table}
\begin{center}
\begin{tabular}{|l|rrrrr|}
\hline
Algorithm :    &   \multicolumn{5}{c|}{graph, with run time in seconds}  \\
 package       &   Kron    &   Urand   &   Twitter  &  Web    &    Road  \\
\hline
BC   : GAP     &  31.52    &  46.36    &  10.82     &  3.01   &    1.50  \\
% BC : Jan26   &  26.85    &  31.78    &  10.04     &  9.25   &   51.91  \\
% BC : Feb6    &  24.75    &  30.66    &   9.28     &  8.97   &   43.66  \\
BC   : SS      &  24.52    &  30.69    &   9.11     &  8.43   &   34.06  \\     % Feb7
\hline
BFS  : GAP     &    .31    &    .58    &    .22     &   .34   &     .25  \\
%BFS : Jan26   &    .52    &   1.31    &    .33     &   .67   &    3.33  \\
%BFS : Feb6    &    .52    &   1.20    &    .33     &   .66   &    3.36  \\
BFS  : SS      &    .52    &   1.22    &    .33     &   .66   &    3.32  \\     % Feb7
\hline
PR   : GAP     &  19.81    &  25.29    &  15.16     &  5.13   &    1.01  \\
%PR  : Jan26   &  21.96    &  27.75    &  17.22     &  9.30   &    1.34  \\
%PR  : Feb6    &  21.67    &  27.60    &  17.14     &  9.34   &    1.32  \\
PR   : SS      &  22.17    &  27.71    &  17.21     &  9.30   &    1.34  \\     % Feb7
\hline
CC   : GAP     &    .53    &   1.66    &    .23     &   .22   &     .05  \\
%CC  : Jan26   &   3.42    &   4.59    &   1.48     &  1.97   &    1.00  \\
%CC  : Feb6    &   3.35    &   4.56    &   1.47     &  1.96   &     .97  \\
CC   : SS      &   3.36    &   4.47    &   1.47     &  1.97   &     .98  \\     % Feb7
\hline
SSSP : GAP     &   4.91    &   7.23    &   2.02     &   .81   &     .21  \\
%SSSP: Jan26   &  17.62    &  25.62    &   8.44     &  9.67   &   48.49  \\
%SSSP: Feb6    &  17.58    &  25.60    &   8.18     &  9.60   &   48.24  \\
SSSP : SS      &  17.37    &  25.54    &   8.54     &  9.61   &   46.79  \\     % Feb7
\hline
TC   : GAP     & 374.08    &  21.83    &  79.58     & 22.18   &     .03  \\
%TC  : Jan26   & 943.47    &  34.10    & 242.36     & 35.15   &     .29  \\
%TC  : Feb6    & 922.35    &  33.97    & 238.71     & 34.67   &     .23  \\
TC   : SS      & 917.99    &  34.01    & 239.58     & 34.65   &     .23  \\     % Feb7
\hline
\end{tabular}
\caption{Run time of GAP and LAGraph+SS:GrB
\label{table:results}}
\end{center}
\end{table}

\begin{table}
\begin{center}
\begin{tabular}{|l|rrr|}
\hline
graph   & nodes        & entries in $A$ & graph kind \\
\hline
Kron    & 134,217,726 &  4,223,264,644 &  undirected   \\
Urand   & 134,217,728 &  4,294,966,740 &  undirected   \\
Twitter &  61,578,415 &  1,468,364,884 &  directed     \\
Web     &  50,636,151 &  1,930,292,948 &  directed     \\
Road    &  23,947,347 &     57,708,624 &  directed     \\
\hline
\end{tabular}
\caption{Benchmark matrices\label{table:matrices}
(\url{https://sparse.tamu.edu/GAP})}
\end{center}
\end{table}

% Notes: TC in SS is SandiaDot method only

With the simple addition of the bitmap (needed for the pull step), the
push/pull optimization in BC resulted in a nearly 2x performance gain in the
GraphBLAS method for the largest matrices, as compared to the SS:GrB version
used for the results presented in \cite{DBLP:conf/iiswc/AzadABBCDDDDFGG20}.

With this change the BC method in LAGraph+SS:GrB is not only expressible in a
simple, elegant code, but it is also faster than the highly-tuned GAP benchmark
method, \verb'bc.cc', for the three largest matrices (1.3x for Kron, 1.5x for
Urand, and 1.2x for Twitter).

% Feb7 update:  the sort is completely lazy in BC.

% Feb6 results: the sort partially lazy in BC.

% For the Jan26 results:
% We expect the BC in LAGraph+SS:GrB to become faster still in the next
% release because we have not yet fully exploited the lazy sort.  The frontier
% matrix $\grbm{F}$ is left jumbled by the lazy sort, but it is sorted right away by a
% subsequent assignment.  Most uses of \verb'GrB_assign' are intolerant of
% jumbled input matrices, so it sorts them on input.  However, \verb'GrB_assign'
% includes about 40 internal variations, a few of which do not actually require
% sorted input matrices.  The particular method used in \verb'GrB_assign' in the
% LAGraph BC method is one of those methods, so this would be simple to exploit.
% With this change, the sort would be so lazy that it would {\em never} occur.
% The frontier would be computed, left jumbled, used in subsequent computations,
% and then recomputed (and thus discarded), just as we currently do in the
% LAGraph BFS.

With the addition of bitmap format (which makes push/pull optimization very
simple to express, and very fast) and the $\grbanysecondi$ %ANY-SECONDI
semiring, the BFS of a
directed or undirected graph is easily expressed in GraphBLAS, and has a
performance that is only about 1.5x to 2x slower than the GAP benchmark.  We expect
the remaining performance gap arises from two issues:

\begin{enumerate}
\item
The GAP assumes that the graph has fewer than $2^{32}$ nodes and edges, and
thus uses 32-bit integers throughout.  GraphBLAS is written for larger
problems, and thus relies solely on 64-bit integers.  This cannot be easily
changed in GraphBLAS, but rather than ``fixing'' GraphBLAS to use smaller
integers, it is the GAP algorithms that would need to be updated for larger
graphs in the future.  In the current GAP benchmark graphs, two graphs are
chosen with almost exactly 4 billion edges.  Graphs of current interest in
large data science can easily exceeed $2^{32}$ nodes and edges \cite{9286235}.

\item In GraphBLAS, the BFS must be expressed as two calls.  The first computes
$\grbv{q} \grbmask{\grbneg \grbv{p}} = \grbv{q}\grbt \grbm{A}$, and the second updates the parent vector,
$\grbv{p} \grbmask{\grbstr{\grbv{q}}} = \grbv{q}$:

{\footnotesize
\begin{verbatim}
  GrB_vxm (q, p, NULL, semiring, q, A, GrB_DESC_RSC) ;
  GrB_assign (p, q, NULL, q, GrB_ALL, n, GrB_DESC_S) ; \end{verbatim}}

In Scott Beamer's \verb'bfs.cc', these two steps are fused, and the
matrix-vector multiplication can write its result directly into the parent vector
\verb'p'.  This could be implemented in a future GraphBLAS library, since the
GraphBLAS API allows for a non-blocking mode where work is queued and done
later, thus enabling a fusion of these two steps.  SS:GrB exploits the
non-blocking mode (for its lazy sort, pending tuples, and zombies) but does not
{\em yet} exploit the fusion of \verb'GrB_vxm' and \verb'GrB_assign'.  We
intend to exploit this in the future.
\end{enumerate}

Note that for the Road graph,
LAGraph+SS:GrB is quite slow for all but PageRank (PR).
The primary reason for this is the high diameter of the Road graph
(about 6000).  This requires 6000 iterations of GraphBLAS in the BFS, each with
a tiny amount of work.  Each call to GraphBLAS does several \verb'malloc' and
\verb'free's, and in some cases the workspace must be initialized.  A future
version of SS:GrB is planned that will eliminate this work entirely, by
implementing an internal memory pool.  There may be other overheads, but we
hope that a memory pool, fusion to fully exploit non-blocking mode, and other
optimization will address this large performance gap for the Road graph for
these algorithms.

LAGraph+SS:GrB is also up to 3x slower than the GAP for the triangle counting
problem (for all but the Road graph, where it is even slower).  This
performance gap can be eliminated entirely in the future, if the \verb'GrB_mxm'
and \verb'GrB_reduce' are combined in a single fused step, by a full
exploitation of the GraphBLAS non-blocking mode.  The current method computes
$\grbm{C} \grbmask{\grbstr{\grbm{L}}} = \grbm{L}\grbm{U}\grbt$, followed by the reduction of $\grbm{C}$ to a single
scalar.  The matrix $\grbm{C}$ is then discarded.  All that GraphBLAS needs is a fused
kernel that does not explicitly instantiate the temporary matrix $\grbm{C}$.
This is permitted by the GraphBLAS C API Specification, but not yet implemented
in SS:GrB.



\section{Conclusion}
\label{sec:conclusion}

In this paper we introduced the LAGraph library, the rationale behind its design,  
and a performance baseline with the GAP benchmark suite.   We also introduced
a notation for graph algorithms expressed in terms of linear algebra which we hope becomes
a consensus-notation adopted by the 
larger ``Graphs as Linear Algebra'' community.

This paper defines the foundation for our future work on the LAGraph project.  
We plan to explore Python wrappers for LAGraph that work well for data analytics workflows.  
In addition to the GAP benchmark, which focuses on graph algorithms, we will  
investigate end-to-end workflows based on the LDBC Graphalytics benchmark~\cite{DBLP:journals/pvldb/IosupHNHPMCCSAT16}.

Algorithmically we see a number of research directions to pursue.   With end-to-end workflows, the performance
of data ingestion heavily impacts performance.  We are interested in improving data ingestion performance
by exploiting a CPU's SIMD instructions~\cite{DBLP:journals/vldb/LangdaleL19}.  We are also interested in how  
LAGraph maps onto GPUs using versions of the GraphBLAS optimized for GPUs.

\section*{Acknowledgements}

G.~Sz\'arnyas was supported by the SQIREL-GRAPHS NWO project.
D.~Bader was supported in part by NSF CCF-2109988 and NVIDIA (NVAIL Award).
T.~Davis was supported by NSF CNS-1514406, NVIDIA, Intel, MIT Lincoln Lab,
Redis Labs, and IBM.
This material is also based upon work funded and supported by the Department of
Defense under Contract No.~FA8702-15-D-0002 with Carnegie Mellon University for
the operation of the Software Engineering Institute, a federally funded research
and development center [DM21-0298].
References herein to any specific commercial product, process, or service by trade name, trade mark, manufacturer, or otherwise, does not necessarily constitute or imply its endorsement, recommendation, or favoring by Carnegie Mellon University or its Software Engineering Institute.



%\clearpage
\bibliographystyle{IEEEtranS}
\bibliography{ms}

\end{document}

% TODOs for the camera-ready
% - make sure page numbering is turned off
% - check whether GraphBLAS v2.0 is out (maybe cross-cite other GrAPL paper)
% - check whether cited CoRR papers (PIUMA, GraphBLAST, FPGA survey, ...) have been accepted

\section{Conclusion}
\label{sec:conclusion}

In this paper we have introduced the LAGraph library, the rationale behind our design decisions 
and have established a performance baseline based on the GAP benchmark suite.  This will 
allow us to track performance enhancements as the library evolves.  We also introduced
a notation for describing graph algorithms expressed in terms of linear algebra.  It is our hope that this
notation will lead to a lively discussion leading to a consensus notation the 
larger ``Graphs as Linear Algebra'' community might adopt.

This is very much a foundational paper to support our ongoing work one the LAGraph project.  
We plan to explore Python wrappers for LAGraph that work well with workflows common in the data analytics community.  
In addition to the GAP benchmark suite, which is focussed on a key set of graph algorithms, we will next 
investigate end-to-end workflows based on the LDBC Graphalytics benchmark~\cite{DBLP:journals/pvldb/IosupHNHPMCCSAT16}.

Algorithmically we see a number of research directions to pursue.   With end-to-end workflows, the performance
of data ingestion heavily impacts overall performance.  We are interested in improving data ingestion performance
by exploiting the CPUs SIMD instructions~\cite{DBLP:journals/vldb/LangdaleL19}.  We are also interested in how 
the LAGraph algorithms map onto the GPU using future versions of the GraphBLAS optimized for GPUs.